
%---------- Inleiding ---------------------------------------------------------

\section{Introductie} % The \section*{} command stops section numbering
\label{sec:introductie}Het democratisch stemmen is een cruciaal beslissingsmechanisme aanwezig in iedere laag van onze moderne samenleving, het vormt niet alleen de basis van het politieke systeem in vele landen, ook de bedrijfswereld is er van doordrongen. Stemprocessen zijn niet meer weg te denken uit de organisaties van vandaag. Of het nu gaat over besluiten op laag, functioneel niveau of over strategische beslissingen op hoog bestuurlijk niveau, de verantwoordelijkheid voor het maken van keuzes ligt steeds vaker bij een groep van mensen via een stemproces, dan bij één enkel individu. Het belang van stemmen op zowel economisch als politiek vlak valt dan ook  niet te onderschatten. De vraag die we onszelf echter moeten stellen is de volgende: hoe houdt men een proces waar zoveel van afhangt betrouwbaar, veilig en eerlijk? 
In de meeste gevallen blijkt het antwoord op die vraag vrij eenvoudig: waar het aantal participanten klein is, kan er gemakkelijk door iedere deelnemer of observator geverifieerd worden wat het resultaat van de stemming is. Iedere individuele deelnemer kan getuigen dat alles correct verloopt en dat maakt de kans op frauduleuze praktijken veel kleiner. Wordt het aantal participanten echter groter, dan is een dergelijk systeem onmogelijk. In zo'n geval bepaalt men het resultaat van de stemming door middel van een zogenaamde centrale autoriteit. Deze derde partij voert een controlerende functie uit en garandeert de betrouwbaarheid voor alle participanten. Bij nationale verkiezingen is dit bijvoorbeeld het stembureau. 
Het is van fundamenteel belang dat de centrale autoriteit neutraal is. Eén van de gevaren bij  dergelijke instanties is namelijk het risico op machtsmisbruik. Zo zijn voorafbepaalde of frauduleuze verkiezingen -hoewel bij ons gelukkig zeldzaam- niet ongekend. Een ander potentieel gevaar bij grote stemprocessen zijn onmoedwillige fouten. Er kunnen tal van menselijke of technische fouten optreden waardoor het resultaat van de stemming niet of niet geheel correct is. Een persoon die stemmen moet tellen kan fouten maken of een machine die stemmen moet registreren kan het laten afweten. Er is met andere woorden een probleem op het vlak van betrouwbaarheid. 
Een oplossing die een betrouwbaarder stemresultaat  garandeert lijkt de liggen in het digitaliseren van het proces. Bij digitalisatie krijgt men echter te maken met nieuwe risico's op het vlak van veiligheid, denk aan digitale fenomenen zoals: hacking, cyberaanvallen en identiteitsdiefstal. Toch zijn de voordelen die zijn verbonden aan digitalisatie immens: niet alleen op vlak van betrouwbaarheid, maar ook op vlak van efficiëntie, effectiviteit en gebruiksvriendelijkheid. Digitalisatie combineert zich echter zeer moeilijk met een centrale autoriteit: om het stemproces te digitaliseren zou men eigenlijk van het systeem van een centrale autoriteit moeten afstappen. 
Een mogelijkheid biedt zich aan in de vorm van de Blockchain technologie. Deze werd in feite net voor dit soort van problemen ontwikkeld. De oorsprong van deze technologie is namelijk gelinkt aan het ontstaan van 's werelds eerste gedecentraliseerde digitale munteenheid: de bitcoin. 
Net zoals dat bij het stemproces het geval is, wordt de legitimiteit van een reguliere munteenheid bepaald door een centrale autoriteit, in dit geval is dat de bank. Iedere (elektronische) transactie van geld gebeurt via deze autoriteit. De bitcoin is echter niet gelinkt aan een centrale bank, haar legitimiteit komt van een systeem dat blockchain wordt genoemd. De werking daarvan is in essentie niets meer dan de terugkeer naar een gedecentraliseerd systeem waarin iedere participant aan verificatie doet. 
Dit concept vormt de basis voor een systeem met een zeer hoog niveau van veiligheid en betrouwbaarheid. Omdat net dit de zaken zijn die bij stemprocessen van belang zijn, is een stemapplicatie op basis van blockchain zeker het onderzoeken waard.  De intentie van deze scriptie is dan ook antwoord te kunnen bieden op de volgende onderzoeksvragen:
\begin{itemize}
  \item Wat zijn de voor- en nadelen van de blockchain-technologie in het kader van een stemsysteem?
  \item Is er sprake van een onoverkomelijk schaalbaarheidsprobleem voor blockchain-technologie?
  \item Welke tools heeft men nodig om een blockchain-gebaseerd stemsysteem op te zetten en wat zijn de voor- en nadelen hiervan?
  \item Is een blockchain-gebaseerd stemsysteem haalbaar in de praktijk ?
\end{itemize}

%---------- Stand van zaken ---------------------------------------------------

\section{State-of-the-art}
\label{sec:state-of-the-art}

\subsection*{Blockchain 1.0: Bitcoin}
In dit eerste deel van de state-of-the-art wordt er een beeld geschetst van wat blockchain is en hoe het werkt achter de schermen. Dit wordt aan de hand van blockchain 1.0 gedaan, ofwel de originele blockchain-bitcoin-implementatie.
\subsubsection*{Wat is blockchain?}
 Blockchain is een gedecentraliseerde manier om gegevens op te slaan die voor het eerst geconceptualiseerd werd in de jaren negentig  ~\autocite{Dai1998} maar pas echt groot werd na de publicatie van ~\textcite{Nakamoto2008}. Het bijzondere aan de bitcoin-blockchain-implementatie is dat het peer-to-peer technologie combineert met wiskundige cryptografie, om tot een systeem te komen dat kan functioneren zonder de hulp van een vertrouwde derde partij  (centrale autoriteit) om de legitimiteit van iedere bitcoin te verifiëren. Blockchain werkt namelijk niet op basis van vertrouwen maar op basis van wiskundig bewijs.
 
 \subsubsection*{Hoe werkt het?}
Een blockchain is in datastructuur die bestaat uit een ketting van informatieblokken. Het is een ketting die fungeert als een soort publiek gedistribueerde boekhouding. Eenmaal informatie is opgeslagen in een blockchain, is het vrijwel onmogelijk om deze ongedetecteerd te wijzigen. Ieder blok van de ketting bestaat namelijk uit 3 velden: de data, de hash van het blok en de hash van het vorige blok. In het dataveld kan informatie worden opgeslagen. Het soort informatie dat men hier vindt is afhankelijk van de specifieke implementatie. In een bitcoin-blockchain wordt bijvoorbeeld een bitcoin-transacties opgeslagen. Ieder blok heeft daarnaast ook een hash, dit is een unieke identifier die wordt gegenereerd op basis van de inhoud van het blok. Iedere verandering aan het dataveld brengt ook een wijziging van de hashcode teweeg. Tenslotte is er nog het derde veld dat de hashcode van het vorige blok bijhoudt. Als iemand de data in één blok zou wijzigen, dan zou dit zorgen voor een aanpassing van de hash van dat blok. Het volgende blok bevat dan echter nog steeds een verwijzing naar de oorspronkelijke hash, die plots niet meer bestaat. De hashcode van het volgende blok kan dus niet  correct berekend worden. Ook de hashcodes van al de daarop volgende blokken kunnen bijgevolg niet meer correct berekend worden. Het gevolg is dus dat na een gecorrumpeerd blok ieder volgend blok ongeldig wordt. 

Dit systeem volstaat echter niet om te voorkomen dat aanpassingen worden gedaan. Moderne computers kunnen immers zodanig snel hashes berekenen, dat iemand die data in een bepaald blok aanpast mogelijks in no-time alle hashes van de daarop volgende blokken opnieuw zou kunnen berekenen, om zo de volledige blockchain weer valabel te maken, en dat alles voordat iemand dat zou kunnen opmerken. Om dit tegen te gaan werkt het systeem met een zogenaamde proof-of-work, dit is een mechanisme dat het berekenen van de hashes aanzienlijk lastiger en  langduriger maakt. In het geval van de bitcoin duurt het ongeveer 10 minuten om een hash te berekenen. Een nieuw blok met data aan de ketting toevoegen duurt dus ook 10 minuten, gezien  de hashcode voor het  nieuwe blok moet worden gevonden. Het zoeken van de correcte hash wordt gedaan door speciale nodes binnen het netwerk die miners worden genoemd. Miners specialiseren in het berekenen van hashcodes en kunnen wat verdienen wanneer ze de hash voor een bepaald blok als eerste kunnen vinden. In het geval van de bitcoin-blockchain verdienen miners bijvoorbeeld een bedrag in bitcoin. Dit hele systeem maakt het bijna onmogelijk om aanpassingen te doen aan reeds bestaande blokken. 

Blockchain heeft daarenboven nog een extra vorm van beveiliging. In plaats van een gecentraliseerde blockchain bij te houden, is het systeem gedistribueerd. Het maakt gebruik van een P2P-netwerk waarvan iedere node een eigen versie van de blockchain bijhoudt. Wanneer een nieuwe blok aan de blockchain wordt toegevoegd, valideert iedere node van het netwerk deze individuele blok. Blockchain werkt met een consensus van meer dan 50\%. Dat betekent dat een nieuwe blok enkel kan worden toegevoegd aan de blockchain wanneer meer dan de helft van de nodes deze valabel verklaart, als dit niet het geval is wordt de nieuwe blok afgewezen. Een dataveld wijzigen in een blockchain betekent in de praktijk dat men alle opeenvolgende hashes opnieuw moet berekenen, wat ontzettend lang duurt door de proof-of-work, en  gedurende  al die tijd moet men meer dan 50\% van de nodes in een netwerk controleren om een fake consensus te creëren bij iedere toevoeging aan de blockchain. Het spreekt voor zich dat dit een zeer moeilijke opgave is waardoor blockchain praktisch feilloos is op het vlak van veiligheid en betrouwbaarheid.

\subsection*{Blockchain 2.0}
In dit tweede deel van de state-of-the-art wordt een beeld geschetst van wat blockchain 2.0 inhoud volgens ~\textcite{Swan2015}. Concreet wordt er dieper ingegaan op het aspect Smart Contracts, vervolgens wordt er ook een overzicht gegeven van Ethereum.

\subsubsection*{Wat is zijn smart contracts?}
Smart contracts leiden de blockchain-technologie een stap verder. In ~\textcite{Swan2015} worden ze omschreven als gedecentraliseerde contracten die niet langer een autoriteit (zoals een rechtbank) nodig hebben. Het zijn in feite digitale contractprogramma's die zichzelf kunnen valideren en uitvoeren wanneer aan bepaalden voorwaarden is voldaan. Ook dit concept bestaat al sinds de jaren negentig ~\autocite{Szabo1996}. Bij het lezen van ~\textcite{Nakamoto2008} is het duidelijk dat er vanaf het prille begin van de bitcoin een visie was om een dergelijke systeem te implementeren. Blockchain-technologie staat immers niet alleen toe om data op te slaan, ook programma's kunnen in de blockchain worden opgeslagen. Smart contracts vormen de basis van de nieuwe Blockchain 2.0 van vandaag, zowat iedere grote crypto currency implementeert ze.

\subsubsection*{Wat is ethereum?}
Ethereum is een opensourceplatform dat werd opgericht in 2015. Net zoals bitcoin maakt het gebruik van een decentraal netwerk, gebaseerd op het oorspronkelijke blockchain-concept. Het valideren van informatie gebeurt ook hier door zogenaamde miners, het enige verschil met bitcoin is dat de miners worden beloond met de munteenheid ether in plaats van bitcoin. Ethereum kan men echter niet zien als een zuivere variant op de bitcoin of een andere crypto currency. Ethereum is veel meer dan dat, ~\textcite{Swan2015} omschrijft het als een "Turing-Complete Virtual Machine", die zowel een platform als een programmeertaal biedt voor het ontwikkelen en publiceren van gedistribueerde applicaties. Turing -compleetheid betekent in deze context dat het over een platform gaat dat het vermogen heeft om eender welke digitale munt, protocol of blockchain te ondersteunen.

\subsubsection*{Hoe werkt het?}
Zonder in teveel technisch detail te gaan, zou men kunnen stellen dat de werking van het ethereum-netwerk in grote lijnen dezelfde structuur volgt als de eerder omschreven Blockchain 1.0. In tegenstelling tot (relatief) eenvoudige bitcoin-transacties, bevat de data die bij Ethereum wordt opgeslagen in de blokken iets wat men zou kunnen omschrijven als een toestandsmachine, bestaande uit een lijst van allerhande transacties. Om misbruik en spamming van transacties tegen te gaan is aan iedere transactie een kleine kostprijs verbonden, in ether genaamd  \emph{gas}. Transacties in ethereum vinden plaats tussen twee gebruikers of tussen een gebruiker en een smart contract. Een transactie tussen een gebruiker en een smart contract kan één of meerdere nieuwe transacties vanuit het contract naar andere gebruikers doen ontstaan. Tenslotte kan een transactie van een gebruiker naar een smart contract ook transacties naar andere smart contracts triggeren, die dan op hun beurt hetzelfde doen en zo complexe kettingreactie creëren. De combinatie van mogelijke transacties en mogelijkheden met smart contracts zorgt volgens ~\textcite{Wood2017} voor een systeem waarop in theorie iedere toepassing mogelijk is. 
\subsection*{Blockchain 3.0}
\label{sec:wat-is-blockchain}

In dit derde en laatste deel van de state-of-the-art wordt eerst een overzicht gegeven van Blokchain 3.0. Vervolgens wordt een literatuurstudie gevoerd naar het concrete onderwerp van deze scriptie: namelijk het technisch opzetten van een klein stemsysteem op basis van een ethereum-blockchain.

\subsubsection*{Nieuwe toepassingen voor blokchain-technologie}

Platforms zoals ethereum  stellen de blockchain-technologie beschikbaar voor iedere ontwikkelaar. Het is nu niet langer een vereiste om een uitgebreide kennis van wiskundige cryptografie of gedistribueerde computersystemen te hebben om zo'n toepassing te ontwerpen. De impact van toekomstige blockchain-systemen is zeker niet te onderschatten. Blockchain zou niet alleen in stemprocessen een grote rol kunnen spelen, ook voor gezondheidszorg, justitie en anticensuur ontstaan implementaties. ~\textcite{Swan2015} stelt dat met het ontplooien van Blockchain 3.0 het potentieel zodanig groot is, dat er sprake zou zijn van een nieuw computer-paradigma, een technologische sprong die de samenleving op alle vlakken kan veranderen. Bronnen zoals ~\textcite{Wood2017} zijn iets minder optimistisch: hoewel zij de potentie van een wijdbeschikbare blockchain-technologie zeker niet ontkennen, blijven zij voorlopig sceptisch omwille van de schaalbaarheidsproblemen waarmee de technologie vandaag de dag kampt.

\subsubsection*{Literatuurstudie voor technische opzet}

De voornaamste bron voor het technische aspect van dit onderzoek is zonder enige twijfel ~\textcite{McCorry2017}. In deze paper werd een kleinschalig stemsysteem gerealiseerd op basis van de ethereum-blockchain. De implementatie telde 40 stemmers en werd getest op het officiële ethereum-testnetwerk. Er werd gekozen voor een implementatie op basis van een smart contract, de focus van ~\textcite{McCorry2017} lag op het uitwerken van een protocol dat het tellen van stemmen kan laten verlopen zonder een centrale autoriteit. Concreet gaat het hier dus over de implementatie van een zelftellend stemprotocol. Eén van de grote moeilijkheden tijdens het onderzoek was het garanderen van de anonimiteit van de stemmers. Er zijn voor een dergelijk systeem op het internet verscheidene andere implementaties vindbaar, maar op het vlak van anonimiteit  laten deze allemaal veel te wensen over. De opzet van dit onderzoek slaagde echter: het onderzoek resulteerde in het eerste decentraal online stemprotocol op basis van blockchain. Hoewel het protocol geen absolute anonimiteit bereikt, is er wel sprake van maximale privacy en afscherming van de gegevens van de gebruiker.

%---------- Methodologie ------------------------------------------------------
\section{Methodologie}
\label{sec:methodologie}

Dit onderzoek zal op gelijkaardige manier te werk gaan als het ~\textcite{McCorry2017} onderzoek. Aan de hand van een technische implementatie zal een praktische gids worden opgesteld met best practices, voorkomende problemen, valkuilen, enz. Een belangrijk verschil met het genoemde onderzoek is dat in dit onderzoek niet op voorhand assumpties worden gemaakt over de haalbaarheid  wat betreft de schaalbaarheid, en het zich dus niet tot dezelfde schaal zal beperken als ~\textcite{McCorry2017}. De hoop is dat -na bijna 2 jaar van verbeteringen en updates van de technologie- nu meer zaken mogelijk zijn.

%---------- Verwachte resultaten ----------------------------------------------
\section{Verwachte resultaten}
\label{sec:verwachte_resultaten}

Het verwachte resultaat van dit onderzoek is dat er een praktische gids kan worden uitgewerkt voor het opzetten van een blockchain-gebaseerd stemsysteem dat draait op ethereum. Daarnaast verwacht dit onderzoek ook een kleine bijdrage te kunnen leveren door een kleine verbetering aan te brengen op het vlak van schaalbaarheid. Het werk van ~\textcite{McCorry2017} verbeteren op het vlak van privacy verwacht dit onderzoek niet te doen. 

%---------- Verwachte conclusies ----------------------------------------------
\section{Verwachte conclusies}
\label{sec:verwachte_conclusies}

De verwachte conclusie van dit onderzoek is in de eerste plaats dat de schaalbaarheid van een blockchain-systeem nog steeds de grootste hindernis vormt voor een bredere toepassing op het vlak van stemsystemen. Deze conclusie wordt verwacht omdat de meeste andere onderzoeken betreffende dit onderwerp tot diezelfde conclusie kwamen. Desalniettemin verwacht dit onderzoek door positieve resultaten toch een positieve conclusie te kunnen trekken over de toekomst van blockchain-technologie inzake stemsystemen.

