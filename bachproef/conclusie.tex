%%=============================================================================
%% Conclusie
%%=============================================================================

\chapter{Conclusie}
\label{ch:conclusie}
In deze scriptie werd een onderzoek gevoerd naar de haalbaarheid van blockchain gebaseerd stemmen, daarnaast werd ook de benodigde informatie voor het ontwikkelen van een dergelijk systeem  verzameld. De bijdrage en \textit{proof of concept} van deze scriptie werden gegeven in de vorm van een praktische handleiding waarin een \textit{decentralized application} - ofwel DApp -  genaamd EthereumVote werd ontwikkeld. 

 Dit onderzoek is bedoeld voor iedereen die geïnteresseerd is in zaken als burgerparticipatie, verkiezingen en democratie. Hoewel vaak technisch van aard is het geschreven met niet technisch-onderbouwde personen in gedachten. Alleen de handleiding verondersteld een technische achtergrond. Ze is specifiek gericht op ontwikkelaars met interesse in blockchain stemsystemen. De meerwaarde is dat een compleet stappenplan gegeven wordt, met extra uitleg over gebruikte tools, complexe code, enzovoort. Waar andere handleidingen een louter functionele focus hebben, is er bij EthereumVote rekening gehouden met anonimiteit en schaalbaarheid. Encryptie en registratie werden daartoe als `extra' stappen gebruikt, resulterend in een gulden middenweg tussen absolute veiligheid en kost-efficiëntie.
 
In dit finale het hoofdstuk wordt er een antwoord geformuleerd op de onderzoeksvragen waarrond deze scriptie gebouwd werd:
\begin{itemize}
	\item Wat zijn de voor- en nadelen van de blockchain-technologie in het kader van een stemsysteem?
	\item Is er sprake van een onoverkomelijk schaalbaarheidsprobleem voor blockchain-technologie?
	\item Welke tools heeft men nodig om een blockchain-gebaseerd stemsysteem op te zetten en wat zijn de voor- en nadelen hiervan?
	\item Is een blockchain-gebaseerd stemsysteem haalbaar in de praktijk?
\end{itemize}
\section{Verwachte resultaten en conclusies versus realiteit}
Bij aanvang van dit onderzoek werden resultaten verwacht gelijkaardig aan het onderzoek van \textcite{McCorry2017}. De verwachting was immers dat het werk van dit onderzoek beschikbaar zou zijn als code-bibliotheek. Tijdens de implementatie van EthereumVote bleek dit echter niet het geval, er kon dus ook niet op verder gebouwd worden. 

Veranderingen in de onderliggende technologie leidden dit onderzoek bovendien tot de conclusie dat een meer performante oplossing nodig was. De zeer simpele implementatie van \textcite{McCubin2019} werd daartoe gecombineerd met concepten uit het werk van \textcite{McCorry2017}. EthereumVote is dus niet het verwachte resultaat, het is een overwogen compromis: meer performantie in de ruil voor verlaagde veiligheid op vlak van anonimiteit. 

De algemene conclusies van dit onderzoek komen grotendeels overeen met de initiële verwachtingen.  Blockchain gebaseerde stemsystemen lijken inderdaad over een groot potentieel te beschikken maar worden - ook zoals verwacht - voornamelijk gehinderd  door een schaalbaarheidsprobleem. Dit onderzoek bevestigd hiermee ook de conclusies van zijn voorgangers. Het zijn conclusies die voorzichtig positief kunnen genoemd worden. Het beantwoorden van de onderzoeksvragen leverde deze scriptie bovendien ook enkele nieuwe inzichten op, deze zijn te lezen in de volgende secties.
\section{Voordelen en nadelen}
De vele voordelen van blockchain gebaseerde stemsystemen kwamen aan bod in sectie \ref{sec:blockchain-gebaseerd-stemmen}. Samengevat zijn blockchain gebaseerde stemsystemen  - indien correct geïmplementeerd - beter op het vlak van \textit{privacy}, \textit{transparantie},  \textit{fouttolerantie},  \textit{veiligheid} en \textit{correctheid} dan hedendaagse elektronische tegenhangers. Decentralisatie speelt daarbij een significante rol. Blockchain gebaseerde systemen hebben dus het potentieel om beter te zijn, maar zijn dit niet automatisch.
	
Verder aansluitend op het vorige punt, is het feit dat veel zogenaamde nadelen aan blockchain gebaseerd stemmen heel implementatie gebonden zijn. Sommige implementaties benutten bijvoorbeeld al het volle potentieel van blockchain als technologie, andere doen dit dan weer niet. 

In sectie \ref{sec:blockchain-gebaseerd-stemmen} lag de focus vooral op algemene, recurrente problemen. Een van de hoofdreden die  wordt gegeven is dat er te weinig ondersteuning is op bestaande blockchain platformen voor de ontwikkeling van stemsystemen. Er is voornamelijk een gebrek aan cryptografische ondersteuning, dat was ook de bevinding tijdens de \textit{proof of concept} van deze scriptie.
	
Om die reden lijkt het logisch dat men (voor grootschalige verkiezingsscenario's) een eigen blockchain netwerk opzet. Het grote nadeel hier is echter dat dit bijzonder kostelijk is, zowel op vlak van nodige infrastructuur als het verzamelen van technische kennis.
	
Tenslotte is er ook nog een veelbesproken probleem met blockchain en schaalbaarheid.
\section{Onoverkomelijk schaalbaarheidsprobleem}
 lijkt er sprake te zijn van een schaalbaarheidsprobleem voor blockchain technologie . Het is een probleem dat de ingebruikname van blockchain technologie in verschillende velden verhindert, ook in de context van stemsystemen is dit het geval. Het probleem komt voort uit de structuur van blockchains. 
 
 Zoals besproken in sectie \ref{sec:blockchain-gebaseerd-stemmen} zijn blockchains intrinsiek niet schaalbaar: De snelheid waaraan het toevoegen van blokken aan een keten via het \textit{proof of work} algoritme gebeurt zal altijd het aantal transacties dat per seconden kan worden verwerkt limiteren. 
 
 De complexiteit van \textit{proof of work} verlagen is vaak niet wenselijk gezien dit vaak tot minder decentralisatie leidt en dus de blockchain minder veilig maakt. Zonder twijfel is er sprake dus van een serieus schaalbaarheidsprobleem.
	
De vraag is nu of het schaalbaarheidsprobleem onoverkomelijk is, naar de toekomst toe. 
	
Het is onmogelijk om thans met zekerheid een antwoord op die vraag te bieden. Het is duidelijk dat er pogingen ondernomen worden om het probleem te trotseren: in sectie \ref{sec:ethereum-en-smart-contracts} bespraken we bijvoorbeeld Ethereum's oplossing in de vorm van een nieuw mining algoritme genaamd \textit{proof of stake}. Bitcoin lijkt het over een andere boeg te gooien met een oplossing in de vorm van \textit{off-chain} transacties. De toekomst zal moeten uitwijzen of deze inspanningen het probleem effectief zullen oplossen.
\section{Tools en ontwikkeling}
	
In deze scriptie werd EthereumVote voorgesteld als een zelf-geimplementeerd blockchain gebaseerd stemsysteem. De verschillende tools nodig voor de ontwikkeling van deze DApp kwamen aanbod in sectie \ref{sec:benodigdheden}, de mogelijke problemen die men ermee kan ondervinden werden besproken in hoofdstuk \ref{ch:methodologie}.
	 
Voor Truffle en Ganache is er nauwelijks van nadelen te spreken, beide tools werken uitstekend samen en stellen ontwikkelaars instaat om de werking van hun DApps op een lokale blockchain te simuleren. 

Ook bij de front-end tools Metamask en truffle-contracts zijn er niets dan voordelen. Metamask is een Ethereum wallet die ons instaat stelt om met het netwerk te interageren zonder zelf een node te zijn. Truffle contracts stellt ons instaat om een interface van het smart-contract aan te spreken in de front-end. 

De enige noemenswaardige problemen zijn te vinden bij web3, in de vorm van slecht onderhouden dependencies. In hoofdstuk \ref{ch:methodologie} van deze scriptie werden de daaruit volgende comptabiliteitsproblemen beschreven. Hoewel dit zeker een minpunt is,  blijft web3 een essentiële tool voor de ontwikkeling van DAppps: met enkele lijnen code verbindt het de front-end applicatie met het Ethereum netwerk.
\section{Haalbaarheid}
Kleinschalige blockchain stemsystemen zijn technisch volledig haalbaar, de vele praktijk voorbeelden die werden aangehaald in sectie \ref{sec:blockchain-gebaseerd-stemmen} bevestigen dit. EthereumVote, voorgesteld in hoofdstuk \ref{ch:handleiding} van deze scriptie toont bovendien aan dat de ontwikkeling van een veilig en efficiënt stemsysteem niet complex of kostelijk hoeft te zijn.
	
De inherente schaalbaarheidsproblemen en het gebrek aan absolute veiligheid (op het vlak van anonimiteit) maken blockchain momenteel niet geschikt als onderliggende technologie voor grootschalige verkiezingen zoals we die kennen in de electorale politiek. Zelfs al zouden alle schaalbaarheidsproblemen van de technologie verholpen worden, dan nog zou de adaptatie in politieke verkiezingen erg moeilijk zijn.  Momenteel is er al veel kritiek op elektronisch stemmen, voor nieuwe technologieën zoals blockchain lijkt er alleen nog meer wantrouwen te heersen. 
	
Zolang er duidelijke nadelen verbonden blijven aan blockchain, kan de technologie niet doorbreken in de context van verkiezingen.  Zelfs als men zou kunnen concluderen dat een bepaald blockchain gebaseerd stemsysteem in zijn geheel voordeliger is dan de huidige elektronische en papieren stem (dewelke minder `perfect' zijn dan ze vaak worden uitgemaakt), dan nog blijft de ingebruikname zo goed als onmogelijk.
	
Het lijkt erop dat een praktisch onfeilbaar stemsysteem gepresenteerd moet worden alvorens blockchain zelfs maar in aanmerking kan komen als oplossing vanuit het politieke en juridische standpunt. Zo'n onfeilbaar systeem is volgens dit onderzoek met de huidige blockchain technologie moeilijk, als niet onmogelijk.  Er zou infeite een nieuwe, speciale variant van blockchain technologie moeten ontwikkelt worden. Projecten zoals het Moscow's Citizen's Initiative tonen immers aan dat er met voldoende politieke wil is en de juiste economische middelen al veel meer mogelijk is. 
	
De conclusie is dus dat grootschalige blockchain gebaseerde stemsystemen voorlopig niet haalbaar zijn en dat ook niet plots zullen worden. Potentieel naar de toekomst toe is er wel, alleen is het de vraag of de omstandigheden  (politiek, economisch, technologisch) het benutten er van zullen toelaten. Mocht dit het geval zijn dan ziet deze scriptie de toekomst van blockchain gebaseerde stemsystemen als een positieve zaak.
	
Verder onderzoek zou kunnen uitwijzen hoe groot het potentieel precies is. Concreet zou er een studie kunnen gevoerd worden naar het financiële aspect, waarin de totale kostprijs voor de opzet van een stemsysteem als ook het energie verbruik in operatie wordt berekent. Een andere studie die gevoerd zou kunnen worden is een vergelijkend onderzoek naar de verschillende manieren om anonimiteit   voor gebruikers te creëren. Het zou bijvoorbeeld bijzonder interessant zijn om verkiezingen op basis van hoogstaande cryptografische protocollen zoals OVNP en BroncoVote simultaan naast elkaar te draaien en de resultaten te vergelijken.
	
%% onderzoeksvra(a)g(en). Wat was jouw bijdrage aan het onderzoeksdomein en
%% hoe biedt dit meerwaarde aan het vakgebied/doelgroep? Reflecteer kritisch
%% over het resultaat. Had je deze uitkomst verwacht? Zijn er zaken die nog
%% niet duidelijk zijn? Heeft het onderzoek geleid tot nieuwe vragen die
%% uitnodigen tot verder onderzoek?


