%%=============================================================================
%% Voorwoord
%%=============================================================================

\chapter*{Woord vooraf}
\label{ch:voorwoord}

%% TODO:
%% Het voorwoord is het enige deel van de bachelorproef waar je vanuit je
%% eigen standpunt (``ik-vorm'') mag schrijven. Je kan hier bv. motiveren
%% waarom jij het onderwerp wil bespreken.
%% Vergeet ook niet te bedanken wie je geholpen/gesteund/... heeft

Ter voltooiing van mijn opleiding als Bachelor in de toegepaste informatica presenteer ik u deze scriptie.  Voor ik aan deze bachelorproef begon had ik slechts een flauwe notie van wat blockchain was. Het idee om blockchain-gebaseerde stemsystemen te onderzoeken ontstond dan ook uit pure interesse, na verscheidene interessante artikels te hebben gelezen. Ik werd dus in een wereld geworpen waar ik aanvankelijk weinig van begreep, geleidelijk aan slaagde ik er gelukkig in om kennis rond het onderwerp op te bouwen. Gedurende dit onderzoek leerde ik enorm veel bij, zodat ik mijzelf nu een blockchain-developer kan noemen.

De voltooiing van dit onderzoek vormt voor mij  de afsluiting van de voorbije drie jaar op de Hogeschool Gent, een periode waarin ik mijn technische kennis beetje bij beetje zag groeien leerde om mijzelf steeds te blijven bijscholen, vooral die laatste vaardigheid kwam erg van pas tijdens dit onderzoek en het is daarvoor dat ik alle leerkrachten en medewerkers van mijn opleiding zou willen bedanken.

Mijn dank gaat specifiek ook uit naar mijn promotor, Olivier Rosseel, wiens praktische tips en suggesties bijzonder waardevol waren, net als de literatuur die hij mij  heeft aangereikt.

Verder zou ik ook mijn co-promotor willen bedanken, niet alleen voor het lezen van mijn bachelorproef, maar ook voor het inspireren ervan. Zijn artikel over blockchain-gebaseerd stemmen was een van de redenen die mij aanzette tot het voeren van dit onderzoek.

Als laatste zou ik ook mijn familie en vriendin willen bedanken, niet alleen voor het nalezen en verbeteren van deze scriptie, maar ook voor alle steun die ze mij steeds boden en omdat ik steeds op hen kan rekenen.







