%%=============================================================================
%% Inleiding
%%=============================================================================

\chapter{Inleiding}
\label{ch:inleiding}

%%De inleiding moet de lezer net genoeg informatie verschaffen om het onderwerp te begrijpen en in te zien waarom de onderzoeksvraag de moeite waard is om te onderzoeken. In de inleiding ga je literatuurverwijzingen beperken, zodat de tekst vlot leesbaar blijft. Je kan de inleiding verder onderverdelen in secties als dit de tekst verduidelijkt. Zaken die aan bod kunnen komen in de inleiding~\autocite{Pollefliet2011}:

Democratisch stemmen is een cruciaal beslissingsmechanisme  dat aanwezig is in iedere laag van onze moderne samenleving. Het vormt  de basis van het politieke systeem in veel landen en ook de bedrijfswereld is er van doordrongen. Stemprocessen zijn niet meer weg te denken uit de organisaties van vandaag: of het nu gaat over kleine besluiten op het allerlaagste niveau of over grote strategische beslissingen op het allerhoogste, de verantwoordelijkheid voor het maken van keuzes is bijna altijd beter besteed aan een groep met diverse visies en talenten dan aan één enkele persoon. Het totale belang van alle beslissingen die worden bekomen uit stemmen is niet te onderschatten, het is gigantisch. Dit is het belang van het stemproces.

De volgende vraag dringt zich echter op: hoe houdt men een iets waar zoveel van afhangt betrouwbaar, veilig en eerlijk? In de meeste gevallen blijkt het antwoord op die vraag vrij eenvoudig: waar het aantal participanten klein is, kan het resultaat van de stemming gemakkelijk door iedere deelnemer of observator geverifieerd worden. Iedereen kan getuigen dat alles correct verloopt en dat maakt de kans op frauduleuze praktijken veel kleiner. Wordt het aantal participanten echter groter, dan is een dergelijk systeem onmogelijk. In zo'n geval bepaalt men het resultaat van de stemming via een \textit{centrale autoriteit}. Deze derde partij voert een controlerende functie uit en garandeert de betrouwbaarheid voor alle participanten. Bij nationale verkiezingen is dit bijvoorbeeld het stembureau. 
\newpage
\section{Probleemstelling}
\label{sec:probleemstelling}
Er zijn verschillende problemen bij het gebruik van een centrale autoriteit. Eén daarvan is dat het volledige systeem gebaseerd is op vertrouwen. De kiezer brengt een stem uit en de centrale autoriteit doet de rest. Men heeft weinig\footnote{In sommige landen, waaronder Nederland, kan de kiezer de telling van stemmen bijwonen.} tot geen inzicht in het verdere proces. Er is geen manier om te controleren of de eigen stem werd meegeteld, om zich van de correctheid van het eindresultaat te verzekeren. Eén van de gevaren is dus het risico op machtsmisbruik vanuit de centrale autoriteit. Voorafbepaalde en frauduleuze verkiezingen komen op die manier nog veel te vaak voor in ontwikkelingslanden. Dit gebrek aan transparantie is niet alleen gevaarlijk, het is ook fundamenteel ondemocratisch.

Eigenlijk vertrouwen we bijna blindelings op de goede wil en correctheid van de centrale autoriteit, terwijl daar geen goede reden voor is, in tegendeel zelfs. Ook een goedwillige centrale autoriteit kan fouten maken. Al worden er vaak controles gebruikt, toch zijn zowel menselijke als technische fouten op termijn onvermijdelijk. Een persoon die handmatig stemmen telt kan bijvoorbeeld af en toe fouten maken, zelfs een elektronische stemmachine laat het wel eens afweten. Naast top-down electorale fraude en onopzettelijke fouten zijn er ook nieuwe bedreigingen voor de correctheid en eerlijkheid van verkiezingen. Door de digitalisering van het stemproces worden fenomenen zoals hacking, cyberaanvallen en identiteitsdiefstal ook potentiële gevaren. Externe partijen kunnen proberen in een verkiezing te frauderen, waarbij de elektronische centrale meestal het mikpunt vormt. Toch zien we in dat veel landen in toenemende mate gebruik maken van elektronische systemen in hun verkiezingen.

Een potentiële oplossing biedt zich aan in de vorm van blockchain. Deze datastructuur werd in feite net voor dit soort problemen ontwikkeld. De oorsprong van blockchain is gelinkt aan de eerste gedecentraliseerde digitale munteenheid:  Bitcoin. Net als bij het stemproces, wordt de legitimiteit van de klassieke munteenheden bepaald door hun centrale autoriteit, de centrale banken. De structuur die we vandaag kennen als blockchain zorgt ervoor dat Bitcoin geen centrale bank nodig heeft. Met een hypothetisch blockchain gebaseerd stemsysteem willen we dus het volgende bereiken: een stemsysteem waarin we niet meer hoeven te vertrouwen op een stembureau of andere vorm van centrale autoriteit, maar waarin iedere kiezer de verkiezingsresultaten voor zichzelf kan verifiëren.

Onderzoek naar de mogelijkheden van blockchain stemmen is zeker de moeite waard. Specifiek kan deze scriptie een grote meerwaarde bieden voor ontwikkelaars die interesse hebben in het implementeren van een stemsysteem. De aangeboden handleiding kan hen daarbij op weg helpen. Daarnaast kan ook iedereen die geïnteresseerd is in zaken als burgerparticipatie, verkiezingen en democratie, waaronder co-promotor Dr. Jurgen Goossens, meerwaarde in de scriptie vinden. Deze is - op de handleiding na - gestructureerd en geschreven om perspectief te verschaffen in de complexe aard van blockchain.
\section{Onderzoeksvraag}
\label{sec:onderzoeksvraag}
\begin{itemize}
	\item Wat zijn de voor- en nadelen van blockchain-technologie in het kader van een stemsysteem?
	\item Is er sprake van een onoverkomelijk schaalbaarheidsprobleem voor blockchain-technologie?
	\item Welke tools heeft men nodig om een blockchain-gebaseerd stemsysteem op te zetten en wat zijn de voor- en nadelen hiervan?
	\item Is een blockchain-gebaseerd stemsysteem haalbaar in de praktijk?
\end{itemize}
\section{Onderzoeksdoelstelling}
\label{sec:onderzoeksdoelstelling}
Het beoogde resultaat van deze scriptie is drievoudig:

Ten eerste wenst dit onderzoek een zo helder mogelijk antwoord te bieden op de onderzoeksvragen (\ref{sec:onderzoeksvraag}). We kijken naar de technologie en de implementaties van vandaag om te ontdekken wat de toekomst zal bieden.

Ten tweede wenst dit onderzoek zelf een kleinschalig blockchain gebaseerd stemsysteem te ontwikkelen. Succescriteria zijn dat het systeem \textit{betrouwbaarder}, \textit{efficiënter} en \textit{ten minste even schaalbaar} is als de huidige gecentraliseerde methode. Het ontwikkelen van een grootschalig systeem dat aan deze voorwaarden voldoet valt buiten de scope van deze bachelorproef.

Ten derde wenst dit onderzoek ook een handleiding te bieden waarin de ontwikkeling van het blockchain-gebaseerd stemsysteem stap voor stap wordt toegelicht. Deze handleiding moet software-ontwikkelaars met interesse  in het onderwerp aan de slag kunnen helpen.
\section{Opzet van deze bachelorproef}
\label{sec:opzet-bachelorproef}
% Het is gebruikelijk aan het einde van de inleiding een overzicht te
% geven van de opbouw van de rest van de tekst. Deze sectie bevat al een aanzet
% die je kan aanvullen/aanpassen in functie van je eigen tekst.
De rest van deze bachelorproef is als volgt opgebouwd:

In Hoofdstuk~\ref{ch:stand-van-zaken} wordt een overzicht gegeven van de stand van zaken binnen het onderzoeksdomein, op basis van een literatuurstudie.

In Hoofdstuk~\ref{ch:methodologie} wordt de methodologie toegelicht en worden de gebruikte onderzoekstechnieken besproken om een antwoord te kunnen formuleren op de onderzoeksvragen.

In Hoofdstuk~\ref{ch:handleiding} wordt de handleiding gegeven om een eigen implementatie van een blockchain-gebaseerd stemsysteem te realiseren.

In Hoofdstuk~\ref{ch:conclusie}, tenslotte, wordt de conclusie gegeven en een antwoord geformuleerd op de onderzoeksvragen. Daarbij wordt ook een aanzet gegeven voor toekomstig onderzoek binnen dit domein.