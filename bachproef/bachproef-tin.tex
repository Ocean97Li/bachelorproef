%%=============================================================================
%% LaTeX sjabloon voor bachelorproef, HoGent Bedrijf en Organisatie
%% Opleiding Toegepaste Informatica
%%=============================================================================

\documentclass[fleqn,a4paper,12pt]{book}

\input{structure}

%%---------- Documenteigenschappen --------------------------------------------
%% TODO: Vul dit aan met je eigen info:

% Je eigen naam
\newcommand{\student}{Hu Ocean Li}

% De naam van je promotor (lector van de opleiding)
\newcommand{\promotor}{Olivier Rosseel}

% De naam van je co-promotor. Als je promotor ook je opdrachtgever is en je
% dus ook inhoudelijk begeleidt (en enkel dan!), mag je dit leeg laten.
\newcommand{\copromotor}{Dr. Jurgen Goossens}

% Indien je bachelorproef in opdracht van/in samenwerking met een bedrijf of
% externe organisatie geschreven is, geef je hier de naam. Zoniet laat je dit
% zoals het is.
\newcommand{\instelling}{Hogent  Faculteit Bedrijf en Organisatie}

% De titel van het rapport/bachelorproef
\newcommand{\titel}{Blockchain-gebaseerde stemsystemen, analyse en praktische gids voor het opzetten}

% Datum van indienen (gebruik telkens de deadline, ook al geef je eerder af)
\newcommand{\datum}{27 Agustus 2019}

% Academiejaar
\newcommand{\academiejaar}{2018-2019}

% Examenperiode
%  - 1e semester = 1e examenperiode => 1
%  - 2e semester = 2e examenperiode => 2
%  - tweede zit  = 3e examenperiode => 3
\newcommand{\examenperiode}{3}

%%=============================================================================
%% Inhoud document
%%=============================================================================
\usepackage{amsmath}
\usepackage{nccmath}
\usepackage{graphicx}
\usepackage{listings}
\usepackage{color}
\definecolor{lightgray}{rgb}{.9,.9,.9}
\definecolor{darkgray}{rgb}{.4,.4,.4}
\definecolor{purple}{rgb}{0.65, 0.12, 0.82}
\definecolor{ao(english)}{rgb}{0.0, 0.5, 0.0}
\lstdefinelanguage{JavaScriptSolidity}{
	keywords={pragma,uint,struct,string,number, break, case, catch, continue, get, set, memory, debugger, default, delete, do, else, false, finally, for, function, if, in, instanceof, new, null, enum, return, constructor, contract, switch, this, constructor, throw, view, true, try, typeof, it, assert, mapping, var, void, while, with, public, private},
	morecomment=[l]{//},
	morecomment=[s]{/*}{*/},
	morestring=[b]',
	morestring=[b]",
	ndkeywords={ class, Election, Candidate, Observable, Subject, Map, electionInstance, export, boolean, throw, implements, import, this},
	keywordstyle=\color{blue}\bfseries,
	ndkeywordstyle=\color{ao(english)}\bfseries,
	identifierstyle=\color{black},
	commentstyle=\color{darkgray}\ttfamily,
	stringstyle=\color{orange}\ttfamily,
	sensitive=true
}
\lstdefinelanguage{HTML5}{
	language=html,
	otherkeywords={ >,<, /, [, ],  formControlName, formGroup
	},  
	ndkeywords={
		(ngSubmit),*ngIf, *ngFor
	},  
	ndkeywordstyle=\color{ao(english)}\bfseries,
	morecomment=[s]{<!--}{-->},
	tag=[s]
}

\lstset{
	language=JavaScriptSolidity,
	extendedchars=true,
	basicstyle=\footnotesize\ttfamily,
	showstringspaces=false,
	showspaces=false,
	numbers=left,
	numberstyle=\footnotesize,
	numbersep=9pt,
	tabsize=2,
	breaklines=true,
	showtabs=false,
	captionpos=b
}


\begin{document}

%---------- Taalselectie ------------------------------------------------------
% Als je je bachelorproef in het Engels schrijft, haal dan onderstaande regel
% uit commentaar. Let op: de tekst op de voorkaft blijft in het Nederlands, en
% dat is ook de bedoeling!

%\selectlanguage{english}

%---------- Titelblad ---------------------------------------------------------
\inserttitlepage

%---------- Samenvatting, voorwoord -------------------------------------------
\usechapterimagefalse
%%=============================================================================
%% Voorwoord
%%=============================================================================

\chapter*{Woord vooraf}
\label{ch:voorwoord}

%% TODO:
%% Het voorwoord is het enige deel van de bachelorproef waar je vanuit je
%% eigen standpunt (``ik-vorm'') mag schrijven. Je kan hier bv. motiveren
%% waarom jij het onderwerp wil bespreken.
%% Vergeet ook niet te bedanken wie je geholpen/gesteund/... heeft

Ter voltooiing van mijn opleiding als Bachelor in de toegepaste informatica presenteer ik u deze scriptie.  Voor ik aan deze bachelorproef begon had ik slechts een flauwe notie van wat blockchain was. Het idee om blockchain-gebaseerde stemsystemen te onderzoeken ontstond dan ook uit pure interesse, na verscheidene interessante artikels te hebben gelezen. Ik werd dus in een wereld geworpen waar ik aanvankelijk weinig van begreep, geleidelijk aan slaagde ik er gelukkig in om kennis rond het onderwerp op te bouwen. Gedurende dit onderzoek leerde ik enorm veel bij, zodat ik mijzelf nu een blockchain-developer kan noemen.

De voltooiing van dit onderzoek vormt voor mij  de afsluiting van de voorbije drie jaar op de Hogeschool Gent, een periode waarin ik mijn technische kennis beetje bij beetje zag groeien leerde om mijzelf steeds te blijven bijscholen, vooral die laatste vaardigheid kwam erg van pas tijdens dit onderzoek en het is daarvoor dat ik alle leerkrachten en medewerkers van mijn opleiding zou willen bedanken.

Mijn dank gaat specifiek ook uit naar mijn promotor, Olivier Rosseel, wiens praktische tips en suggesties bijzonder waardevol waren, net als de literatuur die hij mij  heeft aangereikt.

Verder zou ik ook mijn co-promotor willen bedanken, niet alleen voor het lezen van mijn bachelorproef, maar ook voor het inspireren ervan. Zijn artikel over blockchain-gebaseerd stemmen was een van de redenen die mij aanzette tot het voeren van dit onderzoek.

Als laatste zou ik ook mijn familie en vriendin willen bedanken, niet alleen voor het nalezen en verbeteren van deze scriptie, maar ook voor alle steun die ze mij steeds boden en omdat ik steeds op hen kan rekenen.








%%=============================================================================
%% Samenvatting
%%=============================================================================

% TODO: De "abstract" of samenvatting is een kernachtige (~ 1 blz. voor een
% thesis) synthese van het document.
%
% Deze aspecten moeten zeker aan bod komen:
% - Context: waarom is dit werk belangrijk?
% - Nood: waarom moest dit onderzocht worden?
% - Taak: wat heb je precies gedaan?
% - Object: wat staat in dit document geschreven?
% - Resultaat: wat was het resultaat?
% - Conclusie: wat is/zijn de belangrijkste conclusie(s)?
% - Perspectief: blijven er nog vragen open die in de toekomst nog kunnen
%    onderzocht worden? Wat is een mogelijk vervolg voor jouw onderzoek?
%
% LET OP! Een samenvatting is GEEN voorwoord!

%%---------- Nederlandse samenvatting -----------------------------------------
%
% TODO: Als je je bachelorproef in het Engels schrijft, moet je eerst een
% Nederlandse samenvatting invoegen. Haal daarvoor onderstaande code uit
% commentaar.
% Wie zijn bachelorproef in het Nederlands schrijft, kan dit negeren, de inhoud
% wordt niet in het document ingevoegd.

%\IfLanguageName{english}{%
%\selectlanguage{dutch}
%\chapter*{Samenvatting}
%\lipsum[1-4]
%\selectlanguage{english}
%}{}

%%---------- Samenvatting -----------------------------------------------------
% De samenvatting in de hoofdtaal van het document

\chapter*{\IfLanguageName{dutch}{Samenvatting}{Abstract}}

Democratie is een van de bouwstenen van onze moderne samenleving, toch zijn er heel wat problemen met de manier waarop we stemmen. 
Een van de voornaamste problemen is een gebrek aan \textit{transparantie} en \textit{betrouwbaarheid}. In dit onderzoek werd daarom blockchain-gebaseerd stemmen voorgesteld als mogelijke oplossing. 

Ter verdieping in het onderwerp werd een uitgebreide literatuurstudie gevoerd. De meest vooruitstrevende blockchain-gebaseerde stemsystemen van dit moment bleken het \textit{Open Vote Network Protocol} en Moskou's \textit{Active Citizens Project} te zijn. Beide kunnen een betere veiligheid en betrouwbaarheid bieden dan gecentraliseerde tegenhangers. De aard van blockchain-technologie blijkt echter ook voor problemen te zorgen op het vlak van \textit{schaalbaarheid}. Deze kunnen momenteel niet overkomen worden. Het ontwikkelen van private gespecialiseerde blockchain kan wel enig soelaas bieden. Naar de toekomst lijkt het zelfs mogelijk dat een oplossing hieruit voortvloeit.  

In deze scriptie werd EthereumVote voorgesteld. Deze \textit{decentralized application} bestaat uit een \textit{Ethereum-smartcontract} en een verbonden web-applicatie. De implementatie combineert op succesvolle wijze \textit{veiligheid} met \textit{kost-efficiëntie}. EthereumVote vormt het eindresultaat van een praktische handleiding voor het ontwikkelen van een blockchain gebaseerd-stemsysteem. Deze handleiding toont ook aan dat het ontwikkelen zeer vlot gaat met de juiste tools: \textit{Ganache}, \textit{truffle}, \textit{truffle contracts}, \textit{web3} en \textit{Metamask}.

De bevindingen van de literatuurstudie werden bevestigd tijdens het implementeren van EthereumVote. Kleinschalige blockchain-gebaseerde stemsystemen zijn zeer performant,  grootschalige implementaties zijn dat nog niet. Het grootste probleem hier is schaalbaarheid, het overkomen daarvan lijkt een vooral kwestie van politieke wil en financiële middelen, het is iets dat zeker verder onderzocht dient te worden.

%---------- Inhoudstafel ------------------------------------------------------
\pagestyle{empty} % No headers
\tableofcontents % Print the table of contents itself
\cleardoublepage % Forces the first chapter to start on an odd page so it's on the right
\pagestyle{fancy} % Print headers again

%---------- Lijst figuren, afkortingen, ... -----------------------------------

% Indien gewenst kan je hier een lijst van figuren/tabellen opgeven. Geef in
% dat geval je figuren/tabellen altijd een korte beschrijving:
%
%  \caption[korte beschrijving]{uitgebreide beschrijving}

\listoffigures
\listoftables

% Als je een lijst van afkortingen of termen wil toevoegen, dan hoort die
% hier thuis. Gebruik bijvoorbeeld de ``glossaries' package.
% https://www.sharelatex.com/learn/Glossaries

%%---------- Kern -------------------------------------------------------------
\input{inleiding}
\chapter{Stand van zaken}
\label{ch:stand-van-zaken}

% Tip: Begin elk hoofdstuk met een paragraaf inleiding die beschrijft hoe
% dit hoofdstuk past binnen het geheel van de bachelorproef. Geef in het
% bijzonder aan wat de link is met het vorige en volgende hoofdstuk.

% Pas na deze inleidende paragraaf komt de eerste sectiehoofding.

TO DO

\newpage
\section{Blockchain}
	\subsection*{Inleiding}
	In het eerste deel van dit hoofdstuk wordt de blockchain-technologie in haar basisvorm besproken. We beginnen met het introduceren  van  de verschillende concepten en principes waarop blockchain gebouwd is.  We doen dit door het bespreken van hetgeen \textcite{Swan2015} blockchain 1.0 noemt, ofwel de Bitcoin blockchain. In het paper Bitcoin: A Peer-to-Peer Electronic Cash System beschrijft de auteur, onder het pseudoniem Satoshi Nakamoto een monetair systeem dat niet verbonden is aan een bank of  andere financiële institutie. Satoshi Nakamoto is de oorspronkelijke ontwerper van de Bitcoin en richtte de eerste blockchain-database op. Tot op vandaag blijft de ware identiteit van deze persoon (of entiteit) een mysterie. Het paper werd gepubliceerd op 31 oktober 2008 en wordt vandaag gezien als het blockchain white paper,  de structuur die in \textcite{Nakamoto2008} beschreven wordt vormde de basis voor het wat men vandaag blockchain noemt.
	\subsection{Noodzaak}
			\subsubsection{Vertrouwen in de plaat van bewijs}
			\textcite{Nakamoto2008} start met de vaststelling dat het verkopen van waren en diensten op het internet zo goed als volledig afhankelijk is van grote financiële instituties die optreden als derde partij voor het verwerken van elektronische transacties. 
		
			Hoewel ons huidige betaalmodel naar behoren werkt voor de meeste transacties, kent het volgens de auteur van het paper, toch een inherent zwaktepunt: het is namelijk een systeem gebaseerd op vertrouwen en niet op bewijs.
			
			Transacties in het huidige systeem, zo stelt \textcite{Nakamoto2008}, zijn namelijk niet definitief, een pas uitgevoerde transactie is een veel gevallen nog omkeerbaar. Voor de financiële instellingen die fungeren als derde partij kan het ook niet anders: het is onvermijdelijk dat geschillen zullen optreden over bepaalde transacties. Bijgevolg is het ook onvermijdelijk dat er situaties zullen zijn waarin de instellingen via wie de transacties lopen moet ingrijpen door transacties ongedaan te maken. Omdat transacties in het huidige systeem niet als definitief kunnen beschouwd worden, is er een zekere graad van vertrouwen nodig om een transactie aan te gaan tussen de betrokken partijen.  Het aangaan van een transactie vergt daarom, vooral voor de ontvangende partij, een grotere nood aan vertrouwen. 
			
			Men geeft hier het voorbeeld van online-verkopers die zich genoodzaakt zien om hun klanten om meer persoonlijke informatie te vragen dan ze eigenlijk nodig hebben. En ook al worden er meer gegevens gevraagd dan nodig, dan nog is een zeker fraude percentage onvermijdbaar. 
		
			Wanneer men de analogie maakt voor gebruik van cash geld dan ziet men dat de bovenstaande problemen in veel mindere mate voorkomen. Bij het gebruik van cash geld gebeurt de transactie immers niet enkel op basis van vertrouwen maar vooral op basis van wederzijdse controle. Als een klant een bepaalt bedrag betaald of wisselgeld ontvangt dan telt deze het geld ter controle. Idem dito de ontvangende kant. Bij cash geld is het minder evident om een transactie ongedaan te maken. 
			
			Het paper komt tot de conclusie dat een alternatief elektronisch betalingssysteem nodig is. In zo’n systeem zouden transacties niet mogen gebeuren op basis van vertrouwen maar eerder basis van een wederzijdse controle, net zoals bij het voorbeeld van cashgeld- het geval was. Een controlemechanisme gebaseerd op crypto-grafisch bewijs zou twee partijen in staat kunnen stellen om online transacties rechtstreeks met elkaar aan te gaan, zonder daarbij gebruik te moeten maken van een vertrouwde partij. De noodzaak tot een 3e partij, in de vorm van een financiële instantie zoals een bank, is in een dergelijk systeem volledig geëlimineerd.
		
			In het systeem dat \textcite{Nakamoto2008} voorstelt zijn de transacties beschermt door cryptografie die het computationeel onpraktisch maakt om ze ongedaan te maken. De transacties zijn volledig onomkeerbaar en beschermen hiermee verkopers tegen fraude. Daarbovenop zijn borg-mechanismen, nodig om ook kopers beschermen, volgens de auteur ook makkelijk te implementeren in het systeem.
			
			\subsubsection{Het double-spending probleem}
			Het paper wijdt veel aandacht aan het oplossen van het dubble-spending probleem. Dit probleem beschrijft het bestaande risico dat bij een digitale vorm van geld of een ander digitaal-middel dezelfde middelen meerdere malen gespendeerd kunnen worden.
			
			Een van de fundamentele verschillen tussen contant en elektronisch geld is dat het eerste van een fysieke aard is. Dat betekend dat het geld tastbaar is, men draagt het bij zich, en men geeft het door wanneer men transacties maakt. Daarna is het geld weg. Eenmaal gespendeerd is contant geld op, hetzelfde geld kan elders niet opnieuw aangewend worden. Er is geen bank of andere financiële instantie voor nodig om dit verifiëren. Hoeveel men spendeert en hoeveel vermogen er nog rest is evident. 
			
			Bij elektronische vormen van geld is dit alles vele malen complexer. Ter verduidelijking nemen we hier het voorbeeld van een klassieke rekening bij een bank.
			
			\begin{figure}
				\includegraphics[width=\linewidth]{img/double_spending2.png}
				\caption{Double spending bemoeilijkt door een centrale autoriteit}
				\label{fig:double_spending2}
			\end{figure}
			
			Het geld op de hedendaagse bankrekening bestaat enkel en alleen als een elektronisch getal, een reeks van digitale cijfers bestaande uit 1’en en 0’en. Zo’n getal op zichzelf is gemakkelijk gewijzigd, men hoeft slechts wat 1’en en 0’en toe te voegen om een veelvoud van het oorspronkelijke te bekomen. Het is de bank die instaat voor de beveiliging van elektronisch geld. Waar de bank van weleer de bewaker was van vermogens in de vorm van kluizen of goudreserves, is de bank van vandaag de digitale bewaker van binaire vermogens.
			
			Bij een typische elektronische transactie tussen twee partijen vindt er geen transfer plaats van fysieke objecten, hetgeen er wel plaats vindt is aan de ene kat een verlaging van het elektronische vermogen van de eerste partij en aan de andere kant een verhoging van het vermogen van de tweede partij. 
		
			Gezien de enorme complexiteit die de beveiliging digitale systemen met zich meebrengt is het uitvoeren van elektronische transacties binnen het klassieke monetaire systeem altijd de verantwoordelijkheid van een vertrouwde derde partij geweest. Wanneer men een elektronische aankoop doet is het deze derde partij die een centrale autoriteit vormt wat betreft de veiligheid en de geldigheid van de transactie. Het is de vertrouwde derde partij die controleert op dubble-spending, vervolgens een bepaald bedrag in mindering breng van het oorspronkelijke vermogen en dit bedrag tenslotte toevoegt aan de kant van de verkoper. Dit wordt geïllustreerd in figuur \ref{fig:double_spending2}
			
			In de meeste gevallen neemt vertrouwde derde partij de vorm aan van een bank, maar alternatieve instanties die kunnen optreden als vertrouwde derde partij voor transacties. Voorbeelden hiervan zijn: PayPal, TransferWise, Google Pay en Apple Pay.
		
			Haalt men deze derde partij echter volledig uit het proces dan dringt de nood voor een compleet nieuwe oplossing voor het dubble-spending probleem zich onmiddellijk op.  Dit wordt geïllustreerd in figuur \ref{fig:double_spending1}
			
			\begin{figure}
				\includegraphics[width=\linewidth]{img/double_spending1.png}
				\caption{Double spending zonder een centrale autoriteit}
				\label{fig:double_spending1}
			\end{figure}
		
			Double-spending vormt een dusdanig groot probleem dat het de ontwikkeling van elektronisch geld zonder een vertrouwde autoriteit of een centrale server lange tijd onmogelijk werd geacht. In de volgende secties wordt de oplossing van \textcite{Nakamoto2008} voor het probleem besproken. 
			
			\subsubsection{Het Byzantijnse Generaalsprobleem}
			Het double-spending probleem is een probleem dat specifiek is voor digitale valuta. Het Byzantijnse Generaalsprobleem, is een gelijkaardig probleem, alleen is het van toepassing op een iets bredere context. Het byzantijnse vraagstuk omschrijft eigenlijk het achterliggende probleem bij double-spending: de vraag hoe men vertrouwen achterwege laat in een systeem zonder centrale autoriteit.
			
			Het probleem luidt als volgt: generaals van Byzantium moeten hun troepen coördineren voor een aanval, en dit op basis van berichten die ze elkaar versturen. De aanval moet met meer dan de helft van het totaal aantal troepen worden uitgevoerd op een exact moment, anders zal ze falen. Het is echter mogelijk dat een of meerdere generaals verraders zijn, die de aanval willen dwarsbomen. De identiteit van deze generaals kan niet achterhaald worden. De probleemstelling is dus hoe de generaals die te goeder trouw zijn toch hun troepen kunnen coördineren tot een aanval, vrij en open met elkaar communicerend, zonder dat een verrader hun plannen kan dwarsbomen door valse berichten te versturen. Het Byzantijnse Generaals probleem wordt geillustreerd door figuur \ref{fig:byzantium}
			
			Het Byzantijnse generaalsprobleem is tot op vandaag erg relevant omdat het van toepassing is op eender welke context waarin communicatie tussen verschillende entiteiten een cruciale rol speelt en er geen absoluut vertrouwen is. Het gebied van gedistribueerde computernetwerken is hier een perfect voorbeeld van. De oplossing van \textcite{Nakamoto2008} claimt niet alleen het double-spending probleem op te lossen, maar ook het byzantijnse generaals probleem. Men spreekt ook wel van een model met Byzantijnse fouttolerantie.
			
			\begin{figure}
				\includegraphics[width=\linewidth]{img/byzantine_generals.png}
				\caption{Het Byzantijnse generaals probleem.}
				\label{fig:byzantium}
			\end{figure}
			
	\subsection{Transacties}
	Het monetaire systeem dat \textcite{Nakamoto2008} voorstelt werkt op basis van digitale munten. Deze munten worden gedefinieerd als ketens opgebouwd uit digitale ondertekeningen. Als de eigenaar van zo’n een munt een transactie aangaat wordt de munt doorgegeven aan de volgende eigenaar door ze digitaal te ondertekenen met een hash. Deze hash bestaat uit de hash van de vorige transactie gecombineerd met de publieke sleutel die de volgende eigenaar identificeert. De nieuwe hash wordt aan de keten van ondertekeningen toegevoegd waaruit de munt bestaat. De ontvanger kan dan de ondertekeningen, en daarmee historiek van eigendom van de munt verifiëren. 
			
	De historiek van een digitale munt kennen lost het dubble-spending probleem echter niet op. Men kan immers niet controleren of dezelfde munt niet ergens anders werd aangewend. \textcite{Nakamoto2008} stelt dat er een manier nodig is om voor iedere munt te verifiëren dat de vorige eigenaar geen eerdere transacties met de munt heeft aangegaan. Daartoe beslist men om de chronologisch eerst-voorkomende transactie van een eigenaar met een munt als valabel te beschouwen en alle daaropvolgende transacties van die eigenaar als double-spending. 
			
	Om de chronologische orde van een transactie te kunnen bepalen, is er kennis nodig van alle transacties. De enige manier om dit zonder vertrouwde partij te doen is door alle transacties publiek aan te kondigen. Vervolgens is er een mechanisme nodig waarbij alle participanten van het systeem (alle computers binnen het netwerk), hierna nodes genoemd, gezamenlijk kunnen beslissen over de exacte volgorde waarin transacties gebeurden. \textcite{Nakamoto2008} stelt voor om het probleem op te lossen startend vanuit een timestamp server. 
			
	\subsection{Timestamp Server}
	Een digital timestamp is een digitaal certificaat dat verzekerd dat een digitaal document op een bepaald ogenblik bestond. Er zijn meerdere, zeer specifieke technieken om betrouwbare digitale timestamps te kunnen produceren. In het algemeen zijn deze technieken onder te verdelen in twee grote categorieën: degene die gebaseerd zijn op een vertrouwde derde partij en degene die gebaseerd zijn op gedistribueerd vertrouwen. (BRON: https://nakamotoinstitute.org/static/docs/secure-timestamping-service.pdf )
	
	In dit geval wordt door de auteur er door \textcite{Nakamoto2008} gekozen voor de gedistribueerde techniek. In de volgende subsectie wordt een methode besproken om niet langer op basis van vertrouwen te werken.
	
	\textcite{Nakamoto2008} stelt voor om een timestamp server te gebruiken om de chronologische ordening van transacties die gebeuren met zijn digitale munt te bepalen.  Concreet werkt deze timestamp server door meerdere transacties die een timestamp moeten krijgen samen te nemen in een blok, er een hash van te berekenen en deze vervolgens openbaar te maken. (insert foto uit white paper)
	
	De timestamp bewijst het bestaan van de items op dat specifieke moment, gezien deze verwerkt zijn in de hash, en de hash uniek is en alleen maar kon gegenereerd worden uit specifieke combinatie van items. De timestamp bevat verder ook de voorafgaande timestamp in de hash, waardoor een ketting ontstaat. Iedere timestamp versterkt daarbij de voorafgaande, een chronologische historiek van transacties ontstaat.
	
	De reden dat men hier van een gedistribueerde techniek spreek is omdat de keten van transactie blokken niet op een centraal punt bestaat. Er bestaat een exemplaar van de database-structuur, met alle informatie erin op iedere node van een peer-to-peer netwerk. 
	
	Een peer-to-peer netwerk is een netwerk bestaande uit onderling verbonden computers waarbinnen iedere computer gelijkwaardig is en er dus geen sprake is van centrale autoriteit. 
	
	Het netwerk in zijn geheel wordt gebruikt als timestamp server om bewijs te genereren van de chronologische volgorde van transacties. Iedere node van het netwerk kent de volledige historie van transacties en kan de validiteit van gemaakte transacties bewijzen. Een partij die een frauduleuze transactie probeert te plegen valt snel door de mand gezien ze slechts 1 node in het netwerk representeert. Alle andere nodes leveren immers een bewijs dat afwijkt van dat van de frauduleuze node. Het netwerk accepteert periodiek een aantal transacties tegelijkertijd, deze vormen een zogenaamd blok. Blokken waarover er een consensus van 50\% of meer is worden geaccepteerd en in alle nodes aan een interne keten van blokken toegevoegd.
	
	Een database systeem, waarbij informatie periodiek in blokken geaccepteerd wordt en in iedere node van een P2P netwerk wordt toegevoegd aan een keten van blokken, noemt men vandaag de dag een blockchain.
	
	Een blockchain kan als veilig beschouw worden zolang minstens 50\% van de rekenkracht van het netwerk niet gecomprimeerd is, en het merendeel van de bewijzen dus steeds de waarheid representeert. In praktijk betekent dit een systeem dat nagenoeg oncomprimeerbaar is. In het geval van het hedendaagse Bitcoin netwerk spreekt men bijvoorbeeld van een netwerk met rond de 10.000 nodes (LaTex Bron Vermelden). Om succesvol te frauderen zouden aanvallers van het Bitcoin netwerk maar liefst de helft van al deze computers, verspreid over heel de wereld moeten controleren. 
			
	\subsection{Proof-of-Work}
	\label{subsec:pow}
	Om een gedistribueerde timestamp server op peer-to-peer basis te laten werken is er een extra veiligheidsmechanisme nodig dat het berekenen van hashes opzettelijk moeilijker maakt zodat het langer duurt om een blok toe te voegen. 
	
	Het concept, dat door \textcite{Nakamoto2008} proof-of-work genoemd wordt, is essentieel omdat moderne CPU’s over enorme rekenkracht beschikken. Mits er geen extra veiligheidsmechanisme zou zijn zouden hashes zeer snel berekent kunnen worden. Zodanig snel dat het mogelijk zou zijn voor een enkele aanvaller om een aanpassing te maken in de ketting en voor alle volgende blokken de hashes te opnieuw te berekenen. 
	
	Naargelang de ketting groeit en er meer hashes gebaseerd zijn op een voorgaande hash wordt het door proof-of-work steeds moeilijker om gegevens te wijzigen. Immers: een wijziging van gegevens betekent dat de hash ook herrekend moet worden. Dit veroorzaakt corruptie van de hashes van alle daaropvolgende blokken, de enige manier om dit op te lossen is de hashes van iedere blok te herrekenen. 
	Gezien de gemiddelde proof-of-work echter 10 minuten duurt en er ook om de 10 minuten een nieuwe blok aan de keten wordt toegevoegd is het bijna onmogelijk dat de corruptie niet door andere nodes wordt opgemerkt.
	
	Concreet bestaat de proof-of-work eruit dat er moet gezocht worden naar een bepaalde waarde. De waarde moet aan een voorwaarde voldoen: als men de hash functie toepast op de waarde dan moeten de eerste x aantal bits van het resultaat een 0 zijn. 
	
	De aard van een hashfunctie laat niet toe om aan reverse-engineering te doen, of anders gezegd men kan niet van een resultaat beginnen dat aan de voorwaarden voldoet om zo de gevraagde waarde te vinden. Proof-of-work kan dus alleen geleverd worden door miljarden waarden te overlopen, er de hash-functie op toe te passen en te controleren of er aan de voorwaarde voldaan is. Het is met andere woorden zeer tijdsconsumerend rekenwerk. De gemiddelde tijd nodig om een hash te vinden is exponentieel in x (het gevraagde aantal start-bits van de hash die 0 zijn).
	
	Zoals vermeld wordt het accepteren van nieuwe blokken gedaan op basis van een meerderheids-stem. Een vraag die hier opkomt is wat er precies als stem moet tellen en wat niet. Als de meerderheid bijvoorbeeld bepaald zou worden aan de hand van een zogenaamd one-ip-address-one-vote model dan zou het systeem misbruikt kunnen worden door nodes meerdere IP-adressen te laten alloceren. Proof-of-work biedt hier de oplossing. De meerderheidsbeslissing wordt gerepresenteerd door de langste keten, dewelke waarin de langste proof-of-work tijd geïnvesteerd is. Zolang de meerderheid van de CPU-kracht in het netwerk gecontroleerd wordt door oprechte nodes, zal deze ketting sneller groeien dan enige andere ketting. In essentie is het een one-CPU-one-vote model ~\autocite{Nakamoto2008}.
	
	Om te compenseren voor toenemende hardware-capabiliteit wordt de moeilijkheid van de proof-of-work dynamisch bepaald door een bewegend gemiddelde dat er op gericht is om het aantal blokken dat per uur toegevoegd wordt constant te houden. Als hashes te snel worden gegenereerd, wordt de moeilijkheid simpelweg verhoogd ter compensatie. 
	\subsection{Netwerk}
		\subsubsection{Verwerking  van transacties}
		In deze paragraaf wordt een abstracte schets gegeven van de conceptuele werking van \textcite{Nakamoto2008}’s netwerk:
		\begin{enumerate}
			\item Transacties worden gebroadcast naar ieder node in het netwerk
			\item Binnenkomende transacties worden in een blok gegroepeerd door iedere node.
			\item Iedere node werkt om de proof-of-work op te lossen
			\item De node die de proof-of-work als eerste vindt broadcast de blok van transacties zoals gekend door die node naar alle ander nodes.
			\item Al de andere nodes accepteren dit blok, op voorwaarde dat de bevatte transacties geldig zijn en niet double-spent.
			\item Nodes bevestigen hun acceptatie door aan de creatie van het volgende blok te beginnen, gebruik makende van de hash van het vorige blok.
		\end{enumerate}
		\subsubsection{Langste ketting}
		Het is steeds de langste ketting die als de correcte wordt beschouwd en verder wordt uitgebreid. Als twee nodes gelijktijdig een verschillend blok broadcasten dan zullen sommige nodes de ene versie eerst ontvangen en andere nodes de andere. Er ontstaat dan een situatie waarin er twee verschillende versies van de keten zijn binnen het netwerk. Deze discrepantie wordt pas weggewerkt wanneer de volgende proof-of-work wordt geleverd en er een weer een langste ketting is. Nodes die op de andere alternatieve keten werkten, schakelen weer over naar de ware keten.
		\subsubsection{Fout-tolerantie}
		Het systeem kent een hoge fout-tolerantie, het kan overweg met vrij veel fouten die zich in een realistisch scenario  kunnen voordoen binnen een netwerk. Zo kan het best zijn dat de broadcast van een nieuwe transactie niet iedere node in het netwerk bereikt. Dit is geen probleem zolang het merendeel van de nodes de transactie wel ontving. Hetzelfde geld voor de broadcasts van blokken, als een node door bepaalde omstandigheden een broadcast van een blok niet ontvangt, zal de node bij de volgende broadcast realiseren dat het dat blok mist en het netwerk verzoeken om dit door te sturen.
	\subsection{Mining}
		\subsubsection{Nieuwe bitcoins}
		Gezien er voor de bitcoin geen centrale autoriteit is die het geld maakt of verdeeld is er een alternatieve wijze nodig om de digitale munten te creëren en in circulatie te brengen. Een analogie voor de creatie van nieuwe bitcoins, zo stelt \textcite{Nakamoto2008}, is het mijnen van een kostbare grondstof zoals goud. Bij de goud-mijnbouw moet er grote hoeveelheden middelen besteed worden om nieuw goud te ontginnen en in circulatie te brengen. Bij bitcoin is de situatie vergelijkbaar. De creatie van iedere bitcoin, hoewel enkel digitaal, vergt een zekere prijs in tijd en in middelen, met name elektriciteit.
		
		Conventie in het bitcoin netwerk is dat de eerste transactie van ieder blok een speciale transactie is, die een munt toekent aan de eigenaar van het blok. Het idee hier is om een stimulans te creëren die nodes ertoe aanzet om aan proof-of-work te doen en zo het netwerk te ondersteunen. De periodieke creatie van iedere blok is dus eigenlijk een race tussen duizenden nodes van het netwerk om als eerste de proof-of-work te kunnen leveren en in ruil daarvoor een beloning in bitcoin te ontvangen. 
		
		De analogie die \textcite{Nakamoto2008} maakt tussen bitcoin en mijnbouw, leidde tot de hedendaagse benaming voor dit concept: mining. 
		
		\subsubsection{Transacties}
		De stimulans om het netwerk te ondersteunen zal deels gecreëerd worden aan de hand van transactie kosten. \textcite{Nakamoto2008} stelt dat als de output waarde van een transactie minder is dan de input waarde, het verschil een transactie kost is die toegevoegd wordt op de stimulanswaarde van het blok dat de transactie bevat. Hiermee wordt bedoeld dat de nodes van het netwerk niet alleen bitcoin verdienen door het creëren van nieuwe bitcoins maar ook door het verwerken van transacties. Dit is van fundamenteel belang gezien het totaal ‘ontginbare’ bitcoins eindig is. Eenmaal een aantal bitcoins in omloop is zullen er geen nieuwe bitcoins meer gecreëerd kunnen worden en zal de ondersteuning van het netwerk overschakelen naar volledige financiering door middel van transactie kosten, deze zullen volgens \textcite{Nakamoto2008} op hun beurt stabiliseren en inflatievrij\footnote{De Bitcoin is sinds haar creatie steeds onderhevig geweest grote fluctuaties, zie https://bitcoinfees.info/} worden.
		
		De stimulans kan ook helpen om nodes eerlijk te houden. Zo zal de potentiële aanvaller van het netwerk ondervinden dat de CPU kracht, nodig om fraude te plegen, veel lucratiever blijkt wanneer aangewend voor eerlijke mining-doeleinden. 
		
		
		
	\subsection{Disk Space}
		\subsubsection{Hash-boom}
		\textcite{Nakamoto2008} stelt  een manier voor waarop data van oude transacties kan verwijderd worden om opslag ruimte te besparen. Om een dergelijke actie mogelijk te maken, zonder dat de hash corrumpeert, worden transacties binnen ieder blok in een structuur opgeslagen die men een hash-boom noemt. Zo’n structuur stelt instaat om de onderliggende transactie data te verwijderen maar de hashes te behouden.
		
		\subsubsection{Wet van Moore}	
		\textcite{Nakamoto2008} stelt dat gezien de Wet van Moore voorspelt dat hardware capaciteiten per jaar veel sneller zullen blijven toenemen dan de opslagruimte nodig om de groeiende keten van blokken in te bewaren, miners zich eigenlijk geen zorgen zouden moeten maken over opslagruimte. De Wet van Moore wordt geïllustreerd door figuur \ref{fig:moore}.
		\begin{figure}
			\includegraphics[width=\linewidth]{img/moore.png}
			\caption{De Wet van Moore.}
			\label{fig:moore}
		\end{figure}
		
		\subsection{Simpelere Verificatie van Betaling}
		Volgens \textcite{Nakamoto2008} is het mogelijk om betalingen te verifiëren zonder de hulp van een volledige netwerk-node, die iedere blok ooit gecreëerd kent. Om een transactie te verifiëren moet een gebruiker enkel een kopie hebben van de headers van alle blokken uit de langste keten. Deze kunnen alleen verkregen worden door de netwerk-nodes te bevragen en op een gegeven te beslissen dat men de langste keten gevonden heeft, en vervolgens van deze keten de tak van de hash-boom die de transactie aan het blok linkt te nemen.
	
		Met deze hash-tak kan de gebruiker wel is waar niet de transactie zelf controleren, maar er kan aan de hand van de positie van de hash van de transactie wel gecontroleerd worden of de transactie al geaccepteerd is door de node en het netwerk.
		
		Op deze manier is verificatie gemakkelijk, zolang eerlijke nodes controle over het netwerk hebben. Nodes kunnen transacties altijd zelf verifiëren, maar de vereenvoudigde methode kan beetgenomen worden door aanvallers die meer dan de helft van het netwerk controleren. Een mogelijke strategie hiertegen zou zijn om de volledige keten te downloaden wanneer er een ongeldige blok binnenkomt. 
	\subsection{Combineren en splitsen van transacties}
	Opdat er geen aparte transactie voor iedere munt die van eigenaar wisselt zou moeten gemaakt worden, voorziet \textcite{Nakamoto2008} een systeem waarin transacties gecombineerd en gesplitst kunnen worden. Zo kan men bijvoorbeeld drie Bitcoins in een enkele transactie versturen, maar ook 0.5 Bitcoin als wisselgeld ontvangen.
	
	Om dit moglijk te maken wordt er gewerkt met een systeem van inputs en outputs. Een transactie kan een of meerdere inputs hebben en een of twee outputs. De inputs stellen de bron(nen) waar het geld vandaan komt voor, eenderwelke waarde in bitcoin kan hier meegegeven worden. De output stellen de ontvangende kant voor, de eerste output is de waarde voor de ontvanger, de tweede output is optioneel en is voor het eventuele wisselgeld dat de verzender kan ontvangen.
	\subsection{Privacy}
	In het traditionele monetaire systeem ligt alle kennis over transacties bij de financiële instituties. De bank, als vertrouwde derde partij kan gemakkelijk privacy creëren door de toegang tot informatie over een transactie te limiteren tot de betrokken partijen. 
	
	In het model dat we vanaf nu de blockchain zullen noemen, is deze methode onmogelijk door de noodzaak tot het publiek aankondigen van transacties over het hele netwerk.  \textcite{Nakamoto2008} stelt echter dat privacy wel enigszins kan behouden worden door informatie op een ander plaats te beperken. Men stelt voor om de publieke sleutels, die bitcoin-portefeuilles identificeren anoniem te houden, en er dus geen naam aan te koppelen. Op deze manier mag iedereen dan wel kunnen zien welke transacties er plaats vinden, maar door de anonimiteit van zender en ontvanger wordt privacy grotendeels gegarandeerd.
	
	Om de privacy nog te verbeteren zou men ook kunnen opteren voor een systeem waarin iedere gebruiker per transactie een nieuwe sleutel krijgt die uniek identificerend is. Voor transacties met meerdere inputs is er echter geen manier om te verbergen dat de inputs van een eigenaar afkomstig zijn.
	\subsection{Nakamoto’s Conclusie}
	In het paper presenteert men een nieuw systeem voor elektronische transacties. Men start vanuit munten die opgemaakt zijn uit digitale ondertekeningen. Vervolgens lost men het double-spending probleem op. Om dit te doen wordt er een peer-to-peer netwerk voorgesteld dat gebruik maakt van zogenaamde proof-of-work om de historiek van gemaakte transacties op te slaan. 
	
	De implementatie van dit alles is een aard die het computationeel onpraktisch maakt om het systeem aan te vallen voor frauduleuze doeleinden. Het beslaat immers een robuust netwerk met weinig tot geen complexe structuur, maar waar door anonimiteit de privacy grotendeels gerespecteerd blijft. 
	
	Nodes binnen het netwerk werken allemaal tegelijk, doch zonder enige coördinatie of afhankelijkheid. Ze kunnen het netwerk verlaten en zich naar believen weer vervoegen. Nodes die zijn weggeweest moet enkel de volgende proof-of-wok accepteren om weer helemaal mee te zijn met alles wat er gebeurd is. 
	
	Er wordt gestemd op basis CPU-kracht: accepteren van blok gebeurt door aan de proof-of-work te beginnen werken, afwijzen van een blok gebeurt door dit te weigeren. Op basis van dit consensus mechanisme kunnen extra regels afhankelijk voor een der welke usecase worden toegevoegd ~\autocite{Nakamoto2008}. 
	\newpage
\section{Ethereum en smart contracts}
\label{sec:ethereum-en-smart-contracts}
	\subsection*{Inleiding}
		In het tweede deel van dit hoofdstuk wordt een beeld geschetst van wat \textcite{Swan2015} blockchain 2.0 noemt, ofwel 'Blockchain voorbij Bitcoin'. De concepten die in het vorige hoofdstuk werden besproken vormen de basis van de blockchain technologie. De Bitcoin blockchain staat intussen echter al weer een stuk verder en de Bitcoin is ook bijlange na niet meer de enige cryptomunt. De Bitcoin blockchain is dus bijlange niet meer de enige blockchain. Concreet wordt er in dit hoofdstuk dieper ingegaan op het concept  smart contracts, vervolgens wordt er ook een overzicht gegeven van Ethereum. Dit alles wordt uitgebreid besproken omdat het van belang zal worden eenmaal we blokchain stemsystemen (zie \ref{sec:blockchain-gebaseerd-stemmen}) bespreken.
	\subsection{Noodzaak}
		\subsubsection{Wat  zijn smart contracts?}
			Smart contracts zijn een concept dat de blockchain-technologie een stap verder neemt. In \textcite{Swan2015} worden ze omschreven als gedecentraliseerde contracten die niet langer een autoriteit (zoals een rechtbank) nodig hebben. Het zijn in feite digitale contractprogramma’s die zichzelf kunnen valideren en uitvoeren wanneer aan bepaalden voorwaarden is voldaan. Ook hier bestaat het concept sinds de jaren negentig ~\autocite{Szabo1996}. Bij het lezen van \textcite{Nakamoto2008} is het duidelijk dat er vanaf het prille begin van de bitcoin een visie was om een dergelijke systeem te implementeren. Blockchain-technologie staat immers niet alleen toe om data op te slaan, ook programma’s kunnen in de blockchain worden opgeslagen. Smart contracts vormen de basis van de nieuwe Blockchain 2.0 van vandaag, zowat iedere grote blockchain-speler probeert ze te implementeren~\autocite{Swan2015}.
		\subsubsection{Wat is Ethereum?}
			Ethereum (figuur \ref{fig:ethereum}) is een opensource-platform dat werd opgericht in 2015. Net zoals bitcoin maakt het gebruik van een gedecentraliseerd netwerk, gebaseerd op het oorspronkelijke blockchain- concept. Het valideren van informatie gebeurt ook hier door zogenaamde miners, het verschil met bitcoin is dat de miners worden beloond met de munteenheid ether in plaats van bitcoin. Ethereum kan men niet zien als een zuivere variant op de bitcoin of een andere vorm van cryptogeld, het is veel meer dan dat. Om te beginnen maken smart contracts  een groot deel uit van het Ethereum ontwerp . \textcite{Swan2015} omschrijft Ethereum als een ”Turing-Complete Virtual Machine”, die zowel een platform als een programmeertaal biedt voor het ontwikkelen en publiceren van gedistribueerde applicaties. Turingcompleetheid betekent in deze context dat het over een platform gaat dat het vermogen heeft om eender welke digitale munt, protocol of blockchain te ondersteunen, iets wat bij de Bitcoin blockchain niet het geval is \autocite{Swan2015}. Ethereum is momenteel (1 april 2019) de tweede grootste cryptomunt na de Bitcoin\footnote{zie https://coinmarketcap.com/all/views/all/}. 
			
			\begin{figure}
				\includegraphics[width=\linewidth]{img/ethereum.png}
				\caption{Het Ethereum logo}
				\label{fig:ethereum}
			\end{figure}
			
			Ethereum is ontworpen met de volgende filosofie in gedachten (Ethereum Wiki)\footnote{zie https://github.com/ethereum/wiki/wiki/White-Paper\#ethereum}:
			\begin{itemize}
				\setlength\itemsep{1em}
				\item \textbf{Simpliciteit}: 
				Een van de hoofddoelen van Ethereum is om zo simpel mogelijk te zijn, zelfs als dit soms ten koste komt van data-opslag of tijdefficiëntie. Ethereum werd ontworpen met de bedoeling dat een gemiddelde programmeur, zonder een diepgaande kennis van cryptografie, er applicaties op zou kunnen implementeren. De bedoeling is dat de lage instapdrempel die de simpliciteit van Ethereum creëert er toe bijdraagt dat het ongekende potentieel van cryptocurrencies en blockchain technologie verder uitgebouwd wordt.
				\item \textbf{Universaliteit}: Ethereum doelt er op om Turingcompleet te zijn. Men wil  geen systeem aan bieden waar er gebruik kan worden gemaakt van bepaalde features, men wil een platform aan bieden waarop ontwikkelaars zelf iedere mogelijke toepassing kunnen implementeren aan de hand van smart-contracts en transacties.
				\item \textbf{Modulariteit}: 
				Een ander belangrijk aspect in het ontwerp van Etheruem is modulariteit. De bedoeling is dat de verschillende onderdelen waaruit Ethereum is opgebouwdt, zaken zoals Ethash, Patricia bomen en RLP, zo scheidbaar mogelijk worden gehouden. De verschillende bouwstenen van Ethereum worden als feature-complete bibliotheken gezien en kunnen ook buiten Ethereum gebruikt worden.
				\item \textbf{Agiliteit}: 
				Ethereum moet op een agile manier ontwikkelt worden. Men moet heel flexibel kunnen zijn op het vlak van aanpassingen. Hoewel men heel voorzichtig is wanneer het aankomt op modificaties bij high-level constructies, heeft Ethereum ook het doel om nieuw ontdekte mogelijkheden die verbetering brengen aan het systeem zo snel mogelijk te benutten.
				\item \textbf{Non-discriminatie en Non-censuur}: 
				Tot slot zou Ethereum niet mogen aansturen op een bepaalde vorm van gebruik. De regulerende mechanismen in het protocol moeten op zodanig wijze ontwikkelt zijn dat ze alleen schade zelf tegenhouden en niet specifieke ongewenste applicaties. Het voorbeeld van een oneindige lus wordt hier gegeven. Een applicatie die zo'n lus bevat is  ongewenst omdat ze middelen van het netwerk in beslag zal nemen en zo de verwerking van informatie zal vertragen. Toch wordt een degelijke applicatie niet verboden door Ethereum. Het regulerende mechanisme dat een transactiekost per computationele stap garandeert zorgt er immers voor dat een oneindige lus uitvoeren bijzonder nadelig wordt.
			\end{itemize}
	\subsection{Werking van Ethereum}
		De werking van het Ethereum-netwerk volgt op een hoog conceptueel niveau dezelfde lijnen als de eerder omschreven Bitcoin Blockchain~\autocite{Wood2017}. In dit segment wordt daarom vooral de foucs gelegd op de verschillen met Bitcoin. 
		
		In tegenstelling tot de (relatief) eenvoudige bitcoin-transacties, bevatten de blokken die door het Ethereum netwerk wordt opgeslagen iets wat men zou kunnen omschrijven als een toestandsmachine, bestaande uit een lijst van allerhande transacties. Waar de Bitcoin blockchain voornamelijk ontworpen is om fiscale transacties mogelijk te maken, is de Ethereum blockchain meer general-purpose~\autocite{McCorry2017}. Om misbruik en spamming van transacties tegen te gaan en spoedige verwerking te stimuleren is er aan iedere transactie een kleine kostprijs verbonden. Transactie en uitvoerings kosten worden \textit{gas} genoemd en betaald in ether, het cryptogeld van Ethereum. De naam gas is toepasselijk omdat men ether omschrijft als de brandstof waarop het Ethereum netwerk draait (Ethereum Wiki)\footnote{zie https://github.com/ethereum/wiki/wiki/White-Paper\#messages-and-transactions}.
		
		Binnen Ethereum is er sprake van twee soorten accounts: 
		\begin{itemize}
			\item Accounts van externe gebruikers, bestaande uit een publieke en private sleutel, dewelke een gebruiker in zijn bezit heeft.
			\item Contract accounts, smart-contract dewelke bestaan uit code die enkel wordt uitgevoerd bij interactie met gebruikers.
		\end{itemize}	
		Zowel gebruikers als smart-contracts kunnen ether bewaren ~\autocite{McCorry2017}. 
		
		Naast beide accounttypes zijn ook transacties van significant belang voor de werking van Ethereum. De Ethereum blockchain kan gezien worden als een geordende transactie-staat machine~\autocite{McCorry2017}. Net als bij Bitcoin zijn het  de transacties die  de core van het systeem vormen. Ethereum's transacties worden in een blockchain structuur opgeslagen, wat de volledige historiek van transacties oplevert, die  op haar beurt de huidige staat van het netwerk weergeeft.
		
		Een Ethereum-transactie bestaat uit de volgende velden (Ethereum Wiki):
		\begin{itemize}
			\item \textbf{From}: De ondertekening van het account dat de transactie autoriseert, dit kan alleen een gebruiker zijn.
			\item \textbf{To}: De ontvanger van de transactie, dit kan zowel een gebruiker als een contract zijn. 
			\item \textbf{Data}: Een optioneel veld. Hier  kan code meegegeven worden, ofwel voor de creatie van een smart contract, ofwel voor het uitvoeren van een smart contract.Er kan ook andere data worden meegegeven worden.
			\item \textbf{Gas Price}: Bedrag in ether dat de kost voorstelt die de verzender betaald per computationele stap.
			\item \textbf{Start Gas}: Het maximum aantal computationele stappen die mag worden uitgevoerd door de transactie.
			\item \textbf{Amount}: Bedrag in ether dat wordt overgemaakt van zender naar ontvanger.
		\end{itemize}	
		Transacties in Ethereum vinden plaats tussen twee gebruikers of tussen een gebruiker en een smart contract. Een transactie tussen een gebruiker en een smart contract kan één of meerdere nieuwe transacties vanuit het contract naar andere gebruikers doen ontstaan. Tenslotte kan een transactie van een gebruiker naar een smart contract ook transacties naar andere smart contracts triggeren, die dan op hun beurt hetzelfde doen en zo complexe kettingreactie creëren. De combinatie van mogelijke transacties en het potentieel dat smart contracts bieden zorgt voor een systeem waarop in theorie iedere toepassing mogelijk is~\autocite{Wood2017}. 
	\subsection{Conclusie}
		 Ethereum is een platform dat blockchain-technologie beschikbaar en toegankelijk maakt voor gewone ontwikkelaars. Een van Ethereums hoofddoelen is dan ook om multi-purpose te zijn: het is bedoelt als een platform waarop iedere mogelijke blockchain toepassing op geïmplementeerd kan worden. In deze sectie bespraken uitgebreid de filosofie waar Ethereum rond ontworpen is, in het algemeen kan men stellen dat de focus vooral ligt op de functionaliteit en gebruiksvriendelijkheid naar ontwikkelaars toe. Voorbeelden hiervan zijn de keuzes op het vlak van modulariteit en agiliteit. Men probeert duidelijk ook een omgeving te creëren waarin er zoveel mogelijk vrijheid is op vlak van het type van implementatie, geen enkele applicatie wordt geweerd op basis van haar functie. Alleen applicaties de voor schade aan het netwerk zelf zorgen worden niet toegelaten, dit om de werking van het volledige blockchain intact te houden. 
		
		\newpage
\section{Blockchain gebaseerd stemmen}
\label{sec:blockchain-gebaseerd-stemmen}
	\subsection*{Inleiding}
			In het derde een laatste deel van dit hoofdstuk wordt Blockchain gebaseerd stemmen besproken.  Als eerste wordt de noodzaak aangebracht, daarna worden bestaande implementaties besproken. De werking en achterliggende theorie worden als laatste besproken. Concreet bespreken we het Open Vote Network, een stemprotocol dat  aan de hand van smart contracts en Ethereum wordt geïmplementeerd. We bespreken dit protocol omdat het een van de weinige blockchain gebaseerde stemprotocollen is die volledig open source is. Er zijn ongetwijfeld veel andere manieren om een blockchain gebaseerd stemsysteem te realiseren,  hetgeen we hier bespreken is dus maar één voorbeeld, bedoeld ter illustratie.  In het vorige hoofdstuk bespraken zowel Ethereum als smart contracts reeds uitvoerig, in dit hoofdstuk zal de focus daarom meer op het cryptografische aspect liggen van het stemmen liggen. De werking van het stemprotocol wordt stap voor stap uitgelegd. De volledige natuur van de onderliggende cryptografie valt vaak buiten de scope van deze bachelorproef, bepaalde cryptografische technieken en wiskundige concepten worden daarom enkel vermeld en niet verder uitgewerkt. Tot slot wordt ook een selectie van bestaande implementaties besproken.
			Het concept blockchain gebaseerd stemmen bevind zich duidelijk nog in een vroeg stadium, men kan het zien als een onderdeel van wat \textcite{Swan2015} classificeert als blockchain 3.0 ofwel de blockchain implementaties van de toekomst.
	\subsection{Noodzaak}
			\subsubsection{Democratie en stemmen}
			Democratie wordt gedefinieerd als: ``een systeem waar bestuur van een natie door de volledige bevolking gebeurt, of op zijn minst door de in aanmerking komende leden er van''\footnote{ Definitie van democratie door Oxford Dictonairies, vertaalt uit het Engels. Verkregen op 1 Mei 2019 van https://en.oxforddictionaries.com/definition/democracy.}. In de meeste democratieën wordt het bestuur niet op een directe wijze door de bevolking uitgeoefend, maar via verkozen vertegenwoordigers. Om een democratie te laten functioneren moeten er dus processen zijn om vertegenwoordigers aan te duiden.  Aan de basis van elke succesvolle democratie ligt daarom de voorwaarde dat men kan stemmen op een wijze die toegankelijk, veilig en correct is.~\autocite{Osgood2016}. 
			
			Klassieke verkiezingen gebeuren aan de hand van een stem op papier: Men duidt de voorkeur aan op een stembiljet, vouwt het dicht en werp dit vervolgens in een `zwarte doos' waar alle stembiljetten ongeopend in blijven liggen tot ze geteld worden. Het tellen gebeurt handmatig, ambtenaren of burgers worden door een centraal stembureau opgeroepen om lokale stemmen te tellen. De resultaten worden vervolgens doorgeven aan hogere niveaus tot er uiteindelijk een volledig resultaat bekend is\footnote{ Omschrijving van hoe stemprocessen op papier doorgaans verlopen, lokale variaties zijn uiteraard mogelijk}.
			
			\subsubsection{Elektronisch stemmen}
			Hoewel de papieren stem al eeuwen gebruikt wordt zijn er toch een aantal problemen aan verbonden. In recente decennia heeft het zogenaamde elektronisch stemmen dan ook een enorme opkomst gekend. Ook in België, met name in het Vlaamse Gewest is dat het geval: in maar liefst de helft (157/300) van de Vlaamse gemeenten wordt vandaag gebruik gemaakt van stemcomputers (figuur \ref{fig:evote_vlaanderen}).
			
			\begin{figure}
				\includegraphics[width=\linewidth]{img/evote_vlaanderen.png}
				\caption{De helft van de Vlaamse gemeenten stemt digitaal (2018)}
				\label{fig:evote_vlaanderen}
			\end{figure}
			
			Volgens Smartmatic, leverancier van de stemcomputers in België en vele andere landen, biedt de elektronische stem een aantal aanzienlijke voordelen ten opzichte van de klassieke, papieren stem. Een van de voornaamste argumenten is dat de correctheid van de papieren stem in vraag kan worden gesteld gezien het proces bijzonder gevoelig is aan menselijke fouten, het zij opzettelijk of onopzettelijk. Een ander argument is dat de papieren stem veel minder \textit{toegankelijk} en \textit{efficiënt} is. Elektronisch stemmen daarentegen zou het stemproces voor vrijwel iedereen toegankelijk maken, ook personen met een lichamelijke of verstandelijke beperking zouden erdoor zelfstandig èn met behoud van het stemgeheim hun stem kunnen uitbrengen. Tenslotte is de elektronisch stem volgens Smartmatic ook efficiënter (en dus economisch voordeliger) dan de papieren variant. Elektronisch stemmen is namelijk veel sneller, het stelt grote landen in staat om de resultaten van hun verkiezingen in enkele uren te bereken, in de plaats van enkele weken.
			
			\subsubsection{Problemen met elektronisch stemmen}
			Hoewel de genoemde argumenten voor elektronisch stemmen grond hebben, zijn er ook argumenten tegen het elektronisch stemmen te voeren. Uit het onderzoek \cite{Norden2015} blijkt bijvoorbeeld dat een groot deel van de stemcomputers die worden gebruikt voor verkiezingen in de VS sterk verouderd zijn. Naar schatting gebruikte in 2015 maar liefst 43 van de 50 staten nog stemcomputers die op al minstens 10 jaar oud waren, in 14 van de staten zouden er zelfs machines zijn geweest die al meer dan 15 jaar oud waren. De leeftijd van stemcomputers zorgde volgens \cite{Norden2015} tijdens  congresverkiezingen van 2014 voor problemen gaande van vastlopende en uitvallende machines met als gevolg  lange wachtrijen, tot verloren en omgedraaide stemmen (op een andere kandidaat) met als gevolg verkeerde resultaten.
			
			Verouderde software zorgt er voor dat machines die gebruik maken van internet connecties bijzonder kwetsbaar zijn. De beveiligingen van zulke machines zijn niet meer adequaat volgens hedendaagse encryptie standaarden. Externe partijen kunnen in theorie zonder veel moeite toegang tot de zulke stemcomputers krijgen tijdens verkiezingen. Verouderde hardware zorgt er dan weer voor dat reparatie- en onderhoudskosten voor oude stemmachines bijzonder hoog zijn~\autocite{Norden2015}.
			
			Ook in België en Nederland krijgt elektronisch stemmen vaak de wind van voren. Tijdens de Belgische federale verkiezingen van 2014 bleek bijvoorbeeld dat er bij 57 stemcomputers een probleem had plaatsgevonden waardoor in totaal 2250 stemmen verloren gingen. In reactie daarop besliste Wallonië in 2015 voor een volledige terugkeer naar de papieren stem~\autocite{Maddens2018}.  In Nederland werden stemcomputers reeds in 1996 geïntroduceerd, maar ook hier zorgde toenemende problemen uiteindelijk in 2009 voor een terugkeer naar potlood en papier ~\autocite{Schellevis2018}.
			
			\subsubsection{Blockchain gebaseerd stemmen}
			Het is duidelijk dat er goede argumenten voor en tegen zowel papier als elektronisch stemmen kunnen gevoerd worden. Een blockchain gebaseerd stemsysteem zou volgens proponenten van de technologie een nieuwe manier kunnen vormen om elektronische stemmen te realiseren. Het zou veel van de huidige nadelen en problemen van elektronisch stemmen kunnen elimineren  en tegelijk ook de bestaande voordelen ten opzichte van stemmen op papier behouden. De gedistribueerde aard van blockchain zou er bovendien voor kunnen zorgen dat verkiezingen transparanter, veiliger en correcter dan ooit worden. 
			 
	\subsection{Bestaande Implementaties}
			\subsubsection{Concrete Toepassingen}
			Tot op heden zijn operationele toepassingen van blockchain stemsystemen nog relatief zeldzaam.  Er dient wel  een onderscheid gemaakt te worden tussen de verschillende soorten toepassingen. Men kan  blockchain technologie immers op diverse wijzen inzetten in de context van verkiezingen. In veel gevallen wordt een blockchain enkel ter ondersteuning van een groter stemsysteem gebruikt. Dit kan bijvoorbeeld zijn door blockchain als  een onveranderlijke opslagplaats te gebruiken voor de resultaten van een verkiezing. Een andere mogelijkheid is om blockchain technologie als verificatie tool te gebruiken. Er zijn echter ook implementaties die volledig blockchain gebaseerd zijn. Voor dit onderzoek zijn deze implementaties natuurlijk het interessantst ~\autocite{Kshetri2018}.
			
			In de volgende paragrafen worden 4 recente cases besproken waarin blockhain stemsystemen werden gebruikt ~\autocite{Kshetri2018}:
			\begin{enumerate}
				\item Parlements verkiezingen in Sierra Leone
				\item De jaarlijkse algemene vergadering van het Estse technologiebedrijf LVH Groep
				\item De gemeenschapsprojecten in Zuid-Koreaanse provincie Gyeonggi-do
				\item Het Active Citizen-programma van de stad Moskou
			\end{enumerate}
				
				\paragraph{Parlements verkiezingen in Sierra Leone}
				In maart 2018 werd in Sierra Leone blockchain technologie gebruikt om een kleine deelset van de resultaten van de nationale verkiezingen te controleren. Het systeem werd opgezet door het Zwitsers bedrijf Agora, dat aanwezig was als een internationale observator van de verkiezingen. Agora creëerde veel oproer door miscommunicaties de wereld in te sturen die  deden lijken dat de hele verkiezing op hun blockchain systeem was gebeurd. In feite deed Agora's implementatie enkel het tellen en bewaren van resultaten op basis van een blockchain. Bovendien werd slechts een fractie van de populatie geobserveerd, de manier waarop dit gebeurde was daarbovenop ook erg foutgevoelig. Men verzamelde, onafhankelijk van de verkiezingen, via bevraging het stemgedrag van ruim 400,000 kiezers. Tijdens dit verzamelen werd de data live aan een blockchain systeem gevoed. Het resultaat werd op gedistribueerde wijze berekend, ettelijke dagen voordat de klassieke telling afgerond was. De resultaten van Agora's telling kwam echter niet overeen met die van de overheid. Dit voorbeeld, hoewel niet bepaald  inspirerende op vlak van correctheid en fouttolerantie, is toch opmerkelijk omdat het wel aantoont hoe veel sneller en efficiënter blockchain systemen kunnen werken ten opzichte van klassieke manuele tellingen. ~\autocite{Kshetri2018}
				
				\paragraph{De jaarlijkse algemene vergadering van het Estse technologiebedrijf LVH Groep}
				Aandeelhouders van de Esthetische onderneming LVH Groep maken sinds 2015 gebruik van een kleinschalig online blockchain gebaseerd stemsysteem ~\autocite{Kshetri2018}. LVH Groep is een bankiers en financiële diensten bedrijf dat zich gespecialiseerd heeft in blockchain technologie, onder meer door een Bitcoin-wallet te ontwikkelen\footnote{ttps://www.coindesk.com/lhv-bank-backs-wallet-app-built-on-bitcoins-blockchain}. Doordat LVH groep een internationale onderneming is met kantoren in verscheidene landen, was het enorm kostlijk en tijdsrovend om op jaarlijkse basis alle aandeelhouders op een plaats te verzamelen voor een algemene vergadering. Er was  nood aan een veilig systeem om beslissingen over het internet te kunnen maken. Hiertoe werd in samenwerking met onder andere de bedrijven NASDAQ en Chain een systeem geïmplementeerd op de Bitcoin blockchain dat werkt op basis van smart contracts. Verificatie van de kiezers wordt gedaan door gebruik te maken van het bestaande digitale identiteitssysteem in Estland \footnote{http://francescoolcelli.blogspot.com/2017/01/is-blockchain-answer-to-e-voting-nasdaq.html|}. Bijzonder aan deze implementatie was dat ze aantoonde dat ook zonder directe ondersteuning van het platform, er zeer veilige en betrouwbare implementaties kunnen gemaakt worden op blockchain. ~\autocite{Kshetri2018}
				
				\paragraph{De gemeenschapsprojecten in Zuid-Koreaanse provincie Gyeonggi-do}
				In maart 2017 maakte het provinciebestuur van de Zuid-Koreaanse provincie Gyeonggi-do gebruik van een blockchain implementatie om burgers via directe verkiezingen te laten  bepalen welke gemeenschapsprojecten gefinancierd dienden te worden. Burgers konden zelf voorstellen doen en op suggesties van medeburgers stemmen. Het overkoepelende Ddabok Community Support Project zag maar liefst 9000 inwoners participeren. In totaal werden rond de 500  projecten geselecteerd voor subsidies. Het bijzondere aan deze implementatie was dat hoewel, georganiseerd vanuit de overheid, er geen enkele centrale autoriteit in het stemproces was betrokken. Het hele systeem werd ontwikkeld door Blocko, de grootste Zuid Koreaanse blockchain firma. ~\autocite{Kshetri2018}
				
				\paragraph{Het Active Citizen-programma van de stad Moskou}
				De stad Moskou kent sinds 2014 een programma genaamd Active Citizen waarbij haar inwoners via zowel mobiele als online applicaties (zie figuur \ref{fig:active_citizen}) kunnen stemmen over allerhande maatregelen, gaande van de naam voor een nieuwe metro-trein tot de kleuren van zetels in een nieuw sport-stadion. Het Active Citizen-programma telt meer dan 2 miljoen geregistreerde kiezers, er werden tot heden al meer dan 92 miljoen stemmen op uitgebracht. De laatste jaren is er echter steeds minder vertrouwen in de stad wanneer het op het tellen van de stemmen aankomt. Om de inwoners gerust te stellen en vertrouwen terug te winnen werd  daarom in 2017 een private versie van de Ethereum blockchain aan de architectuur van het project toegevoegd. ~\autocite{Kshetri2018}
				
				\begin{figure}
					\includegraphics[width=\linewidth]{img/active_citizen.png}
					\caption{Het Active Citizen programma via een mobiele applicatie}
					\label{fig:active_citizen}
				\end{figure}
				
				Zoals vermeld gaat het om een Ethereum implementatie. Er wordt gewerkt aan de hand van smart contracts om de stemmen te registreren. Alle stemmen komen terecht in een  blockchain, het resultaat van alle verkiezingen is publiek beschikbaar. Ook de broncode voor de implementatie werd publiek gemaakt op GitHub.
				
				De populairste onderwerpen zagen tussen de 137.000 tot 220.000 participanten. Tijdens testen was de  Ethereum implementatie  instaat om tot 1000 transacties per minuut te verwerken. Het is niet duidelijk of de implementatie overweg zou kunnen met  grotere volumes van kiezers. De schaalbaarheid is in deze context wel noodzakelijk, men wil een systeem bekomen dat toegankelijk is voor iedere bewoner en Moskou telt immers een 12 miljoen inwoners. De kans is groot dat het huidige systeem overrompelt zou raken en compleet zou verstoppen moesten miljoenen kiezers gelijktijdig hun stem uitbrengen.
				
				Op het vlak van security is schaalbaarheid dan weer geen probleem. De stads administratie liet het systeem testen door PwC, een internationaal accountants- en belastingadviseursbedrijf. De bevindingen van PwC waren dat er geen reden tot bezorgdheid is voor de veiligheid bij  300.000 of meer kiezers. 
				
				Al bij al levert dit voorbeeld toch een sterke case op voor de blockchain gebaseerd stemmen op grote schaal. Het toont aan dat men met voldoende middelen blockchain stemprotocollen kan ontwikkelen die qua scope de klassieke board-room voting systemen uit de literatuur vele malen overtreffen. 
			\subsubsection{Online Diensten}
				\paragraph{The Blockchain Voting Machine }
					Een van de grootste argumenten tegen blockchain gebaseerd stemmen is dat het een verhoogd risico met zich meebrengt, de bestaat  dat de nodes overgenomen worden of al overgenomen zijn. Voor een systeem dat werkt over het internet zou het gevaar van een nog grotere orde zijn. The Blockchain Voting Machine is hier heel simpele antwoord op: men stelt voor om te werken met elektronische stemmachines zoals die vandaag al bestaan,   de machines zijn echter niet aan het internet verbonden alleen aan een privaat blockchain netwerk. 
				\paragraph{FollowMyVote}
					Het online stemplatform FollowMyVote hanteert de tegenovergestelde redenatie van The Blockchain Voting Machine. Het uitgangspunt van dit opensource platform is net dat blockchain de ideale technologie is om verkiezingen niet alleen elektronisch maar ook online te maken. Online-verkiezingen zijn nodig volgens FollowMyVote omdat het de kosten van verkiezingen enorm kan onderdrukken en de opkomst van kiezers vele malen groter kan maken.  Zwakkere groepen zoals bejaarden of invaliden, maar ook mensen in het buitenland kunnen nu  meestemmen.  Op het vlak van identificatie stelt men voor om te werken met behulp van webcams en door de overheid uitgegeven identiteitsbewijzen. Het resultaat van verkiezingen wordt in de blockchain opgeslagen, de resultaten zijn publiek verifieerbaar. FollowMyVote stelt ook een reeks andere functionaliteiten waaronder: real-time verkiezingen volgen en een stem die binnen de verkiezingstermijn aangepast kan worden. 
				\paragraph{TIVI}
					TIVI hanteert heeft ongeveer dezelfde motivatie als FollowMyVote, maar op vlak van identificatie neemt men het hier nog een stap verder. Men stelt voor om stemmen mogelijk te maken vanaf eender welk platform, zij het laptop, mobiel of tablet. Verificatie kan volledig via gezichtsherkenning gebeuren volgens TIVI, zodat men zich kan identificeren door een selfie te nemen.
				\paragraph{Agora}
					Agora is een Zwitsers bedrijf dat een platform ontwikkeld heeft voor blockchain gebaseerde stemsystemen. Agora heeft een eigen custom blockchain infrastructuur opgezet en bouwt middel tot grote verkiezingstoepassing op maat voor haar klanten. Agora geldt momenteel als een van de grootste spelers in de sector.
					
	\subsection{Cryptografie en Stemprotocollen}
		\subsubsection{Verschillende Protocollen}
		\textcite{Kiayias2002} stelt voor om stemprotocollen self-tallying te maken. Self-tallying ofwel zelf-tellend wordt hier gedefinieerd als volgt: iedere kiezer moet na het einde van een stemming zelf de stemmen kunnen tellen. Zelf-tellende stem-protocollen nemen de verantwoordelijk voor het tellen van stemmen weg van centrale autoriteiten en veranderen het in een open procedure, die iedere deelnemer of derde partij kan uitvoeren. De centrale tellingsautoriteit is dan niet meer nodig, iedereen kan resultaat van de stemming onafhankelijk bekomen. Men definieert ook twee andere eigenschappen waaraan een elektronisch stem-protocol moet voldoen.  Perfect Ballot Secrecy ofwel volledige stembescherming, wordt gedefinieerd als een eigenschap die de verzekering brengt dat de privacy van de stemmer enkel en alleen kan ondermijnd worden als alle andere stemmers tot dit doeleinde collaboreren. Dispute-freeness ofwel geschil-vrijheid, wordt dan weer gedefinieerd als een eigenschap die garandeert dat dat er geen twijfel kan zijn over de authenticiteit van de stemming. Deze eigenschap bestaat eruit dat iedere participant van de stemming, na afloop het correcte verloop van de procedure kan verifiëren voor zichzelf.
			
		Volgens \textcite{McCorry2017}  hebben deze zelf-tellende protocols zwaktepunten op het vlak van eerlijkheid. Ze laten toe dat de alle personen die gestemd hebben op een bepaald tijdstip, zouden kunnen collaboreren om de tussen-resultaten te berekenen voor datzelfde tijdstip. Verder kan de laatste persoon die zijn of haar stem moet uitbrengen het resultaat van de stemming berekenen voor hij of zij effectief gestemd heeft. Dit leidt volgens \textcite{McCorry2017} tot adaptive en abortive issues.  \textcite{McCorry2017} spreekt van een adaptive issue waar de stemkeuze van de laatste participant mogelijks beïnvloed kan worden door het zien van de stemresultaten voor zelf te stemmen. Er is ook sprake van een abortive issue volgens \textcite{McCorry2017} omdat iedere participant de macht heeft om de hele stemming te annuleren. Een enkele participant kan zich immers volledig kunnen onthouden van het stemmen en zo alle andere stemmers verhinderen het resultaat te berekenen. 
			
		\textcite{Kiayias2002} stelt dat het bovenstaande probleem makkelijk te corrigeren valt door middel van een extra stemronde. \textcite{McCorry2017} stelt echter dat daarvoor de volledige coöperatie van alle participanten nodig is, en dat deze niet meer gegarandeerd is op dit punt. Ook wordt er voorgesteld om de laatste stem steeds een lege stem van de organisator te laten zijn, maar ook hiertegen verzet \textcite{McCorry2017} zich, men stelt dat dit in essentie een terugkeer is naar een systeem met centrale autoriteit.  \textcite{McCorry2017} stelt een protocol voor, op basis van het werk van Kiayias and Yung en Groth maar combineert als eerste het concept van self-tallying met een blockchain implementatie. Het resulteren protocol noemt men het \textit{Open Vote Network Protocol} (OVNP).
		
		 \textcite{McCorry2017} maakt gebruik van Ethereum als ontwikkelplatform voor het OVNP. In sectie \ref{sec:ethereum-en-smart-contracts} bespraken we de verschillende voordelen die dit platform biedt reeds uitgebreid. Ook \textcite{Dagher2018} stelt met BroncoVote een stemsysteem voor dat werkt via Ethereum. BroncoVote is blockchain stemsysteem dat gebaseerd is op andere cryptografische methodes dan OVNP en bijgevolg sterk verschilt met protcollen van \textcite{Kiayias2002} en \textcite{McCorry2017}. 
	
	\subsection{Het Open Vote Network Protocol}
		\subsubsection*{Overzicht }
			Het Open Vote Network Protocol is een gedecentraliseerd protocol, ontworpen op basis van het self-tallying principe. De focus ligt hier op de bescherming van privacy en de robuustheid, niet op de schaalbaarheid. Self-tallying gebeurt immers niet op een wijze die bijzonder schaalbaar is. Het  Open Vote Network protocol ondersteund bijgevolg enkel kleine verkiezingen met tientallen participanten, zaken zoals nationale verkiezingen zijn momenteel  zo goed als onmogelijk. Om eenvoud te bewaren wordt er gewerkt met het simpelste verkiezingspatroon dat mogelijk is, deelnemers hebben de keuze tussen twee opties, bijvoorbeeld een ja/nee vraag.  \textcite{McCorry2017} verwijst naar het onderzoek van \textcite{Hao2009} voor zogenaamde multi-way verkiezingen ofwel verkiezingen met meerdere keuze-opties. 
			
			Het stemmen in dit protocol gebeurt in twee fasen, in de eerste fase laten alle kiezers zich registreren, in de tweede fase wordt de effectieve stem uitgebracht. Na afloop van de tweede fase kunnen de stemmen geteld worden.  De self-tallying eigenschap stelt iedere stakeholder die het protocol uitvoert ertoe instaat om zelf de stemmen te tellen. Merk op dat iedereen dit facet van het protocol kan uitvoeren, niet enkel de geregistreerde stemmers.
			 
			Het gedecentraliseerde karakter van dit protocol maakt het uiterst geschikt om op een blockchain te implementeren. \textcite{McCorry2017} is een van de eersten in de literatuur die deze stap maakt. De reden dat men specifiek voor de Ethereum blockchain kiest, wordt gemotiveerd als volgt.
			
			Andere blockhains zoals Bitcoin zouden ook kunnen worden gebruikt als publieke opslagplaats van verkiezingen, maar het verschil is dat het protocol niet op de blockchain kan opgeslagen worden, het moet extern dan opgeslagen en gehandhaafd worden door de kiezers. \textcite{McCorry2017} stelt dat men via Ethereum gebruik kan maken van een smart-contract om  het protocol af te dwingen, Ethereum kan daarnaast niet alleen als opslag fungeren maar ook als geverifieerd netwerk waarover de participanten communiceren.
		\subsubsection*{Fase 0: Setup }
			Het OVN-Protocol start vanuit een zogenaamde \textit{election administator}, deze organisator van de verkiezingen initialiseert het protocol door de een verkiezing aan te maken en de in aanmerking komende kiezers in te stellen, dit gebeurt door de kiezers aan de white-list van het smart-contract toe te voegen. De kiezers worden in dit stadium enkel geïdentificeerd aan de hand van hun Ethereum account. De organisator is meestal degene die  Ethereum verwittigt om over te schakelen op de eerste fase.
		\subsubsection*{Fase 1: Registratie }
			De administrator stelt het onderwerp van de verkiezing in, als ook de beschikbare opties waarvoor gestemd kan worden. Vervolgens wordt Ethereum opnieuw verwittigd, ditmaal om over te gaan naar de registratie van de participanten. Om te registreren voor de verkiezing moeten alle stemmers vooraf een stemsleutel berekenen. Deze sleutel zal worden gebruikt als identificatie van de kiezers. Het berekenen  van de stemsleutel en het verdere verloop (samengevat in figuur \ref{fig:ovnp}) gaat als volgt: 
			
			Alle \textbf{\textit{n}} kiezers moeten het eens zijn over het paar \textbf{\textit{(G,g)}} waarbij \textbf{\textit{G}} een eindige cyclische groep van hoofdorde \textbf{\textit{q}} voorstelt waar het Diffie-Hellman (DDH) probleem niet van toepassing op is, en \textbf{\textit{g}} een generator in \textbf{\textit{G}}. De lijst van in aanmerking komende kiezers, hier ook participanten \textbf{\textit{(P$_{1}$, P$_{2}$, … , P${n}$)}} genoemd, wordt vastgelegd. Elke participant \textbf{\textit{P$_{i}$}} kiest een unieke random waarde  
			
			\textbf{\textit{$x_{i} \in^{R} Z_{q}$}}.
			
			 De waarde \textbf{\textit{x$_{i}$}} wordt gebruikt als private stemsleutel. Iedere kiezer vormt zijn of haar publieke stemsleutel door \textbf{\textit{g$^{x_{i}}$}} te berekenen. Gezien \textbf{\textit{g}} een generator is van \textbf{\textit{G}}, valt niet te achterhalen aan de hand van \textbf{\textit{g$^{x_{i}}$}} wat de eigenlijke waarde van \textbf{\textit{x$_{i}$}} is. Het Ethereum netwerk heeft met andere woorden geen kennis van \textbf{\textit{x$_{i}$}}. Om de participant \textbf{\textit{P$_{i}$}} te kunnen identificeren in het netwerk volstaat de publieke sleutel echter niet, er is ook een bewijs van de private sleutel \textbf{\textit{x$_{i}$}} nodig. Tot dat doeleinde wordt een Zero Knowledge Proof gebruikt. Met \textbf{\textit{ZKP(x$_{i}$)}} levert de participant \textbf{\textit{P$_{i}$}}  het bewijs aan het netwerk dat de \textbf{\textit{x$_{i}$}}  in \textbf{\textit{g$^{x_{i}}$}} gekend is door \textbf{\textit{P$_{i}$}}. Zonder de waarde voor \textbf{\textit{x$_{i}$}}  effectief bekend te maken, wordt er zo bewezen dat \textbf{\textit{P$_{i}$}} de eigenaar van de publieke sleutel \textbf{\textit{g$^{x_{i}}$}} is.
			
			Registratie gebeurt door iedere kiezer \textbf{\textit{P$_{i}$}} te laten broadcasten naar het netwerk. De broadcast bestaat uit \textbf{\textit{g$^{x_{i}}$}}  en \textbf{\textit{ZKP(x$_{i}$)}}. Daarnaast wordt er ook een constante waarde in ether uit de portefeuille van de kiezer aan de broadcast toegevoegd. Het gaat hier om de som van de transactie kosten en een waarborg die de kiezer terugkrijgt. De waarborg instellen gebeurt omdat blockchain filosofie dicteert dat het verbinden aan een potentiele kost participanten zal stimuleren om oprecht te handelen, in dit geval geeft men de waarborg terug van zodra ze hun stem uitbrengen. 
			
			Nadat een kiezer \textbf{\textit{P$_{i}$}}  gebroadcast heeft naar Ethereum zal het netwerk zijn broadcast verwerken. Het bewijs in de vorm van de \textbf{\textit{ZKP(x$_{i}$)}} wordt gecontroleerd, de waarborg wordt opgeslagen en tot slot wordt de  stemsleutel \textbf{\textit{g$^{x_{i}}$}} gebruikt om een nieuwe sleutel te berekenen waarmee de kiezer \textbf{\textit{P$_{i}$}} een stem zal kunnen uitbrengen over het ingestelde onderwerp. Deze nieuwe sleutel noemt men de gereconstrueerde sleutel \textbf{\textit{g$^{y_{i}}$}}, ofwel \textbf{\textit{Y$_{i}$}} en wordt berekend als volgt: 
			\begin{ceqn}
				\begin{align}
					Y_{i} = \prod_{j=1}^{i-1}g^{x_{j}}  / \prod_{j=i+1}^{n}g^{x_{j}} \label{formula1}\
				\end{align}
			\end{ceqn}
			
			\eqref{formula1} \textit{Om de gereconstrueerde sleutel g$^{y_{i}}$ te berekenen van kiezer \textbf{P$_{i}$} worden de publieke stemsleutels van alle andere kiezers \textbf{P$_{j}$} gebruikt. Het product van de publieke sleutels van al de andere kiezers \textbf{P$_{j}$} waar j < i wordt gedeeld door het product van alle publieke sleutels van al de andere kiezers waar \textbf{P$_{j}$} waar  j > i.}
			
		\subsubsection*{Fase 2: Stemmen}
			Eenmaal het netwerk signaal ontvangt dat er gestemd kan worden, kunnen de kiezers een beslissing maken en vervolgens hun stem te broadcasten. Deze broadcast bestaat uit twee zaken: de geëncrypteerde stem en een nieuwe Zero Knowledge Proof, die bewijst dat de stem ofwel 0 of 1 is. De encryptie die gebruikt wordt voor de stem vormt het cryptografische hart van dit protocol. 
			
			De stem wordt geëncrypteerd volgens het Elgamal-cryptosysteem en wordt in de vorm  \textbf{\textit{g$^{x_{i}y_{i}}$g$^{v_{i}}$}} aan het netwerk doorgegeven,  waarbij  \textbf{\textit{g$^{x_{i}y_{i}}$}} het product is van de publieke stemsleutel en de gereconstrueerde sleutel, \textbf{\textit{v$_{i}$}} de eigenlijke stem is in de vorm van een 1 of een 0 en \textbf{\textit{g$^{v_{i}}$}} de factor waarmee het product van de sleutels vermenigvuldigt wordt, zijnde ofwel g ofwel 1. 
			
			Eenmaal \textbf{\textit{P$_{i}$}} de stem broadcast wordt aan de hand van de \textbf{\textit{ZKP(v$_{i}$)}} geverifieerd of de stem in een correct formaat is (0 of 1), is dit het geval dan wordt de Elgamal-geëncrypteerde stem opgeslagen in de Ethereum blockchain en wordt de waarborg aan \textbf{\textit{P$_{i}$}} geretourneerd.
			
			Wanneer alle n stemmen ontvangen en geverifieerd zijn krijgt iedereen er toegang toe. De identiteit van de kiezers noch de betekenis is door de encryptie niet meer af te leiden uit een particuliere stem. Alleen het resultaat kan nog bekomen worden uit de nu persistente geëncrypteerde stemmen.
			
			Het resultaat van de verkiezing bekomt men door het product van alle geëncrypteerde stemmen te nemen \eqref{formula2}: 	
			\begin{ceqn}
				\begin{align}
				\prod_{i=1}^{n}g^{x_{i}y_{i}}g^{v_{i}} \label{formula2}\
				\end{align}
			\end{ceqn}	
			De aard van de encryptie is van zodanig dat alle random factoren geëlimineerd worden \eqref{formula3}, zodat :	
			\begin{ceqn}
				\begin{align}
				\prod_{i=1}^{n}g^{x_{i}y_{i}} = 1 \label{formula3}\
				\end{align}
			\end{ceqn}
			Dit resulteert in:
			\begin{ceqn}
				\begin{align}
				g^{\sum_{i=1}^{n}v_{i}} \label{formula4}\
				\end{align}
			\end{ceqn}
			Waarbij \textbf{\textit{$\sum_{i=1}^{n}v_{i}$}} in \eqref{formula4}, het aantal stemmen voor waarde 1 representeert. Het aantal waar-stemmen kan echter niet rechtstreeks worden afgeleid, \textbf{\textit{g}}  wegwerken impliceert een discreet logaritme. Gezien de rechterhelft van de vergelijking bekend is en \textbf{\textit{i}} gelimiteerd is tot het aantal participanten \textbf{\textit{n}}, is het vrij zeker dat het hier om een kleine waarde gaat. Deze waarde kan gemakkelijk gevonden worden door middel van \textit{brute-force} zoeken. Eenmaal het aantal ja-stemmen gevonden is, wordt het vinden van de nee-stemmen triviaal.
			
			Merk op dat het noodzakelijk is bij self-tallying stemprotocollen dat iedere kiezer die in fase 1 een stemsleutel broadcast ook een geëncrypteerde stem uitstuurt in fase 2. Anders kunnen de resultaten van de stemming niet berekend worden. Er is hier dus nog steeds sprake van een \textit{abortive issue}.
			
			Ook is er het feit dat bij self-tallying stemprotocollen de allerlaatste kiezer mogelijks de stemmen kan tellen alvorens zelf te kiezen. Door een 0-stem te simuleren, kan deze kiezer het resultaat van de verkiezingen berekenen voordat alle anderen dit kunnen. De kiezer kan dus potentieel zijn of haar stem nog veranderen op basis van het voorlopige resultaat van de verkiezing. In de meeste gevallen is dit natuurlijk geen wenselijke situatie. De oplossingen die voor dit probleem in de literatuur worden voorgesteld zijn echter ook sub-optimaal volgens \textcite{McCorry2017}. 
			
			\textcite{McCorry2017} stelt voor om een fase toe te voegen tussen fases 1 en 2. In deze fase wordt de stem van de kiezer opgeslagen voor bekendmaking aan het netwerk, eenmaal opgeslagen kan de stem niet meer veranderen. De volgende fase bestaat er dan uit dat de stem bekent wordt gemaakt aan het netwerk. 

			\begin{figure}
				\includegraphics[width=\linewidth]{img/ovnp.png}
				\caption{Het Open Vote Network Protocol samengevat}
				\label{fig:ovnp}
			\end{figure} 
	
	\newpage
	\subsection{Privacy en transparantie}
		Typische elektronisch stem-protocollen beschermen de privacy van kiezers door gebruik te maken van een centrale autoriteit, om veiligheidsredenen is de autoriteit vaak verdeelt over verschillende tellings-autoriteiten. Een voorbeeld hiervan vinden we terug bij het stemsysteem van Helios\footnote{https://github.com/benadida/helios-server}. Elektronische stemmen worden hier door meerdere tellings-autoriteiten behandeld, in combinatie met cryptografie reduceert men het risico op aantasting van de gebruikers privacy aanzienlijk ~\autocite{Adida2008}. 
	
		Los van dit gegeven echter, staat het feit dat een dergelijk systeem gebaseerd op het vertrouwen van de stemmers. De mogelijkheid bestaat nog steeds dat alle tellingsautoriteiten van kwade wil zijn of dat ze simpelweg overrompeld worden door aanvallers. In zo’n situatie kan de privacy van de gebruikers alsnog geschaad zijn~\autocite{McCorry2017}.
		
		In essentie gaat het hier om een probleem op het vlak van transparantie, de gebruiker van het systeem heeft immers weinig tot geen middelen om zichzelf er van te verzekeren dat het hele proces correct verloopt~\autocite{McCorry2017}. 
		
		We bespraken verschillende manieren waarop blockchain technologie kan aangewend worden in elektronische stemsystemen. De verschillende wijzes die we bespraken kennen enkele noemenswaardige verschillen op het vlak van privacy. Het Open Vote Network dat voorgesteld wordt door \textcite{McCorry2017} vormt een van de meest robuuste structuren die de privacy van gebruikers garanderen. Volledige anonimiteit, zoals men heeft wanneer men een papieren stem in de stembus werpt, is niet mogelijk op blockchain, maar op basis van cryptografie slaagt \textcite{McCorry2017} er wel in om het vinden van andermans identiteit praktisch onmogelijk te maken. Merk op dat  de term identiteit hier niet persoonlijke gegevens van de gebruiker betekent, wel het Ethereum account van de gebruiker.
		
		Ook  de andere systemen die we bespraken, Agora, FollowMyVote, enz. lijken op het eerste zicht relatief goed te scoren op het vlak van privacy. Toch is er bij veel van de besproken voorbeelden een fundamenteel probleem op het vlak van centralisatie, aldus \textcite{McCorry2017}. De meeste voorgestelde oplossingen zijn nog steeds, in variërende mate, afhankelijk van een centrale autoriteit. Men zou het argument kunnen voeren dat bijvoorbeeld een bedrijf zoals FollowMyVote, hoewel hun hele stemproces op de blockchain gebeurd, op een bepaalde manier zelf een centrale autoriteit wordt wanneer men van hun diensten wil gebruik maken. Blockchain gebaseerde stemsystemen waarvan de codebasis  niet open source is, vergen daarboven op een bepaalde manier ook weer het vertrouwen van de gebruiker. Wanneer stemsystemen ook toegang krijgen tot gevoelige informatie zoals elektronische identiteitsgegevens of zelfs biometrische-data voor gezichtsherkenning, zoals het bedrijf TIVI voorstelt, dan speelt dit vertrouwen een rol die zodanig cruciaal is dat er eigenlijk sprake is van een centrale autoriteit.
	\subsection{Fouttolerantie, Veiligheid en Correctheid}
	
	Naast privacy en transparantie zijn ook fouttolerantie, veiligheid en correctheid van centraal belang in de context van stemprotocollen. Deze drie zaken zijn inherent aan elkaar verbonden, er is zelfs sprake van zekere grote overlapping. We definiëren kort de bovenstaande termen voor we bekijken hoe ze van toepassing zijn op blockchain-gebaseerde stemsystemen. \textit{Fouttolerantie} gaat in de eerste plaats over de robuustheid  van het systeem, het is de mate waarin het systeem bestendig is tegen allerhande fouten die zich kunnen voordoen. Een belangrijke distinctie is dat we het hier enkel over onopzettelijk fouten hebben. De mate waarin opzettelijk fouten en andere malafide acties een impact hebben op het systeem wordt gedefinieerd als de \textit{veiligheid}. Tenslotte definiëren we \textit{correctheid} als de mate waarin het berekende resultaat overeenkomt met de werkelijkheid en dit onder invloed van factoren zoals fouttolerantie en veiligheid.
	
	Op het vlak van fouttolerantie zijn er een paar inherente problemen met het OVN-protocol dat \textcite{McCorry2017} voorstelt. De voornaamste problemen zijn zogenaamde \textit{abortive issues} die te maken hebben met het missen van deadlines. Zoals vermeld wordt er in het OVN gewerkt met verschilleden fases, alle kiezers worden steeds veronderstelt om binnen deze fases bepaalde acties te ondernemen. Een kiezer die per ongeluk zo'n een deadline mist, bijvoorbeeld door het uitvallen van de netwerk-connectie, verliest niet alleen zijn voorafbetaalde waarborg, maar breekt ook de volledige verkiezing. De cryptografie van het systeem vereist immers dat iedere geregistreerde kiezer een stem uitbrengt, alvorens de resultaten kunnen berekent worden. Ook op het vlak van veiligheid vormt dit een probleem, het betekent dat voor de prijs van de waarborg, iedere participant de verkiezing kan boycotten. Tenslotte zal dit zich ook laten gelden op het vlak van schaalbaarheid en scope. 
	
	De andere kant van de medaille is hier dat het OVN-protocol wel zeer goed scoort op het vlak correctheid, het is moeilijk om het resultaat op frauduleuze wijze te beïnvloeden. De enige manier waarop dit kan gebeuren is bij overname van het Ethereum account van een gebruiker of bij overname van meer dan 50\% van het betreffende Ethereum netwerk (er zijn verschillende Ethereum netwerken). Beide situaties worden door \textcite{McCorry2017} als onwaarschijnlijk geacht, er wordt dus geen rekening mee gehouden. Voor de eerste situatie is het argument van \textcite{McCorry2017} dat het de verantwoordelijkheid van de gebruiker is om zichzelf en zijn Ethereum account (portefeuille) te beschermen, voor de tweede situatie is het argument dat het zeer onwaarschijnlijk is dat een aanvaller er in slaagt om 51\% van de nodes over te nemen en zo de Ethereum blockchain te vervalsen ~\autocite{McCorry2017}.
	
	 Het OVN-protocol en andere blockchain gebaseerde stemsystemen die we niet bespraken zoals BroncoVote voorgesteld door , zijn niet zonder problemen wanneer het op fouttolerantie en veiligheid aan komt. De waarheid gebied echter om te zeggen dat dit eigenlijk voor geen enkel elektronisch systeem het geval is. Zelfs bij elektronische stemmachines die  in veel landen voor verkiezingen worden ingezet is er kans op een veelvoud van problemen, zo blijkt uit het \textcite{Norden2015} onderzoek. Internet gebaseerd stemmen lijkt op het eerste zicht een goede oplossing te zijn, maar in feite vervangt het de bestaande problemen alleen maar door nieuwe problemen met grotere risico's. Veel experten hebben voorlopig nog grote bedenkingen bij online-stemmen ~\autocite{Norden2015}. 
	 
	 Een van de grootste bezorgdheden is dat het online maken van een stemsysteem de veiligheid van verkiezingen kan comprimeren, een online-systeem kan namelijk door om het even welke persoon of instantie worden aangevallen ~\autocite{Norden2015}. Vooral voor verkiezingen op een nationaal of ander electoraal niveau vormt dit een serieus probleem. Recente gebeurtenissen hebben aangetoond dat dergelijke verkiezingen wel degelijk het doelwit van grote georganiseerde instanties of zelfs volledige `vijandige' naties kunnen zijn. Als voorbeeld geven we hier de Amerikaanse presidentsverkiezingen van 2016 waar volgens het \textcite{Mueller2019} onderzoek Russische instanties op verschillende manieren het resultaat zouden hebben beïnvloed.
	 
	 In dit licht lijken blockchain gebaseerde stemsystemen zo slecht nog niet, of toch qua fouttolerantie, veiligheid en correctheid. Toch zijn veel experts (zowel van blockchain technologie als van stemsystemen) hier erg sceptisch over. Het is duidelijk dat er nog veel problemen zijn die tot op heden de ingebruikname van een blockchain gebaseerde stemsystemen bemoeilijken. Bij de huidige systemen blijken er echter ook heel wat problemen te zijn, soms zelfs in grotere mate. Er dient hier ook een distinctie gemaakt te worden tussen blockchain-gebaseerd stemmen en online stemmen. Hoewel veel van de huidige blockchain implementaties - waaronder het OVN-protocol- online implementaties zijn, is er geen verder verband tussen beiden. Implementaties zoals The Blockchain Voting Machine tonen aan de dat blockchain gebaseerd stemmen evengoed offline kan gebeuren, zonder een connectie met het internet. Veel problemen die online blockchain gebaseerde implementaties hebben op het vlak van veiligheid zijn eerder aan te wijten aan het online aspect dan aan het blockchain aspect.https://nos.nl/artikel/2240863-waarom-stemmen-we-in-2030-nog-niet-elektronisch.html
	 
	\subsection{Schaalbaarheid en Scope}
	We bespreken de schaalbaarheid en scope van blockchain gebaseerde stemsystemen. Schaalbaarheid en scope kunnen makkelijk als synoniemen gezien worden, maar in deze context hechten we er verschillende betekenissen aan. Schaalbaarheid definiëren we als de mate waarin een het stemsysteem groter kan gemaakt worden, dat wil zeggen de hoeveelheid kiezers het systeem kan ondersteunen. De scope definiëren we de dan weer als de context en het belang van de verkiezing. Bepaalde verkiezingen zijn immers van een groter belang dan andere. Het verschil tussen schaal en scope is in veel gevallen wel subtiel, vaak is er een verband tussen de grote orde van een verkiezing en het belang ervan.
	
	De blockchain gebaseerde stemsystemen die we tot nu toe bespraken variëren in grote mate op het vlak van schaalbaarheid en scope. Een belangrijk onderscheid is het verschil tussen de systemen uit de praktijk en die uit de literatuur. Praktijk voorbeelden zoals het  \textit{`Active Citizens Project'} van Moskou ondersteunen tot 220.000 participanten terwijl literatuur voorbeelden zoals het  \textit{`Open Vote Network protocol'} en  \textit{`BroncoVote'} slechts een 30-tal participanten ondersteunen. Een mogelijke verklaring voor dit gegeven is dat de voorbeelden uit literatuur simpelweg niet over de zelfde middelen beschikken als bijvoorbeeld het \textit{`Active Citizens Project'}. Een andere, zeer waarschijnlijke verklaring, die hier ook op aansluit is dat het beschikken over een privaat blockchain netwerk, speciaal ontworpen voor verkiezingen voor een groot verschil op het vlak van schaalbaarheid zorgt.
	
	 Zowel \textcite{McCorry2017} als \textcite{Dagher2018} noemen Ethereum's gebrek aan ondersteuning voor cryptografie als één van de primaire hinderpunten voor hun respectievelijke implementaties. Doordat er geen ingebouwde ondersteuning is vanuit \textit{Solidity} moet men gebruik maken van externe bibliotheken. Dit resulteert in code die te omvangrijk is om in één smart contract op te slaan. Vandaar dat   \textcite{McCorry2017} en \textcite{Dagher2018} beiden met een stem-contract en een cryptografie-contract werken. Cryptografische berekeningen zijn bovendien erg duur om uit te voeren in Ethereum, met als gevolg dat de verwerking van stemmen ook bijzonder traag gaat~\autocite{Dagher2018}.
	 
	 Op het vlak van schaalbaarheid lijkt er dus wel potentieel te zijn, maar voorlopig blijft dit niet het geval: de meeste blockchains kennen immers een inherent slechte schaalbaarheid~\autocite{Blenkinsop2018}. Het fundamentele probleem is de grote van ieder blok. Zowel de Bitcoin als de Ethereum blockchain hebben blokken die steeds van dezelfde grote zijn, terwijl het aantal gebruikers en transacties toeneemt. Dit vormt een probleem gezien grote van iedere blok rechtstreeks is verbonden met het aantal transacties dat kan verwerkt worden per tijdseenheid. De Bitcoin blockchain behandelt 3 tot 4 transacties per seconde, de Ethereum blockchain doet er 15/s. Dit betekent een limitatie op het vlak van schaalbaarheid en scope. Stemprotocollen zoals het OVN op Ethereum zijn dus gelimiteerd: een verkiezing met 1 miljoen participanten zou aan de snelheid van 15/s maar liefst een minimum van 19 uur duren. De grote van blokken aanpassen is ook niet zo simpel, het zou niet alleen een gigantische update zijn waar al de gebruikers en stakeholders het mee eens moeten zijn, het zou ook een enorm risico met zich meebrengen:  grotere blokken zouden leiden tot minder nodes en meer centralisatie, wat op zijn beurt weer zou leiden tot een verlaagde veiligheid van de volledige blockchain ~\autocite{Blenkinsop2018}.

	\subsection{Conclusie} 
	In deze sectie bekeken we het potentieel van blockchain gebaseerde stemsystemen. Startend vanuit de huidige situatie, bekeken we de argumenten voor en tegen het klassieke systeem waarbij men de stem uitbrengt op papier. Vervolgens bekeken we ook argumenten voor en tegen elektronische stemsystemen.  
	
	De hoofdargumenten tegen het klassieke systeem:
	\begin{itemize}
		\item\textit{te afhankelijk van mensen}
		\item\textit{te gevoelig aan fouten of manipulatie}
		\item\textit{noch efficiënt noch schaalbaar genoeg}
	\end{itemize}
	
	De hoofdargumenten tegen elektronisch stemmen:
	
	\begin{itemize}
		\item\textit{een verhoogd risico voor externe manipulatie}
		\item\textit{gevoelig aan veroudering van technologie}
		\item\textit{kostelijk om te onderhouden}
	\end{itemize}
	
	Hier bleek dat zowel de papieren stem als op de elektronische stem eigenlijk maar suboptimaal zijn. De tekortkomingen van deze huidige systemen vormen de noodzaak tot een nieuwe oplossing. Blockchain stemsystemen werden door verschillende proponenten voorgesteld als een alternatief op het huidige elektronische systeem dat de slechte aspecten van zowel papier als elektronisch stemmen achterwege laat en de goede eigenschappen van beiden combineert en daarbij ook nog eens eerlijker, transparanter en robuuster is.
	
	Het Open Vote Network, zoals voorgesteld door \textcite{McCorry2017} werd gegeven als voorbeeld uit de literatuur. Het werk van \textcite{McCorry2017} toont aan dat blockchain gebaseerde stemsystemen wel degelijk kunnen geïmplementeerd worden op een wijze waar de veiligheid, betrouwbaarheid en privacy ongeëvenaard zijn door hedendaagse systemen.
	
	Het werd echter ook snel duidelijk dat blockchain als technologie simpelweg nog niet klaar is voor grootschalige verkiezingsimplementaties. Blockchain gebaseerde verkiezingen op een nationaal niveau blijven daarom voorlopig moeilijk te realiseren.  Enkele van de voornaamste redenen daarvoor kwamen in deze sectie aan bod:
	\begin{itemize}
		\item\textit{Weinig ondersteuning voor ontwikkeling op bestaande blockchains}
		\item\textit{Het opzetten van een eigen blockchain is bijzonder kostelijk}
		\item\textit{Een inherent schaalbaarheidsprobleem in het ontwerp van bestaande blockchains}
	\end{itemize}
	In de praktijk blijven blockchain gebaseerde stemsystemen bijzonder zeldzaam. Van de meeste  implementaties die men kan vinden zijn online diensten, aangeboden als service en enkel gericht op het organiseren van kleinschalige verkiezingen. Enkele projecten, waaronder het succesvolle Moscow Citizen's initiative, tonen wel aan dat - mits er  genoeg middelen worden ingezet om de tekortkomingen van de technologie te overwinnen - er wel degelijk potentieel is, ook op het vlak van schaalbaarheid.
	
	Het succes van projecten zoals Moscow Citizen's Initiative kan echter niet direct vertaald worden naar een generiek systeem voor electorale verkiezingen. De scope van dergelijke projecten is namelijk veel kleiner. Hoewel het project met een 300.000 kiezers qua schaal al dichter in de buurt komt van electorale verkiezingen, zijn de onderwerpen waarover wordt gestemd nog steeds een stuk minder belangrijk dan pakweg het verkiezen van een volksvertegenwoordiger. Bijgevolg is er dus ook minder kans op externe aanvallers. Bovendien is het project ook opgebouwd als een online stemplatform, wat volgens de meeste experten een bijzonder slecht idee is in de context van nationale verkiezingen.
	
	Een ideaal stemsysteem bestaat tot op heden nog niet. Er zijn heel wat verbeteringen mogelijk aan de stemsystemen van vandaag: zowel het stemmen op papier als de elektronisch variant blijken bijlange niet zo veilig als men zou verwachten. Blockchain technologie werd in dit hoofdstuk besproken als een potentiële oplossing voor de problemen die de huidige systemen teisteren. Het Open Vote Netwerk van \textcite{McCorry2017} werd als voorbeeld gegeven van een zo'n stemsysteem. Het OVN illustreert perfect de sterkte en zwaktepunten van het gemiddelde blockchain-stemsysteem. Blockchain is geen wondermiddel dat de bestaande problemen in een slag kan oplossen. Het is een technologie met  voor- en nadelen, die bovendien nog in volle ontwikkeling is. Of het in de toekomst de implementatie van perfecte stemsystemen zal toelaten valt moeilijk te zeggen, maar het potentieel is er alvast wel. 
	
	

	
	
	
		
		
		
		
		
		
	
%%=============================================================================
%% Methodologie
%%=============================================================================

\chapter{Methodologie}
\label{ch:methodologie}

%% TODO: Hoe ben je te werk gegaan? Verdeel je onderzoek in grote fasen, en
%% licht in elke fase toe welke stappen je gevolgd hebt. Verantwoord waarom je
%% op deze manier te werk gegaan bent. Je moet kunnen aantonen dat je de best
%% mogelijke manier toegepast hebt om een antwoord te vinden op de
%% onderzoeksvraag.
Na het voeren van een brede literatuurstudie in hoofstuk \ref{ch:stand-van-zaken}, die  alle contextuele informatie verschaft betreffende het onderwerp \textit{blockchain-gebaseerde stemsystemen}, zal deze scriptie zich nu toespitsen op de implementatie van een dergelijk systeem. Deze implementatie en de beschrijving ervan vormen de  \textit{proof-of-concept} van deze scriptie.

 Dit hoofdstuk beschrijft de werkwijzen, procedures en methoden die gevolgd werden om de  proof-of-concept van deze scriptie te realiseren. De proof-of-concept zelf vindt men terug in de vorm van een praktische handleiding in hoofdstuk \ref{ch:handleiding}, alsook in de vorm van een project waarvan de broncode kan gevonden worden op Github\footnote{Zie https://github.com/Ocean97Li/bachelorproef}.
 
 Gezien het doel van de handleiding is om een volledig beeld te geven van de implementatie, door bijvoorbeeld het belichten van de verschillende technische praktische moeilijkheden, zijn veel van de zaken betreffende de werkwijze opgenomen in het volgende hoofdstuk. Het doel van dit hoofdstuk is voornamelijk om het proces te duiden, dat tot de realisatie en formulering van de handleiding heeft geleid. Tot dat doeleinde wordt de methodologie opgedeeld in drie verschillende delen:

Deel 1 beschrijft en vergelijkt in het kort de verschillende implementaties die als bron dienden voor het project van deze scriptie.

Deel 2 beschrijft de werkwijze en de ondervonden problemen tijdens het implementeren van het back-end project.

Deel 3 beschrijft in het kort de werkwijze en problemen bij de implementatie van het front-end project.

\section{Bronnen}
	\subsection{Boeken}
	Bij gebrek aan ontwikkelervaring binnen Ethereum, begon het realiseren van de proof-of-concept met het leren van de programmeertaal \textit{Solidity}. Vermits dit een taal is die syntactisch erg gelijkaardig is aan het bekende JavaScript, verliep dit proces vrij vlot. Het werk van \textcite{Antonopoulos2018} vormde een stevige basis voor het leren van de taal.
	
	Met de kennis van Solidity begon het experimenteren met een stemsysteem op basis van een smart-contract. Het werk van \textcite{Mukhopadhyay2018} vormde daarvoor de eerste bron van inspiratie, maar werd geleidelijk aan vervangen door andere werken.
	
	\subsection{DappUniversity en Walking Tree Technologies}
	Voor de uiteindelijke proof-of-concept vormt de DappUniversity-implementatie van \textcite{McCubin2019}  de voornaamste bron van inspiratie. De hoofdreden dat \textcite{McCubin2019} binnen deze scriptie de voorkeur geniet, is omdat dit veruit het meest recente werk is binnen de literatuur. De code is geschreven in de laatste stabiele versie van Solidity\footnote{Solidity v0.5.8 in Juni 2019}. In een nog niet zo mature ontwikkelomgeving als Ethereum is dat van cruciaal belang. Het up-to-date blijven van de code is erg belangrijk. Vermits het netwerk en de taal geregeld nog drastische veranderingen ondergaan, kan er een groot potentieel aan performantie verloren gaan bij gebruik van verouderde versies. Daarnaast vormt \textcite{McCubin2019} ook het perfecte startpunt met een relatief eenvoudig, maar vooral ook compact smart-contract. Complementair aan het werk van \textcite{McCubin2019}, wordt de Walking Tree Technologies-implementatie van \textcite{Ranjan2018} gebruikt om de connectie te maken tussen Ethereum en een front-end Angular project.
	
	\subsection{Open Vote Network Protocol}
	Het OVNP van \textcite{McCorry2017} werd reeds uitvoerig behandeld in sectie \ref{sec:OVNP} van het vorige hoofdstuk. Met dit protocol laat \textcite{McCorry2017} zich gelden als een van de meest secure werken in de literatuur. Het kent echter ook verschillende nadelen. Zo is er bijvoorbeeld in het streven naar absolute veiligheid, een stuk gebruiksvriendelijkheid dat verloren gaat. Een ander nadeel is dat de implementatie het Ethereum-platform tot zijn absolute maximum duwt. Ethereum is maar net in staat om het OVNP te draaien, er zijn maar liefst twee contracten voor nodig, waarin er gebruik  gemaakt wordt van complexe externe cryptografie bibliotheken bij gebrek aan native support. Deze dependencies zorgen ervoor dat het OVNP gevoelig is voor Ethereum-updates. Toch ziet deze scriptie het werk van \textcite{McCorry2017} als de gouden standaard voor veilige, volledig anonieme blockchain-gebaseerde stemsystemen. In de praktijk blijkt het echter moeilijk om het OVNP zelf aan te wenden in een implementatie...
	
\section{Back-end project}
	\subsection{Verloop van opzet en implementatie}
	De opzet en initiële implementatie van het \textit{Election.sol} smart-contract verliepen grotendeels zoals voorgeschreven door \textcite{McCubin2019}. Dit alles verliep naadloos. Toen de functionaliteit moest worden uitgebreid met beveiligingsmechanismen kwamen er echter enkele problemen aan het licht.
	
	\subsection{Ondervonden Problemen}
	Een van de moeilijkheden waarmee men te maken krijgt tijdens het ontwikkelen van gedecentraliseerde applicaties binnen Ethereum, is het feit dat het zeer moeilijk is om anonimiteit of privacy voor gebruikers te creëren. De aard van de blockchain-technologie zelf legt immers nadruk op het publiek beschikbaar maken van alle gegevens. Ook in Ethereum is deze filosofie sterk aanwezig, iedere transactie en alle daarbij horende parameters zijn publiek. Eigenlijk is alles openbaar binnen de Ethereum-blockchain. Private attributen en methodes in smart-contracts mogen dan wel bestaan, de waarden zullen steeds publiek leesbaar zijn. \autocite{Buterin2014}
	
	Voor deze scriptie vormde dit een potentieel probleem. Gezien iedere stem als parameter doorgegeven wordt, betekent dit dat het stemgedrag van ieder adres zichtbaar is. Op het vlak van stemsystemen is anonimiteit van kiezers  echter vaak noodzakelijk. Gezien Ethereum geen private transacties ondersteund (sommige blockchains doen dit wel),  moest er simpelweg op zoek gegaan worden naar alternatieve methodes. 
	
	Het leek een logische volgende stap om te onderzoeken of de stemmen die werd doorgeven als parameters niet simpelweg kon worden verborgen door ze te encrypteren. Het probleem hiermee was dat ook standaard encryptie- en decryptietechnieken niet werken op de Ethereum-blockchain: men kan geen enkel geheim bewaren op of doorsturen naar Ethereum. Stemmen geëncrypteerd verzenden en vervolgens op het einde van de verkiezing alle stemmen bekend maken door gebruikers hun decryptiesleutel te laten broadcasten (het veelgebruikte \textit{commit-reveal pattern}) werkt niet in deze specifieke context. Er is immers tot op heden\footnote{Off-chain transacties zouden dit misschien wel mogelijk maken, er is een kans dat deze zullen toegevoegd worden volgens de Ethereum Wiki \autocite{Buterin2014}} geen enkele manier om de decryptiesleutel geheim te houden. Zowel het onthouden van decryptiesleutels  in een smart-contract als het versturen van de sleutel via een transactie resulteert dus in hetzelfde: de sleutel is zichtbaar op de blockchain en kan gebruikt worden om op directe wijze het stemgedrag van de kiezer in kwestie af te leiden.
	
	Indien men ook maar enige vorm van gegarandeerde anonimiteit wil aanbieden, kan men niet anders dan gebruik maken van zeer geavanceerde cryptografische technieken. In dit geval grijpen we hier dus naar de cryptografische techniek die we reeds hebben besproken in het vorige hoofdstuk: het OVNP van \textcite{McCorry2017}. In de praktijk bleek dit echter moeilijker dan gedacht. 
	
	 Het OVNP van \textcite{McCorry2017} mag dan wel volledig opensource als bibliotheek worden aangeboden, het probleem is dat de codebasis niet meer is geüpgedated sinds augustus 2017\footnote{Zie https://github.com/stonecoldpat/anonymousvoting/commits/masters}. Sinds die tijd is de werking van Solidity en het Ethereum-netwerk zodanig veranderd, dat de code in deze vorm noch compatibel met noch performant ten opzichte van huidige Ethereum-implementaties is. 
	 
	 In functie van dit onderzoek werd een poging ondernomen om de implementatie te herschrijven, dit bleek echter algauw buiten de scope van deze scriptie te vallen. Bijgevolg voldoet de door deze scriptie gepresenteerde oplossing niet aan wat \textcite{McCorry2017} en \textcite{Kiayias2002} \textit{perfect ballot secrecy} noemen.

\section{Front-end project}
	\subsection{Verloop van opzet en implementatie}
	Het verloop van de front-end implementatie gebeurde eerst grotendeels op basis van het werk van \textcite{Ranjan2018} en
	in mindere mate ook van \textcite{McCubin2019}. Deze werken werden alleen gebruikt bij het implementeren van de \textit{EthereumConnectionService.ts}. 
	
	Voor de rest van de applicatie werd met de vrije hand een ontwerp ontwikkeld, dat zich enigszins liet inspireren door de visuele stijl van de OVNP-implementatie van \textcite{McCorry2017}.
	\subsection{Ondervonden Problemen}
	De enige problemen die opkwamen bij het front-end project waren gerelateerd aan de \textit{web3 node\_module}. 
	
	Verschillende door web3 gebruikte node packages konden niet worden gevonden. Na enig onderzoek werd duidelijk dat het probleem in kwestie een incompatibiliteit betrof tussen de huidige web3 (1.0.0-beta) en de nieuwere Angular versies (6+). De  eerste oplossing bestond erin de web3-versie te verlagen naar een meer stabiele versie. In dit geval was dat versie v0.20.5. 
	
	Bij de hereninstallatie van de node\_modules van het project deed het probleem zich echter opnieuw voor. Daarom werd de voorgestelde oplossing\footnote{Zie https://gist.github.com/niespodd/1fa82da6f8c901d1c33d2fcbb762947d} gebruikt om een zogenaamd custom patch file te schrijven. Met de toevoeging van deze patch file (zie patch.js) en herinstallatie van alle node modules, kunnen ontbrekende packages gesubstitueerd worden door de pakketten op de werkelijke locatie.
	
	

% Voeg hier je eigen hoofdstukken toe die de ``corpus'' van je bachelorproef
% vormen. De structuur en titels hangen af van je eigen onderzoek. Je kan bv.
% elke fase in je onderzoek in een apart hoofdstuk bespreken.


%%=============================================================================
%% Handleiding
%%=============================================================================

\chapter{Handleiding}
\label{ch:handleiding}

Dit hoofdstuk beschrijft hoe men het best te werk kan gaan voor het opzetten van een \textit{eenvoudig, kleinschalig blockchain-gebaseerd stemsysteem}. Het doel is om een praktische en complete handleiding te vormen, hoofdzakelijk bedoeld voor ontwikkelaars die niet vertrouwd zijn met het ontwikkelen van Ethereum DApps, noch met het ontwikkelen van cryptografische stemsystemen. 

Gezien de vrij technische aard van deze handleiding, wordt er verondersteld dat de lezer op zijn minst een algemene voorkennis van programmeren heeft en enigszins vertrouwd is met \textit{Javascript}. Kennis van de taal \textit{Solidity} wordt niet verondersteld. Tenslotte wordt verondersteld dat de lezer over een basiskennis Engels beschikt, daar de volledige codebasis in die taal geschreven is. 

De  verschillende implementatiekeuzes die hier worden gemaakt, gebeuren op basis van verschillende concepten en technieken die werden toegelicht in hoofdstuk \ref{ch:stand-van-zaken}. Zo wordt er bijvoorbeeld gekozen voor het Ethereum-platform als ontwikkelomgeving op basis van sectie \ref{sec:ethereum-en-smart-contracts} en wordt de veiligheid vergeleken met die van het `state-of-the-art' zelftellende OVNP \autocite{McCorry2017} uit sectie \ref{sec:OVNP}.

Dit hoofdstuk vormt de \textit{proof-of-concept} van deze scriptie, we trachten hier antwoord te bieden op een deel van de onderzoeksvragen van deze scriptie, met name: 
\begin{itemize}
	\item \textit{Welke tools heeft men nodig om een blockchain-gebaseerd stemsysteem op te zetten en wat zijn de voor- en nadelen hiervan?}
	\item \textit{Is een blockchain-gebaseerd stemsysteem opzetten haalbaar in de praktijk?} 
\end{itemize}

\section{Benodigdheden}
\label{sec:benodigdheden}
	De volgende zaken dienen geïnstalleerd te worden voor men van start kan gaan met het implementeren van een gedecentraliseerde Ethereum-blockchain applicatie:
	\begin{itemize}
		\item{Node.js}
		\item{npm}
		\item{Truffle}
		\item{Ganache}
		\item{Metamask}
	\end{itemize}
	Daarnaast heeft men ook nodig:
	\begin{itemize}
		\item{Een IDE code-editor naar keuze met syntax-ondersteuning voor Solidity}
		\item{Google Chrome}
	\end{itemize}
	\subsection{Node en npm}
	Node.js is een populaire runtime-omgeving waarmee Javascript op ieder platform uitgevoerd kan worden, zonder dat daarvoor een browser nodig is. Npm is een pakketbeheerder voor Javascript-code. Npm zit standaard in Node.js en is `s werelds grootste softwareregister. Open source-ontwikkelaars wereldwijd gebruiken het om pakketten te delen, veel organisaties gebruiken npm om ook hun privéontwikkeling te beheren (\ref{fig:nodejs}).\footnote{Verkregen en vertaald van https://docs.npmjs.com/about-npm/}
	
	Eenmaal node\footnote{node met npm is verkrijgbaar via https://nodejs.org} en npm\footnote{npm is ook apart verkrijgbaar via https://www.npmjs.com/get-npm} geïnstalleerd zijn, verifieert men de installatie via het console-commando: 
	\lstset{language=bash}
	\begin{lstlisting}[numbers=none]
	> node -v
	\end{lstlisting}Bij correcte installatie geeft dit de huidige node-versie terug, bijvoorbeeld \textit{v10.15.1}. 
	
	\begin{figure}
		\includegraphics[width=\linewidth/2]{img/nodejs.png}
		\includegraphics[width=\linewidth/2]{img/npm.png}
		\caption{De Node.js en npm logo's}
		\label{fig:nodejs}
	\end{figure}
	
	\subsection{Truffle}
	Truffle is een ontwikkelomgeving, testframework en asset pipline, gericht op het versoepelen van het Ethereum-ontwikkelproces. Het bevat ook verschillende code-templates die als basis kunnen worden gebruikt om gedecentraliseerde applicaties te ontwikkelen.
	
	Truffle kan eenvoudig geïnstalleerd worden via npm (mits dit voorgeïnstalleerd is) met het console-commando:
	 \lstset{language=bash}
	\begin{lstlisting}[numbers=none]
	> npm install truffle
	\end{lstlisting}
	\subsection{Ganache}
	Ganache\footnote{Ganache is verkrijgbaar via https://www.trufflesuite.com/ganache} is een applicatie die ontwikkelaars in staat stelt om een private Ethereum-blockchain op te zetten. We gebruiken deze blockchain gedurende de volledige ontwikkelperiode. Op deze manier kan men kosteloos smart contracts ontwikkelen en testen, als men direct op de Ethereum-hoofdketen ontwikkelt, is er immers aan iedere transactie een kost verbonden. 
	
	Ganache biedt ons  niet enkel een lokale blockchain, maar ook 10 Ethereum accounts om te gebruiken tijdens het ontwikkelen. Deze accounts hebben adressen die corresponderen met adressen op de lokale Ethereum-blockchain. Elke account is vooraf geladen met 100 nep-ether.
	
	Figuur \ref{fig:ganache-blockchain}: Eenmaal Ganache geïnstalleerd is, open we de applicatie. Door het kiezen van de optie \textit{quickstart} zetten we onmiddellijk een lokale blockchain op. 
	
	\begin{figure}
		\includegraphics[width=\linewidth]{img/ganache-blockchain.png}
		\caption{Een Lokale blockchain opgezet via Ganache}
		\label{fig:ganache-blockchain}
	\end{figure}
	
	
	\subsection{Metamask}
	Metamask\footnote{Metamask is verkrijgbaar via https://metamask.io of via https://chrome.google.com/webstore} is een Ethereum wallet-plugin voor Google Chrome, die gebruikers in staat stelt om transacties van Ethereum DApps uit te voeren in de browser, zonder zelf een node in het netwerk te moeten zijn. Door Metamask te installeren hoeft men met andere woorden de volledige blockchain dus niet meer te downloaden. 
	
	Voor deze implementatie is Metamask van cruciaal belang, gezien het verantwoordelijk is voor de verificatie van Ethereum-accounts. Daarnaast is het ook een zeer handige tool wanneer we onze implementatie in de browser willen testen, het staat ons toe zeer snel tussen verschillende accounts te schakelen.
	
	Eenmaal de plugin geïnstalleerd is, dient men een paswoord te creëren, daarnaast ontvangt men ook een fallback-sleutel.
	
	\begin{figure}
		\includegraphics[width=\linewidth]{img/metamask-truffle-ganache.png}
		\caption{Truffle, Metamask en Ganache logo's}
		\label{fig:metamask-truffle-ganache}
	\end{figure}
	\newpage
\section{Implementatie smart contracts}
	Eenmaal alle verschillende tools en plugins uit sectie \ref{sec:benodigdheden} geïnstalleerd zijn, kan men aan het ontwikkelen van smart contracts beginnen. In de volgende subsecties bespreken we de implementatie van een blockchain-gebaseerd stemsysteem op basis van smart contracts. Het volledige project kan worden gevonden op Github\footnote{Zie: https://github.com/Ocean97Li/bachelorproef/tree/master/poc/EthereumVote/backend}. We beginnen met een eenvoudige implementatie die niet \textit{self-tallying} is. In een latere sectie bespreken we wat er nodig is om een veiliger systeem te bekomen.
	\subsection{Opzet Truffle}
	Voor we beginnen met het ontwikkelen van onze smart contracts, moeten we eerst een ontwikkelomgeving opzetten. Voor een vlotte start maken we gebruik van één van de template-projecten beschikbaar via Truffle. 
	
	Na het navigeren naar een gewenste directory gebruikt men het console-commando: 
	 \lstset{language=bash}
	\begin{lstlisting}[numbers=none]
	> truffle unbox pet-shop
	\end{lstlisting}
	
	Dit creëert een nieuw project in de huidige directory. We openen dit in een code-editor (bijvoorbeeld VS-code). Figuur \ref{fig:truffle-template} toont de structuur van de gegenereerde Truffle-template. Het project bevat momenteel zowel een back-end gedeelte (smart contract) als een front-end (javascript project). 
	
	In deze handleiding zullen we - met het principe van herbruikbaarheid in gedachten - de code opsplitsen in afzonderlijke front- en back-end projecten. De folder en inhoud onder \textit{\slash src}, alsook de afbeeldingen \textit{box-img-lg.png} en \textit{box-img-sm.png} mogen dus uit dit project verwijderd worden.
	
	Het project kan nu als start-template worden gebruikt voor onze implementatie.
	
	\begin{figure}
		\centering
		\includegraphics[width=\linewidth/2]{img/truffle-template.png}
		\caption{De structuur van een Ethereum Truffle project}
		\label{fig:truffle-template}
	\end{figure}
	
	\subsection{Basisimplementatie}
	We beginnen met het aanmaken van een nieuw smart contract bestand binnen het template-project. Dit doen we op de locatie \textit{contracts\slash Election.sol}. We overwegen de vereisten voor onze implementatie. Om een simpel stemsysteem te implementeren in een smart contract hebben we nodig:
	\begin{itemize}
		\item een lijst van de `opties' waarop gestemd kan worden
		\item een lijst van de accounts die een stem uitgebracht hebben
		\item per optie het aantal stemmen
	\end{itemize}
	Uiteraard zijn er verschillende manieren waarop we dit kunnen aanpakken. In de context van smart contracts is het echter cruciaal dat we zo weinig mogelijk code schrijven. We baseren ons daarom op de implementatie die wordt gegeven door \textcite{McCubin2019}\footnote{Zie ook https://github.com/dappuniversity/election}.
	
	We beginnen met het declareren van de Solidity-versie:

	\lstset{language=JavaScriptSolidity} 
	\begin{lstlisting}[numbers=none]
	pragma solidity ^0.5.8;
	\end{lstlisting}
	\lstset{language=JavaScriptSolidity} 

	Vervolgens starten we met het definiëren van ons smart contract \textit{Election} :
	\begin{lstlisting}[numbers=none]	
	contract Election {
		constructor () public {
		}
	}
	\end{lstlisting}
	
	Omdat we een lijst van de `opties' willen hebben en ook willen bijhouden per optie hoeveel stemmen er zijn, maken we gebruik van een custom-type dat we \textit{Candidate} noemen. 

	Onze verkiezing hoeft niet per se om het verkiezen van een persoon te gaan, het is echter wel handig om over de opties te denken in termen van`kandidaten'. 
	Iedere optie is een kandidaat die een id, een `naam' (de optie tekst) en een aantal voorkeursstemmen heeft. 
	
	In Solidity definiëren we zo'n custom-type in de vorm van een \textit{struct}:
	
	\begin{lstlisting}[numbers=none]	
	struct Candidate {
		uint id;
		string name;
		uint votes;
	}
	\end{lstlisting}
	
	We breiden \textit{Election} nu ook uit met de volgende attributen:
	
	\begin{lstlisting}[numbers=none]
	//Fetch the candidates	
	mapping(uint => Candidate) public candidates;
	// Store accounts that have voted
	mapping(address => bool) private voters;
	// Read candidate
	uint public candidatesCounter;
	\end{lstlisting}
	
	Het \textit{mapping} keyword in Solidity duidt een hashTable aan, in Solidity is dit de aangeraden verzameling-structuur\footnote{Zie https://ethereum.stackexchange.com/questions/2592/store-data-in-mapping-vs-array\#answer-2597}.  Mappings staan ons toe om key-value searching te doen. In het geval van  \textit{candiates} mappen we de `kandidaten' op basis van hun id's, deze zijn numeriek en incrementeel. We houden het aantal kandidaten bij in \textit{candidatesCounter}, zodat we niet onnodig in de mapping moeten zoeken, maar exact weten wat de range van id's is.
	
	Bij het attribuut \textit{voters} mappen we de adressen van alle kiezers op een booleaanse waarde. Wensen we te weten of een kiezer reeds een stem heeft uitgebracht, dan kunnen we dit via het \textit{voters} attribuut eenvoudig verifiëren. 
	
	We voegen een methode toe aan \textit{Election} die ons in staat stelt om \textit{candiates} te initialiseren:
	
	\begin{lstlisting}[numbers=none]
	function addCadidate(string memory _name) private {
		candidatesCounter++;
		candidates[candidatesCounter] = Candidate(candidatesCounter,_name,0);
	}
	\end{lstlisting}
	Merk op dat de parameter in de bovenstaande functie gemarkeerd is met het Solidity \textit{memory} keyword. Dit duidt aan dat deze parameter niet in de blockchain dient opgeslagen te worden. De tegenhanger van dit keyword is \textit{storage}.
	
	Voorlopig zullen we onze opties `hardcoden' in de constructor-functie, we kiezen voor een implementatie met eenvoudige binaire ja/neen-vragen:
	
	\begin{lstlisting}[numbers=none]
	constructor () public {
		addCadidate("Yes");
		addCadidate("No");
	}
	\end{lstlisting}
	
	Nu we opties hebben toegevoegd waarop gestemd kan worden, rest ons enkel nog het implementeren van een stemfunctie. 
	
	Gebruikers mogen niet meermaals een stem uitbrengen, daarom controleren we iedere gebruiker die probeert te stemmen. 
	
	\begin{lstlisting}[numbers=none]
	function hasVoted() public view returns (bool ok) {
		return voters[msg.sender];
	}
	\end{lstlisting}
	
	Merk ook op dat hier gebruik wordt gemaakt van het Solidity \textit{view} keyword, dit duidt aan dat de betreffende functie geen aanpassingen zal maken aan het contract. 
	
	\begin{lstlisting}[numbers=none]
	function vote(uint _candidateId) public {
		if(!hasVoted()) {
			// Record voter has voted
			voters[msg.sender] = true;
			// Update candidate vote count
			candidates[_candidateId].votes++;
		}
	}
	\end{lstlisting}
	
	We vinden de kandidaat waarvoor de gebruiker stemde op basis van de parameter\textit{\_candidateId}. Door deze waarde in te vullen in de mapping \textit{candidates}, krijgen we toegang tot de correcte Candidate struct. We verhogen het \textit{votes} attribuut van deze struct met 1. Op dit punt is de stem uitgebracht, we hebben nu een eenvoudig stemsysteem bekomen!
	
	De volledige code voor het smart contract \textit{Election.sol} is momenteel:
	
	\lstset{language=JavaScriptSolidity} 
	\begin{lstlisting}[frame=single,  label={lst:election}] 
	pragma solidity ^0.5.8;
	
	contract Election {
		// Store candidate
		struct Candidate {
			uint id;
			string name;
			uint votes;
		}
		//Fetch the candidates
		mapping(uint => Candidate) public candidates;
		// Store accounts that have voted
		mapping(address => bool) private voters;
		// Read candidate
		uint public candidatesCounter;
		
		// Constructor
		constructor () public {
			addCadidate("Yes");
			addCadidate("No");
		}
	
		function addCadidate(string memory _name) private {
			candidatesCounter++;
			candidates[candidatesCounter] = 
			Candidate(candidatesCounter,_name,0);
		}
		
		function hasVoted() public view returns (bool ok) {
			return voters[msg.sender];
		}
		
		function vote(uint _candidateId) public {
			if(!hasVoted()) {
				// Record voter has voted
				voters[msg.sender] = true;
				// Update candidate vote count
				candidates[_candidateId].votes++;
			}
		} // End function
	} // End contract
	\end{lstlisting}
	
	
	\subsection{Migration toevoegen}
	Nu we een functioneel smart contract hebben, wensen we dit op onze lokale blockchain te deployen. Hiertoe dienen we eerst een nieuw migratiebestand aan het project toe te voegen op de locatie \textit{migrations\slash 2\_deploy\_contracts.js}.  
	
	Het nieuwe bestand krijgt de volgende inhoud:
	
	\begin{lstlisting}[frame=single]
	// Find smart contract
	var Election = artifacts.require("./Election.sol");
	
	module.exports = function(deployer) {
	// List all the smart contracts to be deployed
		deployer.deploy(Election);
	};
	\end{lstlisting}
	
	De bovenstaande code zal ervoor zorgen dat Truffle het `Election' smart contract op de blockchain kan plaatsen wanneer we het project deployen.
	
	\subsection{Truffle-project linken aan Ganache}
	
	Om een beter overzicht te krijgen van de `state' van de lokale blockchain, kunnen we ons Truffle-project in Ganache toevoegen. Dit doen we door binnen Ganache naar de menu-optie `Contracts' te navigeren, vervolgens voor de optie `link truffle projects'  te kiezen, naar het project te navigeren en daar het bestand \textit{truffle.js} aan te duiden. Tenslotte  kiezen we voor `save and restart'. Als we nu naar de menu-optie `Contracts' gaan, krijgen we een overzicht van de contracten en hun status (\ref{fig:contracts-ganache1}).
	
	\begin{figure}
		\includegraphics[width=\linewidth]{img/contracts-ganache1.png}
		\caption{De status van smart contracts weergegeven in Ganache}
		\label{fig:contracts-ganache1}
	\end{figure}
	
	
	\subsection{Deployen van smart contracts}
	
	Eenmaal we een migratiebestand voor onze smart contracts hebben, kunnen we deze deployen. Eenmaal de Ganache-blockchain opgestart is en het truffle-project eraan gelinkt, navigeren we naar het project in de console om vervolgens het volgende commando te gebruiken:
	
	\begin{lstlisting}[numbers=none]
	> truffle migrate
	\end{lstlisting}
	
	Merk op dat, indien we nu opnieuw wensen te deployen (in dit geval niet erg omdat we op een lokale blockchain werken), we het volgende commando dienen te gebruiken:
	
	\begin{lstlisting}[numbers=none]
	> truffle migrate --reset
	\end{lstlisting}
	
	Wanneer er geen compilatiefouten aanwezig zijn in de code van de smart contracts, kan er met succes gedeployed worden. In dat geval krijgt men voor elk van de smart contracts een \textit{transaction receipt}:
	\lstset{language=bash}
	\begin{lstlisting}[numbers=none]
	 Deploying `Election'
	--------------------
	> transaction hash:    0xaf22d8d9c9c9a1230e1764d7a3bd9249a3...eb8fa
	> Blocks: 0            Seconds: 0
	> contract address:    0xBF1917F9c9cFee73A4E653de5ad62a6515b78Ed4
	> block number:        3
	> block timestamp:     1560958652
	> account:             0xE03c3692FED9D4f2cBc7c5a30b05Ae9ce7b3b839
	> balance:             99.98553468
	> gas used:            419914
	> gas price:           20 gwei
	> value sent:          0 ETH
	> total cost:          0.00839828 ETH
	
	
	> Saving migration to chain.
	> Saving artifacts
	-------------------------------------
	> Total cost:          0.00839828 ETH
	\end{lstlisting}
	
	Als we de contracts nu in Ganache bekijken, zien we dat de status ook daar veranderd is naar `deployed'. (Zie Figuur \ref{fig:contracts-ganache2})
	
	\begin{figure}
		\includegraphics[width=\linewidth]{img/contracts-ganache2.png}
		\caption{Smart contracts `deployed' weergegeven in Ganache}
		\label{fig:contracts-ganache2}
	\end{figure}
	
	\subsection{Belang van testen}
	Bij het ontwikkelen van DApps speelt testen een cruciale rol. Eenmaal gedeployed naar een officieel netwerk, kunnen functies die bugs bevatten en onverwacht gedrag vertonen erg kostelijk zijn voor gebruikers. Het is daarom best practice om enkel smart contracts te deployen die functioneel-volledig en getest zijn.  Doordat alles wat in de blockchain wordt bewaard \textit{immutable} is, betekent het herdeployen van een smart contract eigenlijk dat het huidige contract wordt vervangen door een nieuwe kopie. Zowel de state als het adres van het oude contract gaan verloren. Dit is brekend voor iedere front-end applicatie die aan de DApp verbonden is. 
	
	Door smart contracts grondig te testen kunnen we dergelijke situaties vermijden.
	
	Er zijn verschillende methodes die men kan toepassen om een smart contract te testen, de standaard is om ze te schrijven in Solidity. In dit voorbeeld zullen we echter gebruik maken van Javascript. Via Truffle kunnen we met behulp van Javascript de interacties van gebruikers met onze smart contracts gemakkelijk simuleren. Truffle bevat immers standaard Mocha\footnote{Apart verkrijgbaar via https://mochajs.org} (testframework) en Chai Assertion Library\footnote{Apart verkrijgbaar via https://www.chaijs.com}. Deze twee tools stellen ons in staat om smart contracts te importeren binnenin Javascript-testen. 
	
	\subsection{Smoke test}
	We voegen een nieuw bestand toe op de locatie \textit{test\slash election.js}. We schrijven een smoke test, een test die nagaat of het smart contract op correcte wijze geïnitialiseerd wordt.

	 \lstset{language=JavaScriptSolidity} 
	 \begin{lstlisting}
	 var Election = artifacts.require("./Election.sol");
	 	
	 contract("Election", function(accounts){
		var electionInstance;
		 
		it("Initializes two candidates", function() {
		 	return Election.deployed().then(function(instance){
		 		return instance.candidatesCounter();
		 	}).then(function(count){
		 		assert.equal(count,2);
		 	});
	 	});
	 	
		it("Initializes yes and no", function() {
	 		return Election.deployed().then(function(instance){
	 			electionInstance = instance;
	 			return electionInstance.candidates(1);
	 		}).then(function(candidate){
				assert.equal(
			 	candidate[0],1,"has the correct id: 1"
			 	);
				assert.equal(
				 	candidate[1],"Yes","has the correct value: `Yes'"
				);
				assert.equal(
				 	candidate[2],0,"has the correct amount of votes: 0"
				);
				return electionInstance.candidates(2);
	 		}).then(function(candidate){
				assert.equal(
					candidate[0],2,"has the correct id: 2"
				);
				assert.equal(
					candidate[1],"No","has the correct value: `No'"
				);
				assert.equal(
					candidate[2],0,"has the correct amount of votes: 0"
				);				
	 		}); // End function
	 	}); // End it()
	}); // End contract
	\end{lstlisting}
	
	Concreet testen we hier dat na instantiatie:
	\begin{itemize}
		\item Het attribuut \textit{candidatesCounter} = 2 is.
		\item Het attribuut \textit{candidates[0]} een `kandidaat' is met id = 1, naam = ``Yes'' en aantal stemmen = 0.
		\item Het attribuut \textit{candidates[1]}  een `kandidaat' is met id = 2, naam = ``No'' en aantal stemmen = 0.
	\end{itemize}
	\subsection{Testen uitvoeren}
	Om geschreven testen uit te voeren, navigeren we naar de project-directory en gebruiken we het console-commando:
	
	\begin{lstlisting}[numbers=none]
	> truffle test
	\end{lstlisting}
	
	Truffle voert hierop alle testen binnen de directory \textit{test\slash} uit:
	
	\begin{lstlisting}[numbers=none]
	> Artifacts written to /var/folders/xq/mnnky8qn6c58pt33xhpz2bhw0000gn/T/test-119519-4744-y0zb1c.1il0l
	> Compiled successfully using:
	- solc: 0.5.8+commit.23d335f2.Emscripten.clang
	
	
	
	Contract: Election
	v Initializes two candidates
	v Initializes yes and no (93ms)
	
	
	2 passing (173ms)
	\end{lstlisting}
	
	De smoke test slaagt.
	
	\subsection{Testen in de Truffle console}
	Uiteraard dienen we  ook testen te schrijven voor de stemfunctionaliteit. Gezien dit onderdeel van de code echter nog aan veranderingen onderworpen zal worden, is het misschien voordeliger om de huidige werking op een andere manier te verifiëren. In plaats van een test te schrijven, kunnen we via de \textit{Truffle console} de toestand van de lokale blockchain bekijken.
	
	In de console navigeren we naar het project, vervolgens gebruiken we het console-commando:
	
	\begin{lstlisting}[numbers=none]
	> truffle console
	\end{lstlisting}
	
	Dit opent de Truffle-console. 
	Hier geven we het volgende commando in:

	\begin{lstlisting}[numbers=none]
	truffle(development)> Election.deployed()
		.then(function(i){app = i})
	\end{lstlisting}
	
	Indien het Election contract gedeployed is, wordt er een asynchrone callback-functie uitgevoerd.  De functie in kwestie krijgt  een instantie van Election (i) mee als parameter en maakt deze toegankelijk door ze in een variabele op te slaan. Eenmaal de asynchrone code uitgevoerd is, hebben we via \textit{app} toegang tot de attributen en functies van het smart contract Election.
	
	Gezien we de stemfunctionaliteit willen testen, zullen we de stemfunctie \textit{vote()} oproepen.
	
	Hiervoor hebben we echter het publieke adres van een van de Ganache-accounts nodig. Deze kan men vinden in Ganache zelf, of men kan ze bekomen via het Truffle console-commando:
	\begin{lstlisting}[numbers=none,language=bash]
	truffle(development)>web3.eth.getAccounts()
	[ `0xE03c3692FED9D4f2cBc7c5a30b05Ae9ce7b3b839',
	`0xe60c19f8a1baC541483e303Dc3d9B4e28d580980',
	`0x47B7B70802E9eC6a5Df8570962574894f0Ac4c15',
	`0xE4978a5B03C37245356c1dFdBBc536093aA4887d',
	`0xeA48794bF89B626392FF85ec9786671286B4a0c4',
	`0x4C57CA30fa4Ba94066bb65F180B87438566400b7',
	`0xb6997ce4fdE3C0D3002C6A6208AB60cd0CF2c1aA',
	`0xa9a71102f482960839b85d389C553fDa69e603F4',
	`0xd99Fe17d5Dc01aDfd691DA97E056979C8ADBefa7',
	`0xa12D538888E90A2DA7763Bd8F64595e568c3dFE0' ]
	\end{lstlisting}
	
	We kopiëren een adres naar keuze en vullen dit in als de \textit{from} property van het object dat de tweede parameter vormt. Op deze manier versturen we een transactie vanuit de console, in naam van het geselecteerde adres. De eerste parameter is het id van de `kandidaat' waarvoor we stemmen.
	\begin{lstlisting}[numbers=none,language=bash]
	truffle(development)> app.vote(1, 
		{ from: `0xE03c3692FED9D4f2cBc7c5a30b05Ae9ce7b3b839'}
	)
	\end{lstlisting}
	
	Ook hier krijgen we een \textit{transaction receipt}.
	Om te controleren of er een stem is bijgekomen voor de gekozen optie, geven we de volgende Truffle console-commando's in:
	\begin{lstlisting}[numbers=none]
	truffle(development)> app.candidates(1)
	truffle(development)> app.candidates(2)
	\end{lstlisting}
	Dit resulteert in:
	\begin{lstlisting}[numbers=none,language=bash]
	Result {												Result {
	...															...
	id: <BN: 1>,										id: <BN: 2>,
	name: `Yes',									  name: `No',
	votes: <BN: 1> }							  votes: <BN: 0> }
	\end{lstlisting}
	
	Voor de eerste optie (`Yes') zien we dat er onder het attribuut \textit{votes} de waarde <BN: 1> staat. Het aantal stemmen hier is met andere woorden 1.
	Voor de tweede optie (`No') is dat niet het geval, \textit{votes} heeft de  waarde <BN: 0>. Het aantal stemmen hier dus onveranderd, 0.
	
	Dit toont aan dat onze stemfunctie wel degelijk werkt! 
	
	Figuur \ref{fig:contracts-ganache3}: Ook in Ganache zien we dat er een nieuwe transactie heeft plaatsgevonden. Merk wel op dat de huidige implementatie totaal geen garantie biedt op het vlak van anonimiteit (aan de hand van de input van iedere transactie kunnen we het stemgedrag van iedere kiezer afleiden). Het enige wat de anonimiteit van een kiezer enigszins beschermt, is de abstracte aard van de accounts: de kiezer is alleen bekend via zijn Ethereum-adres. 
	\begin{figure}
		\includegraphics[width=\linewidth]{img/contracts-ganache3.png}
		\caption{Vote functie aangeroepen weergegeven in Ganache}
		\label{fig:contracts-ganache3}
	\end{figure}
	\newpage
	\section{Front-end applicatie}
	Eenmaal we een werkende back-end hebben, dienen we er natuurlijk ook nog een front-end applicatie voor te ontwikkelen. In deze handleiding zullen we gebruik maken van Angular (figuur \ref{fig:angular}), een open source javascript-framework dat ontwikkelaars in staat stelt performante \textit{single page applications} te schrijven. We verkiezen Angular omdat de structurering van de code bijzonder vast is in dit framework, waardoor de functionaliteiten mooi opgesplitst zijn en de ontwikkelervaring bijzonder aangenaam is.  Het staat de lezer natuurlijk vrij om een javascript-framework naar keuze te gebruiken. Het gaat hier immers niet zo zeer om de ontwikkeling van de front-end applicatie als om het leggen van de verbinding tussen het web en de Ethereum-blockchain...
	\begin{figure}[h!]
		\centering
		\includegraphics[width=\linewidth/3]{img/angular.png}
		\caption{Het Angular logo ~\autocite{Angular2019}}
		\label{fig:angular}
	\end{figure}
	\subsection{Opzet}
	Zoals gezegd beginnen we met een Angular project op te zetten, lezers die gebruik maken van een ander Javascript-framework kunnen de volgende stappen overslaan, zij hoeven enkel de vernoemde node\_modules te installeren en de code gegeven in de \textit{ElectionConnectorService} te herschrijven. 
	
	We beginnen met het installeren van de Angular cli, deze zal ons toestaan een Angular project te maken.
	
	We maken gebruik van het console-commando:
	\begin{lstlisting}[numbers=none,language=bash]
	>npm install -g @angular/cli
	\end{lstlisting}
	Dit installeert de Angular cli op een globaal niveau, vanaf nu kunnen we er overal gebruik van maken.
	
	We navigeren naar de directory waar we het front-end project willen aanmaken, vervolgens gebruiken we het console-commando:
	\begin{lstlisting}[numbers=none,language=bash]
	>ng new EthereumVote
	\end{lstlisting}
	De Angular cli zal nu een nieuw project voor ons creëren in de huidige directory. 
	
	Nu installeren we de nodige node\_packages:
	\begin{itemize}
		\item \textbf{web3.js}: een verzameling javascript libraries  die het mogelijk maken om te communiceren met Ethereum-instanties, hetzij via  een HTTP-, webSocket- of IPC-verbinding. Web3 bevat in feite alles wat nodig is om de front-end van een DApp te ontwikkelen.
		\item \textbf{truffle-contract}: maakt een lokale interface van het smart contract aan, waarmee we kunnen interageren in de front-end. Het aanmaken van de instantie gebeurt op basis van het gecompileerde smart contract in json-format.
	\end{itemize}

	\subsection{Mogelijke problemen met Web3}
	
	Na de installatie van web3.js is het mogelijk dat men problemen ervaart met verschillende node\_modules die niet kunnen gevonden worden. Voor lezers die problemen ervaren met onder andere de \textit{crypto}, \textit{streams} of \textit{http } modules is het aangeraden om een van de volgende oplossingen te proberen:
	\begin{itemize}
		\item Patch file voor de node\_modules
		\item Downgraden naar \textit{web3.js versie 0.20.5}. 
	\end{itemize}
	
	In deze handleiding raden we aan om de oplossing met het \textit{patchfile.js}\footnote{Broncode: https://gist.github.com/niespodd/1fa82da6f8c901d1c33d2fcbb762947d} als eerste te proberen:
	\begin{lstlisting}[numbers=none,language=bash]
	const fs = require("fs");
	const f = "node_modules/@angular-devkit/build-angular
	/src/angular-cli-files/models/webpack-configs/browser.js";
	fs.readFile(f, "utf8", function (err,data) {
		if (err) return console.log(err);
		var result = data.replace(/node: false/g, "node: {crypto: true, stream: true}");
		fs.writeFile(f, result, "utf8", function (err) {
			if (err) return console.log(err);
		});
	});
	\end{lstlisting}
	Deze code dient na iedere wijzing van de node\_modules te worden uitgevoerd, onder \textit{``scripts''} in de \textit{package.json} voegen we daarom toe:
	\begin{lstlisting}[numbers=none,language=bash]
	"scripts": {
	...
	"postinstall": "node patch.js"
	},
	\end{lstlisting}
	
	Enkel en alleen als de eerste optie niet werkt, raden we aan om web3 te downgraden. Het is de lezer aangeraden om over te schakelen naar yarn package manager ten voordele van npm. Yarn kan beter overweg met het toevoegen, verwijderen en onderhouden van dependencies. Ook voor gebruikers die geen verdere problemen meer ondervinden met web3, kan het daarom waardevol zijn om yarn te installeren. Yarn- en npm-commando's zijn helemaal gelijk, op de begin-keywords na. In de rest van deze handleiding wordt er gebruik gemaakt van yarn-commando's.
	
	Eenmaal yarn\footnote{Verkrijgbaar via https://yarnpkg.com/lang/en/docs/install} geïnstalleerd is, kunnen console-commando's gebruikt worden om web3 te herstellen:
	\begin{lstlisting}[numbers=none,language=bash]
	>yarn
	>yarn remove web3
	>yarn add web3@0.20.5
	\end{lstlisting}

	\subsection{Mogelijke problemen met Truffle-Contract}
	Lezers die problemen ondervinden bij de  installatie van de \textit{truffle-contract} module, installeren het best versie v4.0.20. Bij aanhoudende problemen kan men de nodige \textit{yarn.lock }en \textit{package.json} op de Github-pagina\footnote{Zie https://github.com/Ocean97Li/bachelorproef/blob/master/poc/EthereumVote/frontend/yarn.lock} van het project vinden. De meeste problemen met \textit{node\_packages} kunnen opgelost worden door deze twee configuratiebestanden in het lokale project te vervangen door de bestanden die op Github staan. Vervolgens dient men nog eens gebruik te maken van het yarn-commando.
	
	\subsection{Algmenene structuur EthereumConnector}
	Eenmaal de nodige zaken geïnstalleerd zijn, kunnen we beginnen met de implementatie van een stemsysteem. Zoals gezegd zal de focus hier voornamelijk liggen op het implementeren van de functionaliteiten in verband met  de Ethereum-netwerkverbinding. Voor de herbruikbaarheid en performantie worden alle functionaliteiten in een service gebundeld,  zoals gebruikelijk binnen Angular. De code voor de rest van de applicatie (voornamelijk de visuele componenten en verschillende schermen) wordt enkel kort besproken.  Eenmaal de service geïmplementeerd is, staat het de lezer natuurlijk vrij om er direct zelf mee aan de slag te gaan en er een applicatie rond te ontwikkelen.
	
	 We beginnen met het aanmaken van de \textit{EthereumConnectionService.} 
	
	We maken een nieuw bestand aan op de locatie \textit{services/ethereum-connector.service.ts} via de console-commando's:
	\begin{lstlisting}[numbers=none,language=bash]
	>mkdir services
	>cd services
	>ng generate service EthereumConnector
	\end{lstlisting}
	
	Dit geeft ons een service-template waarmee we kunnen starten met ontwikkelen.
	
	We overlopen eerste de grote functionaliteiten die we wensen te implementeren:
	\begin{itemize}
		\item \textbf{vote()}: Een stemfunctie die toestaat de gelijknamige functie in het smart contract \textit{Election} aan te spreken. Het id van de gewenste kandidaat moet als parameter worden mee gegeven.
		\item \textbf{register()}: Een stemfunctie die toestaat  de gelijknamige functie in het smart contract \textit{Election} aan te spreken. Ook hier moet het id van de gewenste kandidaat worden mee gegeven als parameter.
		\item \textbf{getVotes()}: Eenmaal een stem is uitgebracht, willen we de resultaten van de verkiezing visualiseren. Hiervoor moeten we het attribuut \textit{candidates} van \textit{Election} retourneren.
		\item \textbf{checkVotedYet()}: Functie die het \textit{votedVoters} attribuut van \textit{Election} aanspreekt en weergeeft of een kiezer al dan niet een stem heeft uitgebracht.
		\item \textbf{getCandidates()}: Haalt de kandidaten en hun \textit{id}'s op. De namen worden gebruikt om weer te geven, de id's vormen de achterliggende opties waarop men kan stemmen.
		\item \textbf{getAccountInfo()}: Controleert via web3 of er een Ethereum-account is ingelogd in de browser en of het account genoeg ether heeft om te registreren en en te stemmen.
		\item \textbf{truffleContract()}: Initieert een interface van \textit{Election} op basis van het gecompileerde smart contract. Via deze interface kunnen we  interageren met smart contracts.
	\end{itemize}

Daarnaast zijn er ook een heel aantal velden om verschillende zaken bij te houden. 

Tabel \ref{tab:ethservice} geeft een overzicht van de verschillende attributen die we zullen gebruiken om zaken bij te houden in de connectorService. 
Merk op dat er voor veel van de private domeinattributen (die te maken hebben met stemmen) gelijknamige publieke attributen van het type Subject zijn. Externe componenten kunnen de methode \textit{subcribe()} van deze attributen oproepen  om op de hoogte te worden gehouden van specifieke veranderingen aan de private domeinattributen van de connectorService. Via de Subjects kunnen we de veranderingen doorsturen naar de verschillende componenten die hiervan afhankelijk zijn.

	\begin{table}
		\centering
		\begin{tabular}{ |l|c|c|c| } 
			\hline
			\textbf{property naam} & \textbf{toegankelijkheid}&  \textbf{type}  \\
			\hline
			electionInstance 		& private & null \\ 
			web3Provider 			& private & number\\ 
			account 					  & private & string \\ 
			votingKey 					& private & string \\ 
			votes 							 & private & Map<string, number> \\ 
			candidates 					& private & Map<string, number>  \\ 
			refreshVotes\$ 			 & private & Observable<number>  \\ 
			state 							  & private & ElectionState  \\ 
			account\$  					& public & Subject<string> \\ 
			votes\$  					   & public & Subject<Map<string, number> > \\ 
			candidates\$ 			 & public & Subject<Map<string, number> > \\ 
			state\$ 						& public & Subject<ElectionState>  \\ 
			error\$ 						& public & Subject<string> \\ 
			\hline
		\end{tabular}
	\caption{Attributen en hun type binnen de EthereumConnectorService}
	\label{tab:ethservice}
\end{table}
\newpage
\subsection{Bespreking verschillende methodes}
	\subsubsection{truffleContract}
	\lstset{language=JavaScriptSolidity} 
	\begin{lstlisting}[numbers=none]
	import * as TruffleContract from 'truffle-contract';
	...
	const tokenAbi = require('../../../backend/build/contracts/Election.json');
	...
	get truffleContract() {
		if (!this.contract) {
			const contract = TruffleContract(tokenAbi);
			contract.setProvider(this.web3Provider);
			this.contract = contract;
		}
		return this.contract;
	}
	\end{lstlisting}
	De getter-functie \textit{truffleContract()} maakt met behulp van de  truffle-contract bibliotheek een nieuwe interface aan om het smart contract te creëren. De parameter \textit{tokenAbi} geeft het pad naar het gecompileerde \textit{Election.json} smart contract in de back-end. Merk ook op dat deze methode de \textit{lazy-loading} filosofie volgt: zolang het contract niet wordt aangeroepen, wordt het ook niet geïnitialiseerd.
	\subsubsection{getAccountInfo}
	\lstset{language=JavaScriptSolidity} 
	\begin{lstlisting}[numbers=none]
	private getAccountInfo() {
		return new Promise((resolve, reject) => {
			window.web3.eth.getCoinbase((err, account) => {
				if (err === null && account !== null) {
					this.account = account;
					window.web3.eth.getBalance(account, (error, balance) => {
						if (error === null && balance > 0) {
							return resolve();
						} else {
							return reject();
						}
					});
				} else {
					return reject();
				}
			});
		});
	}
	\end{lstlisting}
	 \textit{GetAccountInfo()} is de methode die wordt gebruikt voorafgaand aan iedere transactie om het Ethereum-account op te halen en te controleren. We werken hier (zoals meestal) met een ketting van opeenvolgende en geneste \textit{promises}. We beginnen bij het begin: de initialisatie van een nieuwe promise als retourwaarde. Wanneer men een promise declareert, zijn er twee zaken die in de callback-functie kunnen gebeuren:  ofwel wordt de methode \textit{reject()} aangeroepen en stopt de ketting van promises op dat punt, ofwel wordt \textit{resolve()} aangeroepen en gaat men verder naar de volgende promise.
	\subsubsection{getCandidates}
	\lstset{language=JavaScriptSolidity} 
	\begin{lstlisting}[numbers=none]
	public getCandidates() {
		this.truffleContract.deployed().then((instance) => {
			this.electionInstance = instance;
			return this.electionInstance.candidatesCounter();
		}).then((candidatesCounter) => {
			this.candidatesCounter = candidatesCounter;
			return this.findEachCandidateIds();
		}).then(() => {
			this.candidates$.next(this.candidates);
		}).catch((error) => {
			this.error$.next(error);
			console.log(error);
		});
	}
		
	private findEachCandidateIds() {
		return new Promise((resolve, reject) => {
			const promises = [];
			const number = this.candidatesCounter.toNumber();
			for (let i = 1; i <= number; i++) {
				promises[i] = this.electionInstance.candidates(i)
				.then((candidate) => {
					this.candidates.set(
						candidate[0].toNumber(),
						candidate[1]
					);
				});
			}
			Promise.all(promises).then(() => {
					resolve();
				});
			});
		}
		\end{lstlisting}
		Voor we effectief kunnen stemmen, moeten we weten uit welke opties kan gekozen worden. Tot dat doeleinde dienen de methodes \textit{getCandidates} en \textit{findEachCandidateIds}. Deze zorgen ervoor dat de mogelijke `kandidaten' opgevraagd worden van het smart contract. 
		
		De eerste stap is het ophalen van een nieuwe \textit{electionInstance}. Eenmaal dit voltooid is, bevragen we het smart contract voor de \textit{candidatesCounter}. Wanneer vervolgens ook dit getal binnen is, overlopen we iedere kandidaat in \textit{findEachCandidateIds()} aan de hand van de \textit{candidatesCounter}. We bewaren enkel de eerste twee attributen:  naam en id. Voor iedere kandidaat maken we een asynchrone call naar het smart contract. Het resultaat van iedere call (per kandidaat) slaan we op in een array van promises.
		
		Met \textit{Promise.all(promises)} verzekeren we ons ervan dat de promise die in \textit{findEachCandidateIds()} gemaakt wordt pas de \textit{resove()} uitvoert op het moment dat we een naam en id voor iedere kandidaat hebben.
		
		Tenslotte stellen we alle luisterende componenten via \textit{candidates\$} op de hoogte. 
	\subsubsection{checkVotedYet}
	\lstset{language=JavaScriptSolidity} 
	\begin{lstlisting}[numbers=none]
	public checkVotedYet() {
		return new Promise( (resolve, reject) => {
			return this.truffleContract.deployed().then((instance) => {
				this.electionInstance = instance;
				return this.electionInstance.hasVoted({from:this.account});
			}).then((voted) => {
				if (voted) {
					setTimeout( () => {
						window.alert('It seems you have already voted!');
						this.state = ElectionState.verified;
						this.state$.next(this.state);
					}, 2000);
				}
			}).then(() => {
				resolve();
			}).catch((error) => {
				this.error$.next(error);
				reject();
			});
		});
	}
	\end{lstlisting}
	In onze verkiezing is het uiteraard niet wenselijk dat een persoon meerdere malen kan stemmen. Om dit tegen te gaan, voorzagen we reeds een nodige check bij de vote-functie in de back-end. In de front-end willen we echter onnodig werk en transactiekosten voor de gebruiker vermijden door het aanroepen van de stemfunctionaliteit op bepaalde momenten te verhinderen. 
	
	Via de methode \textit{checkVotedYet()} kunnen we exact dat doen: we vragen het smart contract eenvoudigweg of de gebruiker in kwestie al gestemd heeft.
	\subsubsection{getVotes}
	\lstset{language=JavaScriptSolidity} 
	\begin{lstlisting}[numbers=none]
	private refreshVotes$: Observable<number> = timer(0, 2000);
	...
	public async getVotes() {
		this.truffleContract.deployed().then((instance) => {
			this.electionInstance = instance;
			return this.electionInstance.candidatesCounter();
		}).then((candidatesCounter) => {
			this.candidatesCounter = candidatesCounter;
			return this.findEachCandidateVotes();
		}).then(() => {
			this.votes$.next(this.votes);
		}).catch((error) => {
			this.error$.next(error);
			console.log(error);
		});
	}
	...
	this.refreshVotes$.subscribe(() => {
		this.getVotes().catch();
	});
	\end{lstlisting}
	Zodra een stem is uitgebracht, willen we in onze implementatie het (tussentijdse) resultaat van de verkiezing zien. Gezien er ondertussen mogelijks andere kiezers hun stem hebben uitgebracht, dienen we de resultaten regelmatig te hernieuwen. Tot dat doeleinde wordt de asynchrone functie \textit{getVotes()} gedefinieerd. Voor het regelmatig ophalen van stemmen gebruiken we \textit{refreshVotes\$}, deze property bevat een \textit{observable timer}. De callback-functie zal na het aanroepen van \textit{subscribe(callback)} elke twee seconden worden uitgevoerd. We roepen er \textit{getVotes()} op.
	\subsubsection{register}
	\lstset{language=JavaScriptSolidity} 
	\begin{lstlisting}[numbers=none]
	public register() {
		return new Promise((resolve, reject) => {
			return this.truffleContract.deployed().then((instance) => {
				this.electionInstance = instance;
				return this.getAccountInfo();
			}).then(() => {
				return this.electionInstance.hasRegistered({from: this.account});
			}).then((registered) => {
				if (registered) {
					this.state = ElectionState.loggedin;
					this.state$.next(this.state);
					throw Error('already registered');
				}
			}).then(() => {
				this.votingKey = (Math.floor(Math.random() * 99) + 1).toString();
				return this.electionInstance.register(window.web3.sha3(this.votingKey), {from: this.account})
					.then(() => {
						this.state = ElectionState.loggedin;
						this.state$.next(this.state);
					});
				}).then(() => {
					resolve();
				}).catch((error) => {
					this.error$.next(error);
					reject(error);
				});
		}).catch((error) => {
			if (error === 'already registered') {
				console.log(error);
			}
			this.error$.next(error);
		});
	}
	\end{lstlisting}
		Vooraleer gebruikers een stem mogen uitbrengen, wensen we dat ze zich registreren als kiezer. Bij de registratie genereren kiezers een unieke stemsleutel met een cijfer dat alleen zij kennen. Deze sleutel wordt doorgegeven aan het smart contract en daar opgeslagen. Wanneer gebruikers een stem moeten uitbrengen, wordt hen gevraagd om het geheime cijfer bekend te maken (commit-reveal pattern). In het smart contract zal er vervolgens gecontroleerd worden of de stemsleutel kan gereproduceerd worden met dat cijfer. Is dit het geval, dan weten we met zekerheid dat we met een legitieme kiezer te maken hebben, de stem van de kiezer wordt geregistreerd.
		
		De methode begint met het initialiseren van de \textit{electionInstance}, vervolgens wordt \textit{getAccoutInfo()} aangeroepen. Wanneer \textit{getAccoutInfo()} voltooid is, wordt de \textit{hasRegistered} op de \textit{electionInstance} aangeroepen. Als de gebruiker in kwestie reeds geregistreerd is, wordt er een error geworpen om de promise-ketting te doorbreken. Indien de gebruiker nog niet geregistreerd was, gaat de promise-ketting gewoon verder. Er wordt een stemsleutel voor de gebruiker gegenereerd. Dit gebeurt via het een random getal dat door de \textit{web3Provider} gehashed wordt met een \textit{sha3-hash} (de web3 sha3-hash is dezelfde als de \textit{keccak256-hash} in Solidtiy).
		
		De stemsleutel wordt meegegeven als parameter in \textit{register()}, de functie die wordt opgeroepen op de \textit{electionInstance}.
	\subsubsection{vote}
	\lstset{language=JavaScriptSolidity} 
	\begin{lstlisting}[numbers=none]
	public vote(candidate: string) {
		this.state = ElectionState.voted;
		this.state$.next(this.state);
		this.checkVotedYet().then(() => {
			if (this.state.valueOf() < ElectionState.verified.valueOf()) {
				this.truffleContract.deployed().then((instance) => {
					this.electionInstance = instance;
				}).then(() => {
					return this.getAccountInfo();
				}).then(() => {
					return this.electionInstance.hasRegistered({from: this.account});
				}).then((registered) => {
					if (!registered) {
						throw Error('not registered');
					}
					return this.electionInstance.hasVoted({from: this.account});
				}).then((voted) => {
					if (voted) {
						throw Error('already voted');
					}
					this.state = ElectionState.voted;
					this.state$.next(this.state);
					return this.electionInstance.vote(candidate, this.votingKey, {from: this.account});
				}).then(() => {
					this.state = ElectionState.verified;
					this.state$.next(this.state);
				}).catch((error) => {
					console.log(error);
				});
			}
		});
	}
	\end{lstlisting}
	Wanneer kiezers hun stem uitbrengen zijn er een heleboel zaken die gecontroleerd dienen te worden. 
	
	Als eerste controleren we via \textit{checkVoted()} of het huidige account al dan niet reeds gestemd heeft. Accounts die aan deze stap voldoen krijgen een melding en worden vervolgens direct naar de resultaten-pagina gestuurd.
	
	Via \textit{getAccountInfo()} wordt vervolgens de validiteit van het ingelogde account gecontroleerd (correct ingelogd en geen lege portefeuille). Accounts die niet aan deze controle voldoen worden geweerd. 
	
	Accounts die voor het eerst een stem mogen uitbrengen en over een geldig en ingelogd account beschikken, worden gecontroleerd op registratie. Niet-geregistreerde accounts worden geweerd, geregistreerde accounts worden nu als kiezers gezien.
	
	Een kiezer is iemand die aan alle voorwaarden voldoet. Deze persoon kan een stem uitbrengen via \textit{vote()}. Het geheime getal dat gebruikt werd om de stemsleutel te genereren tijdens het registreren (attribuut \textit{votingKey}), wordt nu samen met het id van de gekozen kandidaat als parameter meegegeven aan de \textit{vote()} functie van \textit{electionInstance}.
	\subsection{Veranderingen in de state}
	De attente lezer merkte  doorheen de code waarschijnlijk verschillende statements op, zoals \textit{this.state = ElectionState.voted} en \textit{this.state\$.next(this.state)}. Dit zijn wijzigingen aan de interne staat van de applicatie. We gebruiken \textit{this.state} om de lokale vooruitgang binnen het verkiezingsproces bij te houden, via \textit{this.state\$.next(this.state)} verwittigen we alle \textit{listeners} (inclusief de connector zelf) van de gemaakte verandering.
	
	De statussen delen het stemproces op in verschillende sequentiële stappen, we definiëren:
	\begin{lstlisting}[numbers=none]
	export enum ElectionState {
		notloggedin,
		loggedin,				
		voted, 						
		verified  					 
	}
	\end{lstlisting}
	De voornaamste reden waarvoor \textit{state} gebruikt wordt, is om de vooruitgang binnen het stemproces visueel weer te geven doorheen de verschillende schermen in de applicatie. Concreet gebruiken we de veranderingen in \textit{state} om de \textit{component routing} te besturen: bij iedere verandering van \textit{state} wordt de \textit{Angular Router} aangeroepen om naar een nieuw scherm te navigeren.
	
	Normaal gezien is het uitvoeren van \textit{routing} iets wat alleen ter hoogte van de componenten gebeurt, maar in dit geval (voor de volledigheid en herbruikbaarheid van de code) lijkt het beter om dit binnen de \textit{EthereumConnectorService} te implementeren. 
	
	Opdat de routing probleemloos zou werken buiten een component (zonder irregulier gedrag), dienen we gebruik te maken van een \textit{wrapping} van de code met \textit{ngZone.run()}.
	
	De code die zorgt voor automatische routing bij verandering van \textit{state} is de volgende:
	\begin{lstlisting}[numbers=none]
	this.state$.subscribe( state => {
		let route: string;
		switch (state) {
			case ElectionState.notloggedin:
			route = "login";
			this.register(); break;
			case ElectionState.loggedin:
			route = "vote"; break;
			case ElectionState.voted:
			route = "verify"; break;
			case ElectionState.verified:
			route = "results"; break;
		}
		this.router.navigateByUrl(`/${route}`);
	});
	\end{lstlisting}
	Voor iedere verandering in \textit{electionState} is er een corresponderende route.
	
	De routes \textit{(`login',`vote',`verify',`results')} definiëren en injecteren we in \textit{app.module.ts}:
	\begin{lstlisting}[numbers=none]
	const routes: Route[] = [
		{path: "vote", component: VotingComponent},
		{path: "results", component: ResultsComponent},
		{path: "login", component: LoginComponent},
		{path: "verify", component: VerifyComponent},
		{path: "**", component: LoginComponent}
	];
	
	@NgModule({
		...
		imports: [
			...
			RouterModule.forRoot(routes),
			...
		]
	})
	\end{lstlisting}
	\subsection{Componenten}
	Tot slot geven we een kort overzicht van de verschillende componenten die gebruikt worden in deze applicatie. Iedere component stemt overeen met één scherm in de DApp en één stap in het verkiezingsproces. 
	
	Merk op dat de onderstaande HTML5-code specifiek is geschreven binnen de context van het Angular Framework. Gebruikers die een zelf een front-end wensen te ontwikkelen via een ander framework of zonder framework, kunnen zich enkel op deze code baseren, maar niet letterlijk overnemen. 
	
	De volledige implementatie is te vinden op Github\footnote{https://github.com/Ocean97Li/bachelorproef/tree/master/poc/EthereumVote/frontend/src/app/components}.
	\newpage
	\subsubsection{Login}
	In \textit{app/components/login/login.html}
	\begin{lstlisting}[frame=single,language=HTML5]
	<div class="card">
		<h1>
			Log into and register your Metamask Account
		</h1>
		<div class="metamask">
			Waiting for login and register...
		</div>
	</div>
	\end{lstlisting}
	\subsubsection{Voting}
	In \textit{app/components/voting/voting.html}
	\lstset{language=HTML5}
	\begin{lstlisting}[frame=single]
	 <div class="card">
		<h1>
			Is blockchain voting feasible?
		</h1>
		<form 
			*ngIf="candidates" 
			[formGroup]="form" 
			(ngSubmit)="vote()">
			
			<p *ngFor="let candidate of candidates">
				<input 
					[id]="candidate.id" 
					type="radio" 
					formControlName="answer" 
					[value]="candidate.id"/>
				<label [for]="candidate.id">
					{{candidate.name}}
				</label>
			</p>
			
			<p class="center">
				<button 
					type="submit" 
					class="vote-button" 
					[disabled]="form.get("answer").value==null">
						Vote
				</button>
			</p>
			
		</form>
	<div>
	\end{lstlisting}
	\newpage
	\subsubsection{Verify}
	In \textit{app/components/verify/verify.html}
	\lstset{language=HTML5}
	\begin{lstlisting}[frame=single]
	<div class="card">
		<h1>
			Is blockchain voting feasible?
		</h1>
		<div class="verify">
			Verifying your vote...
		</div>
	<div>
	\end{lstlisting}
	\subsubsection{Results}
	In \textit{app/components/results/results.html}
	\lstset{language=HTML5}
	\begin{lstlisting}[frame=single]
	<div class="card">
		<h1>Is blockchain voting feasible?</h1>
		<form>
			<ng-container 
				*ngIf="votes && totalVotes() > 0; else fetchingVotes;">
				<h1>Results:</h1>
			</ng-container>
			<div class="votes">
				<div 
					class="yes transistion" 
					[ngStyle]="{width: "calc("+percentage("Yes")+" - 30px)"}">
						Yes
				</div>
				<i class="percentage">
					{{displayPercentage[0]}}%
				</i>
			</div>
			<div class="votes">
				<div 
					class="no transistion" 
					[ngStyle]="{width: "calc("+percentage("No")+" - 30px)"}">
						No
				</div>
				<i class="percentage">
					{{displayPercentage[1]}}%
				</i>
			</div>
			<div class="center total">
				{{totalVotes(votes)}} votes
			</div>
			<ng-template #fetchingVotes>
				<p class="center space-bottom">Fetching votes...</p>
			</ng-template>
		</form>
	<div>
	\end{lstlisting}
	\newpage
		\begin{figure}[h!]
		\centering
		\includegraphics[width=\linewidth]{img/login.png}
		\caption{Het `login' scherm}
		\label{fig:login}
	\end{figure}
	\begin{figure}[h!]
		\centering
		\includegraphics[width=\linewidth]{img/vote.png}
		\caption{Het `voting' scherm}
		\label{fig:vote}
	\end{figure}
	\begin{figure}[h!]
		\centering
		\includegraphics[width=\linewidth]{img/verify.png}
		\caption{Het `verify' scherm}
		\label{fig:verify}
	\end{figure}
	\begin{figure}[h!]
		\centering
		\includegraphics[width=\linewidth]{img/results.png}
		\caption{Het `results' scherm}
		\label{fig:results}
	\end{figure}
	\newpage
	\section{Eindresultaat}
	\subsection{Testen van Eindresultaat}
	Voordat we de werking van de volledige applicatie te testen, dienen we de onderstaande sequentiële stappen te ondernemen:
	\begin{enumerate}
		\item Opstarten van Ganache.
		\item Aanmaken van een nieuwe lokale blockchain-instantie in Ganache.
		\item Navigeren naar het back-end project in de console.
		\item Deployen van Smart contracts via console-commando \textit{truffle migrate} (eventueel \textit{- -reset}).
		\item Opstarten van Angular app .
		\item Inloggen op metamask.
		\item Importeren van accounts (geheime sleutels uit Ganache importeren in Metamask).
		\item Start manueel testen.
	\end{enumerate}
	Tenslotte kunnen we de applicatie testen als volgt:
	\begin{enumerate}
		\item Surfen naar de url waarop de app gehost wordt. (localhost:4200/ deployed instantie).
		\item Metamask: connecteren aan de blockchain via het ingelogde account.
		\item Metamask: `Register' transactie goedkeuren.
		\item DApp: stem uitbrengen op het stemscherm.
		\item Metamask: `Vote' transactie goedkeuren.
		\item DApp: resultaten van de verkiezing bekijken.
	\end{enumerate}
	Figuren \ref{fig:screen_login1}, \ref{fig:screen_login2}, \ref{fig:screen_voting1}, \ref{fig:screen_voting2}  en \ref{fig:screen_voting3} geven de verschillende stappen van het stemproces via de EthereumVoting app weer. Merk op: alle interacties met de blockchain worden direct geïnitialiseerd door de gebruiker vanuit de browser, maar moeten vervolgens ook nog eens bevestigd worden via de Metamask-plugin. De interacties gebeuren dus allemaal via een ingelogde Ethereum-wallet, wat het systeem volledig secuur maakt.
	\begin{figure}
		\centering
		\includegraphics[width=\linewidth/3]{img/screen_login.png}
		\includegraphics[width=\linewidth/3]{img/screen_metamask1.png}
		\caption{Inloggen op Metamask }
		\label{fig:screen_login1}
	\end{figure}
	\begin{figure}
		\centering
		\includegraphics[width=\linewidth/3]{img/screen_metamask2.png}
		\includegraphics[width=\linewidth/3]{img/screen_metamask3.png}
		\caption{Verbinden met het Ethereum netwerk en registreren}
		\label{fig:screen_login2}
	\end{figure}
	\begin{figure}
		\centering
		\includegraphics[width=\linewidth/3]{img/screen_metamask4.png}
		\includegraphics[width=\linewidth/3]{img/screen_voting.png}
		\caption{Stemmen, na de registratie-transactie voltooid is}
		\label{fig:screen_voting1}
	\end{figure}
	\begin{figure}
		\centering
		\includegraphics[width=\linewidth/3]{img/screen_verify.png}
		\includegraphics[width=\linewidth/3]{img/screen_metamask5.png}
		\caption{Stemtransactie goedkeuren en controleren}
		\label{fig:screen_voting2}
	\end{figure}
	\begin{figure}
		\centering
		\includegraphics[width=\linewidth/3]{img/screen_metamask6.png}
		\includegraphics[width=\linewidth/3]{img/screen_results.png}
		\caption{De verkiezingsresultaten, na de stem-transactie voltooid is}
		\label{fig:screen_voting3}
	\end{figure}

	Eenmaal de bovenstaande stappen zijn doorlopen is deze handleiding ten einde gekomen: we hebben met succes een werkend \textbf{blockchain-gebaseerd stemsysteem} gebouwd op basis van Ethereum smart contracts, en deze gedeployed op een lokale blockchain. De werking van de smart contracts hebben we uitvoerig getest, via test files en de Truffle-console. Vervolgens hebben we onze smart contracts verbonden aan een webapplicatie gebouwd met het Angular javascript framework. De werking van de volledige DApp hebben we tenslotte in de browser bevestigd! 
	
	\subsection{Evaluatie Eindresultaat}
	Deze handleiding toont aan dat het ontwikkelen van een blockchain-gebaseerd stemsysteem eenvoudiger is dan ooit. De vele tools en hulpmiddelen waarvan we vandaag kunnen gebruik maken, zorgen voor lagen van abstractie en vereenvoudiging, die van blockchain een ontwikkelomgeving maken zoals alle andere.

	 Het stemsysteem dat we via deze handleiding hebben bekomen, is simpel en efficiënt. Het project is geschreven volgens de laatste standaarden en kan bijgevolg op alle moderne Ethereum-(test)netwerken draaien. Het is een perfect systeem voor kleinschalige verkiezingsscenario's: zaken zoals burgerinitiatieven, buurtcomités of raden van bestuur. EthereumVote biedt een drastische vereenvoudiging voor de organisatie en gebruiksvriendelijkheid van verkiezingen: organisatoren kunnen met enkele clicks een nieuwe verkiezing opzetten, gebruikers kunnen met nog minder clicks hun stem uitbrengen.
	 
	 Voor grotere verkiezingsscenario's zoals gemeentelijke, provinciale of nationale verkiezingen, is EthereumVote minder geschikt. De aard van de blockchain-technologie werkt hier in zekere zin tegen. Iedere transactie, elke interactie tussen gebruiker en systeem is immers publiek zichtbaar. In veel verkiezingen is volledige anonimiteit echter noodzakelijk. Om volledige anonimiteit op een blockchain te bereiken, moet men dus een manier vinden om geheimen te bewaren in een systeem waar alles wat opgeslagen wordt openbaar is. Er zijn systemen - zoals het OVNP (McCorry e.a., 2017) - die er in slagen dit te bereiken. Dergelijke oplossingen steunen echter op complexe cryptografie en zijn bijgevolg zeer inefficiënt en vaak te duur om in de context van een blockchain-netwerk te draaien.
	
	In de plaats van cryptografie aan het project toe te voegen, stellen we met EthereumVote een alternatieve oplossing voor: we encrypteren niet, maar verzekeren dat de accounts die onze kiezers gebruiken strikt anoniem blijven. De identiteit (naam, voornaam, leeftijd of geboortedatum) van de kiezer mag nooit worden gelinkt aan het adres van zijn of haar account. We garanderen dit door zelf de accounts te distribueren vòòr de verkiezing van start gaat. De accounts worden op random wijze toegewezen aan de in aanmerking komende kiezers. De accounts zijn daarvoor reeds bekend gemaakt aan het smart contract in de vorm van een \textit{whitelist}: enkel en alleen de accounts die door het smart contract gekend zijn mogen aan de verkiezingen deelnemen. 
	
	Het uitdelen/verkopen van de accounts is in dit geval wel iets wat een bepaalde vorm van \textit{centrale autoriteit} vergt. De organisator van de verkiezing krijgt in zekere zin ook extra macht: deze persoon of groep van personen kan via het account immers bepalen wie er zal deelnemen aan de verkiezing. Deze  twee nadelen zijn echter aanvaardbaar, bovendien worden ze teniet gedaan door de reeds genoemde voordelen van EthereumVote.
	
	

	

	
	
	
	
	
	
	
	
	
	
	
	
	
	
	
	
	
	
	
	
	

	









%%=============================================================================
%% Conclusie
%%=============================================================================

\chapter{Conclusie}
\label{ch:conclusie}
In deze scriptie werd een onderzoek gevoerd naar de haalbaarheid van blockchain gebaseerd stemmen, daarnaast werd ook de benodigde informatie voor het ontwikkelen van een dergelijk systeem  verzameld. De bijdrage en \textit{proof of concept} van deze scriptie werden gegeven in de vorm van een praktische handleiding waarin een \textit{decentralized application} - ofwel DApp -  genaamd EthereumVote werd ontwikkeld. 

 Dit onderzoek is bedoeld voor iedereen die geïnteresseerd is in zaken als burgerparticipatie, verkiezingen en democratie. Hoewel vaak technisch van aard is het geschreven met niet technisch-onderbouwde personen in gedachten. Alleen de handleiding verondersteld een technische achtergrond. Ze is specifiek gericht op ontwikkelaars met interesse in blockchain stemsystemen. De meerwaarde is dat een compleet stappenplan gegeven wordt, met extra uitleg over gebruikte tools, complexe code, enzovoort. Waar andere handleidingen een louter functionele focus hebben, is er bij EthereumVote rekening gehouden met anonimiteit en schaalbaarheid. Encryptie en registratie werden daartoe als `extra' stappen gebruikt, resulterend in een gulden middenweg tussen absolute veiligheid en kost-efficiëntie.
 
In dit finale het hoofdstuk wordt er een antwoord geformuleerd op de onderzoeksvragen waarrond deze scriptie gebouwd werd:
\begin{itemize}
	\item Wat zijn de voor- en nadelen van de blockchain-technologie in het kader van een stemsysteem?
	\item Is er sprake van een onoverkomelijk schaalbaarheidsprobleem voor blockchain-technologie?
	\item Welke tools heeft men nodig om een blockchain-gebaseerd stemsysteem op te zetten en wat zijn de voor- en nadelen hiervan?
	\item Is een blockchain-gebaseerd stemsysteem haalbaar in de praktijk?
\end{itemize}
\section{Verwachte resultaten en conclusies versus realiteit}
Bij aanvang van dit onderzoek werden resultaten verwacht gelijkaardig aan het onderzoek van \textcite{McCorry2017}. De verwachting was immers dat het werk van dit onderzoek beschikbaar zou zijn als code-bibliotheek. Tijdens de implementatie van EthereumVote bleek dit echter niet het geval, er kon dus ook niet op verder gebouwd worden. 

Veranderingen in de onderliggende technologie leidden dit onderzoek bovendien tot de conclusie dat een meer performante oplossing nodig was. De zeer simpele implementatie van \textcite{McCubin2019} werd daartoe gecombineerd met concepten uit het werk van \textcite{McCorry2017}. EthereumVote is dus niet het verwachte resultaat, het is een overwogen compromis: meer performantie in de ruil voor verlaagde veiligheid op vlak van anonimiteit. 

De algemene conclusies van dit onderzoek komen grotendeels overeen met de initiële verwachtingen.  Blockchain gebaseerde stemsystemen lijken inderdaad over een groot potentieel te beschikken maar worden - ook zoals verwacht - voornamelijk gehinderd  door een schaalbaarheidsprobleem. Dit onderzoek bevestigd hiermee ook de conclusies van zijn voorgangers. Het zijn conclusies die voorzichtig positief kunnen genoemd worden. Het beantwoorden van de onderzoeksvragen leverde deze scriptie bovendien ook enkele nieuwe inzichten op, deze zijn te lezen in de volgende secties.
\section{Voordelen en nadelen}
De vele voordelen van blockchain gebaseerde stemsystemen kwamen aan bod in sectie \ref{sec:blockchain-gebaseerd-stemmen}. Samengevat zijn blockchain gebaseerde stemsystemen  - indien correct geïmplementeerd - beter op het vlak van \textit{privacy}, \textit{transparantie},  \textit{fouttolerantie},  \textit{veiligheid} en \textit{correctheid} dan hedendaagse elektronische tegenhangers. Decentralisatie speelt daarbij een significante rol. Blockchain gebaseerde systemen hebben dus het potentieel om beter te zijn, maar zijn dit niet automatisch.
	
Verder aansluitend op het vorige punt, is het feit dat veel zogenaamde nadelen aan blockchain gebaseerd stemmen heel implementatie gebonden zijn. Sommige implementaties benutten bijvoorbeeld al het volle potentieel van blockchain als technologie, andere doen dit dan weer niet. 

In sectie \ref{sec:blockchain-gebaseerd-stemmen} lag de focus vooral op algemene, recurrente problemen. Een van de hoofdreden die  wordt gegeven is dat er te weinig ondersteuning is op bestaande blockchain platformen voor de ontwikkeling van stemsystemen. Er is voornamelijk een gebrek aan cryptografische ondersteuning, dat was ook de bevinding tijdens de \textit{proof of concept} van deze scriptie.
	
Om die reden lijkt het logisch dat men (voor grootschalige verkiezingsscenario's) een eigen blockchain netwerk opzet. Het grote nadeel hier is echter dat dit bijzonder kostelijk is, zowel op vlak van nodige infrastructuur als het verzamelen van technische kennis.
	
Tenslotte is er ook nog een veelbesproken probleem met blockchain en schaalbaarheid.
\section{Onoverkomelijk schaalbaarheidsprobleem}
 lijkt er sprake te zijn van een schaalbaarheidsprobleem voor blockchain technologie . Het is een probleem dat de ingebruikname van blockchain technologie in verschillende velden verhindert, ook in de context van stemsystemen is dit het geval. Het probleem komt voort uit de structuur van blockchains. 
 
 Zoals besproken in sectie \ref{sec:blockchain-gebaseerd-stemmen} zijn blockchains intrinsiek niet schaalbaar: De snelheid waaraan het toevoegen van blokken aan een keten via het \textit{proof of work} algoritme gebeurt zal altijd het aantal transacties dat per seconden kan worden verwerkt limiteren. 
 
 De complexiteit van \textit{proof of work} verlagen is vaak niet wenselijk gezien dit vaak tot minder decentralisatie leidt en dus de blockchain minder veilig maakt. Zonder twijfel is er sprake dus van een serieus schaalbaarheidsprobleem.
	
De vraag is nu of het schaalbaarheidsprobleem onoverkomelijk is, naar de toekomst toe. 
	
Het is onmogelijk om thans met zekerheid een antwoord op die vraag te bieden. Het is duidelijk dat er pogingen ondernomen worden om het probleem te trotseren: in sectie \ref{sec:ethereum-en-smart-contracts} bespraken we bijvoorbeeld Ethereum's oplossing in de vorm van een nieuw mining algoritme genaamd \textit{proof of stake}. Bitcoin lijkt het over een andere boeg te gooien met een oplossing in de vorm van \textit{off-chain} transacties. De toekomst zal moeten uitwijzen of deze inspanningen het probleem effectief zullen oplossen.
\section{Tools en ontwikkeling}
	
In deze scriptie werd EthereumVote voorgesteld als een zelf-geimplementeerd blockchain gebaseerd stemsysteem. De verschillende tools nodig voor de ontwikkeling van deze DApp kwamen aanbod in sectie \ref{sec:benodigdheden}, de mogelijke problemen die men ermee kan ondervinden werden besproken in hoofdstuk \ref{ch:methodologie}.
	 
Voor Truffle en Ganache is er nauwelijks van nadelen te spreken, beide tools werken uitstekend samen en stellen ontwikkelaars instaat om de werking van hun DApps op een lokale blockchain te simuleren. 

Ook bij de front-end tools Metamask en truffle-contracts zijn er niets dan voordelen. Metamask is een Ethereum wallet die ons instaat stelt om met het netwerk te interageren zonder zelf een node te zijn. Truffle contracts stellt ons instaat om een interface van het smart-contract aan te spreken in de front-end. 

De enige noemenswaardige problemen zijn te vinden bij web3, in de vorm van slecht onderhouden dependencies. In hoofdstuk \ref{ch:methodologie} van deze scriptie werden de daaruit volgende comptabiliteitsproblemen beschreven. Hoewel dit zeker een minpunt is,  blijft web3 een essentiële tool voor de ontwikkeling van DAppps: met enkele lijnen code verbindt het de front-end applicatie met het Ethereum netwerk.
\section{Haalbaarheid}
Kleinschalige blockchain stemsystemen zijn technisch volledig haalbaar, de vele praktijk voorbeelden die werden aangehaald in sectie \ref{sec:blockchain-gebaseerd-stemmen} bevestigen dit. EthereumVote, voorgesteld in hoofdstuk \ref{ch:handleiding} van deze scriptie toont bovendien aan dat de ontwikkeling van een veilig en efficiënt stemsysteem niet complex of kostelijk hoeft te zijn.
	
De inherente schaalbaarheidsproblemen en het gebrek aan absolute veiligheid (op het vlak van anonimiteit) maken blockchain momenteel niet geschikt als onderliggende technologie voor grootschalige verkiezingen zoals we die kennen in de electorale politiek. Zelfs al zouden alle schaalbaarheidsproblemen van de technologie verholpen worden, dan nog zou de adaptatie in politieke verkiezingen erg moeilijk zijn.  Momenteel is er al veel kritiek op elektronisch stemmen, voor nieuwe technologieën zoals blockchain lijkt er alleen nog meer wantrouwen te heersen. 
	
Zolang er duidelijke nadelen verbonden blijven aan blockchain, kan de technologie niet doorbreken in de context van verkiezingen.  Zelfs als men zou kunnen concluderen dat een bepaald blockchain gebaseerd stemsysteem in zijn geheel voordeliger is dan de huidige elektronische en papieren stem (dewelke minder `perfect' zijn dan ze vaak worden uitgemaakt), dan nog blijft de ingebruikname zo goed als onmogelijk.
	
Het lijkt erop dat een praktisch onfeilbaar stemsysteem gepresenteerd moet worden alvorens blockchain zelfs maar in aanmerking kan komen als oplossing vanuit het politieke en juridische standpunt. Zo'n onfeilbaar systeem is volgens dit onderzoek met de huidige blockchain technologie moeilijk, als niet onmogelijk.  Er zou infeite een nieuwe, speciale variant van blockchain technologie moeten ontwikkelt worden. Projecten zoals het Moscow's Citizen's Initiative tonen immers aan dat er met voldoende politieke wil is en de juiste economische middelen al veel meer mogelijk is. 
	
De conclusie is dus dat grootschalige blockchain gebaseerde stemsystemen voorlopig niet haalbaar zijn en dat ook niet plots zullen worden. Potentieel naar de toekomst toe is er wel, alleen is het de vraag of de omstandigheden  (politiek, economisch, technologisch) het benutten er van zullen toelaten. Mocht dit het geval zijn dan ziet deze scriptie de toekomst van blockchain gebaseerde stemsystemen als een positieve zaak.
	
Verder onderzoek zou kunnen uitwijzen hoe groot het potentieel precies is. Concreet zou er een studie kunnen gevoerd worden naar het financiële aspect, waarin de totale kostprijs voor de opzet van een stemsysteem als ook het energie verbruik in operatie wordt berekent. Een andere studie die gevoerd zou kunnen worden is een vergelijkend onderzoek naar de verschillende manieren om anonimiteit   voor gebruikers te creëren. Het zou bijvoorbeeld bijzonder interessant zijn om verkiezingen op basis van hoogstaande cryptografische protocollen zoals OVNP en BroncoVote simultaan naast elkaar te draaien en de resultaten te vergelijken.
	
%% onderzoeksvra(a)g(en). Wat was jouw bijdrage aan het onderzoeksdomein en
%% hoe biedt dit meerwaarde aan het vakgebied/doelgroep? Reflecteer kritisch
%% over het resultaat. Had je deze uitkomst verwacht? Zijn er zaken die nog
%% niet duidelijk zijn? Heeft het onderzoek geleid tot nieuwe vragen die
%% uitnodigen tot verder onderzoek?



%%=============================================================================
%% Bijlagen
%%=============================================================================
\appendix
%%---------- Onderzoeksvoorstel -----------------------------------------------

\chapter{Onderzoeksvoorstel}

Het onderwerp van deze bachelorproef is gebaseerd op een onderzoeksvoorstel dat vooraf werd beoordeeld door de promotor. Dat voorstel is opgenomen in deze bijlage.

% Verwijzing naar het bestand met de inhoud van het onderzoeksvoorstel

%---------- Inleiding ---------------------------------------------------------

\section{Introductie} % The \section*{} command stops section numbering
\label{sec:introductie}Het democratisch stemmen is een cruciaal beslissingsmechanisme aanwezig in iedere laag van onze moderne samenleving, het vormt niet alleen de basis van het politieke systeem in vele landen, ook de bedrijfswereld is er van doordrongen. Stemprocessen zijn niet meer weg te denken uit de organisaties van vandaag. Of het nu gaat over besluiten op laag, functioneel niveau of over strategische beslissingen op hoog bestuurlijk niveau, de verantwoordelijkheid voor het maken van keuzes ligt steeds vaker bij een groep van mensen via een stemproces, dan bij één enkel individu. Het belang van stemmen op zowel economisch als politiek vlak valt dan ook  niet te onderschatten. De vraag die we onszelf echter moeten stellen is de volgende: hoe houdt men een proces waar zoveel van afhangt betrouwbaar, veilig en eerlijk? 
In de meeste gevallen blijkt het antwoord op die vraag vrij eenvoudig: waar het aantal participanten klein is, kan er gemakkelijk door iedere deelnemer of observator geverifieerd worden wat het resultaat van de stemming is. Iedere individuele deelnemer kan getuigen dat alles correct verloopt en dat maakt de kans op frauduleuze praktijken veel kleiner. Wordt het aantal participanten echter groter, dan is een dergelijk systeem onmogelijk. In zo'n geval bepaalt men het resultaat van de stemming door middel van een zogenaamde centrale autoriteit. Deze derde partij voert een controlerende functie uit en garandeert de betrouwbaarheid voor alle participanten. Bij nationale verkiezingen is dit bijvoorbeeld het stembureau. 
Het is van fundamenteel belang dat de centrale autoriteit neutraal is. Eén van de gevaren bij  dergelijke instanties is namelijk het risico op machtsmisbruik. Zo zijn voorafbepaalde of frauduleuze verkiezingen -hoewel bij ons gelukkig zeldzaam- niet ongekend. Een ander potentieel gevaar bij grote stemprocessen zijn onmoedwillige fouten. Er kunnen tal van menselijke of technische fouten optreden waardoor het resultaat van de stemming niet of niet geheel correct is. Een persoon die stemmen moet tellen kan fouten maken of een machine die stemmen moet registreren kan het laten afweten. Er is met andere woorden een probleem op het vlak van betrouwbaarheid. 
Een oplossing die een betrouwbaarder stemresultaat  garandeert lijkt de liggen in het digitaliseren van het proces. Bij digitalisatie krijgt men echter te maken met nieuwe risico's op het vlak van veiligheid, denk aan digitale fenomenen zoals: hacking, cyberaanvallen en identiteitsdiefstal. Toch zijn de voordelen die zijn verbonden aan digitalisatie immens: niet alleen op vlak van betrouwbaarheid, maar ook op vlak van efficiëntie, effectiviteit en gebruiksvriendelijkheid. Digitalisatie combineert zich echter zeer moeilijk met een centrale autoriteit: om het stemproces te digitaliseren zou men eigenlijk van het systeem van een centrale autoriteit moeten afstappen. 
Een mogelijkheid biedt zich aan in de vorm van de Blockchain technologie. Deze werd in feite net voor dit soort van problemen ontwikkeld. De oorsprong van deze technologie is namelijk gelinkt aan het ontstaan van 's werelds eerste gedecentraliseerde digitale munteenheid: de bitcoin. 
Net zoals dat bij het stemproces het geval is, wordt de legitimiteit van een reguliere munteenheid bepaald door een centrale autoriteit, in dit geval is dat de bank. Iedere (elektronische) transactie van geld gebeurt via deze autoriteit. De bitcoin is echter niet gelinkt aan een centrale bank, haar legitimiteit komt van een systeem dat blockchain wordt genoemd. De werking daarvan is in essentie niets meer dan de terugkeer naar een gedecentraliseerd systeem waarin iedere participant aan verificatie doet. 
Dit concept vormt de basis voor een systeem met een zeer hoog niveau van veiligheid en betrouwbaarheid. Omdat net dit de zaken zijn die bij stemprocessen van belang zijn, is een stemapplicatie op basis van blockchain zeker het onderzoeken waard.  De intentie van deze scriptie is dan ook antwoord te kunnen bieden op de volgende onderzoeksvragen:
\begin{itemize}
  \item Wat zijn de voor- en nadelen van de blockchain-technologie in het kader van een stemsysteem?
  \item Is er sprake van een onoverkomelijk schaalbaarheidsprobleem voor blockchain-technologie?
  \item Welke tools heeft men nodig om een blockchain-gebaseerd stemsysteem op te zetten en wat zijn de voor- en nadelen hiervan?
  \item Is een blockchain-gebaseerd stemsysteem haalbaar in de praktijk ?
\end{itemize}

%---------- Stand van zaken ---------------------------------------------------

\section{State-of-the-art}
\label{sec:state-of-the-art}

\subsection*{Blockchain 1.0: Bitcoin}
In dit eerste deel van de state-of-the-art wordt er een beeld geschetst van wat blockchain is en hoe het werkt achter de schermen. Dit wordt aan de hand van blockchain 1.0 gedaan, ofwel de originele blockchain-bitcoin-implementatie.
\subsubsection*{Wat is blockchain?}
 Blockchain is een gedecentraliseerde manier om gegevens op te slaan die voor het eerst geconceptualiseerd werd in de jaren negentig  ~\autocite{Dai1998} maar pas echt groot werd na de publicatie van ~\textcite{Nakamoto2008}. Het bijzondere aan de bitcoin-blockchain-implementatie is dat het peer-to-peer technologie combineert met wiskundige cryptografie, om tot een systeem te komen dat kan functioneren zonder de hulp van een vertrouwde derde partij  (centrale autoriteit) om de legitimiteit van iedere bitcoin te verifiëren. Blockchain werkt namelijk niet op basis van vertrouwen maar op basis van wiskundig bewijs.
 
 \subsubsection*{Hoe werkt het?}
Een blockchain is in datastructuur die bestaat uit een ketting van informatieblokken. Het is een ketting die fungeert als een soort publiek gedistribueerde boekhouding. Eenmaal informatie is opgeslagen in een blockchain, is het vrijwel onmogelijk om deze ongedetecteerd te wijzigen. Ieder blok van de ketting bestaat namelijk uit 3 velden: de data, de hash van het blok en de hash van het vorige blok. In het dataveld kan informatie worden opgeslagen. Het soort informatie dat men hier vindt is afhankelijk van de specifieke implementatie. In een bitcoin-blockchain wordt bijvoorbeeld een bitcoin-transacties opgeslagen. Ieder blok heeft daarnaast ook een hash, dit is een unieke identifier die wordt gegenereerd op basis van de inhoud van het blok. Iedere verandering aan het dataveld brengt ook een wijziging van de hashcode teweeg. Tenslotte is er nog het derde veld dat de hashcode van het vorige blok bijhoudt. Als iemand de data in één blok zou wijzigen, dan zou dit zorgen voor een aanpassing van de hash van dat blok. Het volgende blok bevat dan echter nog steeds een verwijzing naar de oorspronkelijke hash, die plots niet meer bestaat. De hashcode van het volgende blok kan dus niet  correct berekend worden. Ook de hashcodes van al de daarop volgende blokken kunnen bijgevolg niet meer correct berekend worden. Het gevolg is dus dat na een gecorrumpeerd blok ieder volgend blok ongeldig wordt. 

Dit systeem volstaat echter niet om te voorkomen dat aanpassingen worden gedaan. Moderne computers kunnen immers zodanig snel hashes berekenen, dat iemand die data in een bepaald blok aanpast mogelijks in no-time alle hashes van de daarop volgende blokken opnieuw zou kunnen berekenen, om zo de volledige blockchain weer valabel te maken, en dat alles voordat iemand dat zou kunnen opmerken. Om dit tegen te gaan werkt het systeem met een zogenaamde proof-of-work, dit is een mechanisme dat het berekenen van de hashes aanzienlijk lastiger en  langduriger maakt. In het geval van de bitcoin duurt het ongeveer 10 minuten om een hash te berekenen. Een nieuw blok met data aan de ketting toevoegen duurt dus ook 10 minuten, gezien  de hashcode voor het  nieuwe blok moet worden gevonden. Het zoeken van de correcte hash wordt gedaan door speciale nodes binnen het netwerk die miners worden genoemd. Miners specialiseren in het berekenen van hashcodes en kunnen wat verdienen wanneer ze de hash voor een bepaald blok als eerste kunnen vinden. In het geval van de bitcoin-blockchain verdienen miners bijvoorbeeld een bedrag in bitcoin. Dit hele systeem maakt het bijna onmogelijk om aanpassingen te doen aan reeds bestaande blokken. 

Blockchain heeft daarenboven nog een extra vorm van beveiliging. In plaats van een gecentraliseerde blockchain bij te houden, is het systeem gedistribueerd. Het maakt gebruik van een P2P-netwerk waarvan iedere node een eigen versie van de blockchain bijhoudt. Wanneer een nieuwe blok aan de blockchain wordt toegevoegd, valideert iedere node van het netwerk deze individuele blok. Blockchain werkt met een consensus van meer dan 50\%. Dat betekent dat een nieuwe blok enkel kan worden toegevoegd aan de blockchain wanneer meer dan de helft van de nodes deze valabel verklaart, als dit niet het geval is wordt de nieuwe blok afgewezen. Een dataveld wijzigen in een blockchain betekent in de praktijk dat men alle opeenvolgende hashes opnieuw moet berekenen, wat ontzettend lang duurt door de proof-of-work, en  gedurende  al die tijd moet men meer dan 50\% van de nodes in een netwerk controleren om een fake consensus te creëren bij iedere toevoeging aan de blockchain. Het spreekt voor zich dat dit een zeer moeilijke opgave is waardoor blockchain praktisch feilloos is op het vlak van veiligheid en betrouwbaarheid.

\subsection*{Blockchain 2.0}
In dit tweede deel van de state-of-the-art wordt een beeld geschetst van wat blockchain 2.0 inhoud volgens ~\textcite{Swan2015}. Concreet wordt er dieper ingegaan op het aspect Smart Contracts, vervolgens wordt er ook een overzicht gegeven van Ethereum.

\subsubsection*{Wat is zijn smart contracts?}
Smart contracts leiden de blockchain-technologie een stap verder. In ~\textcite{Swan2015} worden ze omschreven als gedecentraliseerde contracten die niet langer een autoriteit (zoals een rechtbank) nodig hebben. Het zijn in feite digitale contractprogramma's die zichzelf kunnen valideren en uitvoeren wanneer aan bepaalden voorwaarden is voldaan. Ook dit concept bestaat al sinds de jaren negentig ~\autocite{Szabo1996}. Bij het lezen van ~\textcite{Nakamoto2008} is het duidelijk dat er vanaf het prille begin van de bitcoin een visie was om een dergelijke systeem te implementeren. Blockchain-technologie staat immers niet alleen toe om data op te slaan, ook programma's kunnen in de blockchain worden opgeslagen. Smart contracts vormen de basis van de nieuwe Blockchain 2.0 van vandaag, zowat iedere grote crypto currency implementeert ze.

\subsubsection*{Wat is ethereum?}
Ethereum is een opensourceplatform dat werd opgericht in 2015. Net zoals bitcoin maakt het gebruik van een decentraal netwerk, gebaseerd op het oorspronkelijke blockchain-concept. Het valideren van informatie gebeurt ook hier door zogenaamde miners, het enige verschil met bitcoin is dat de miners worden beloond met de munteenheid ether in plaats van bitcoin. Ethereum kan men echter niet zien als een zuivere variant op de bitcoin of een andere crypto currency. Ethereum is veel meer dan dat, ~\textcite{Swan2015} omschrijft het als een "Turing-Complete Virtual Machine", die zowel een platform als een programmeertaal biedt voor het ontwikkelen en publiceren van gedistribueerde applicaties. Turing -compleetheid betekent in deze context dat het over een platform gaat dat het vermogen heeft om eender welke digitale munt, protocol of blockchain te ondersteunen.

\subsubsection*{Hoe werkt het?}
Zonder in teveel technisch detail te gaan, zou men kunnen stellen dat de werking van het ethereum-netwerk in grote lijnen dezelfde structuur volgt als de eerder omschreven Blockchain 1.0. In tegenstelling tot (relatief) eenvoudige bitcoin-transacties, bevat de data die bij Ethereum wordt opgeslagen in de blokken iets wat men zou kunnen omschrijven als een toestandsmachine, bestaande uit een lijst van allerhande transacties. Om misbruik en spamming van transacties tegen te gaan is aan iedere transactie een kleine kostprijs verbonden, in ether genaamd  \emph{gas}. Transacties in ethereum vinden plaats tussen twee gebruikers of tussen een gebruiker en een smart contract. Een transactie tussen een gebruiker en een smart contract kan één of meerdere nieuwe transacties vanuit het contract naar andere gebruikers doen ontstaan. Tenslotte kan een transactie van een gebruiker naar een smart contract ook transacties naar andere smart contracts triggeren, die dan op hun beurt hetzelfde doen en zo complexe kettingreactie creëren. De combinatie van mogelijke transacties en mogelijkheden met smart contracts zorgt volgens ~\textcite{Wood2017} voor een systeem waarop in theorie iedere toepassing mogelijk is. 
\subsection*{Blockchain 3.0}
\label{sec:wat-is-blockchain}

In dit derde en laatste deel van de state-of-the-art wordt eerst een overzicht gegeven van Blokchain 3.0. Vervolgens wordt een literatuurstudie gevoerd naar het concrete onderwerp van deze scriptie: namelijk het technisch opzetten van een klein stemsysteem op basis van een ethereum-blockchain.

\subsubsection*{Nieuwe toepassingen voor blokchain-technologie}

Platforms zoals ethereum  stellen de blockchain-technologie beschikbaar voor iedere ontwikkelaar. Het is nu niet langer een vereiste om een uitgebreide kennis van wiskundige cryptografie of gedistribueerde computersystemen te hebben om zo'n toepassing te ontwerpen. De impact van toekomstige blockchain-systemen is zeker niet te onderschatten. Blockchain zou niet alleen in stemprocessen een grote rol kunnen spelen, ook voor gezondheidszorg, justitie en anticensuur ontstaan implementaties. ~\textcite{Swan2015} stelt dat met het ontplooien van Blockchain 3.0 het potentieel zodanig groot is, dat er sprake zou zijn van een nieuw computer-paradigma, een technologische sprong die de samenleving op alle vlakken kan veranderen. Bronnen zoals ~\textcite{Wood2017} zijn iets minder optimistisch: hoewel zij de potentie van een wijdbeschikbare blockchain-technologie zeker niet ontkennen, blijven zij voorlopig sceptisch omwille van de schaalbaarheidsproblemen waarmee de technologie vandaag de dag kampt.

\subsubsection*{Literatuurstudie voor technische opzet}

De voornaamste bron voor het technische aspect van dit onderzoek is zonder enige twijfel ~\textcite{McCorry2017}. In deze paper werd een kleinschalig stemsysteem gerealiseerd op basis van de ethereum-blockchain. De implementatie telde 40 stemmers en werd getest op het officiële ethereum-testnetwerk. Er werd gekozen voor een implementatie op basis van een smart contract, de focus van ~\textcite{McCorry2017} lag op het uitwerken van een protocol dat het tellen van stemmen kan laten verlopen zonder een centrale autoriteit. Concreet gaat het hier dus over de implementatie van een zelftellend stemprotocol. Eén van de grote moeilijkheden tijdens het onderzoek was het garanderen van de anonimiteit van de stemmers. Er zijn voor een dergelijk systeem op het internet verscheidene andere implementaties vindbaar, maar op het vlak van anonimiteit  laten deze allemaal veel te wensen over. De opzet van dit onderzoek slaagde echter: het onderzoek resulteerde in het eerste decentraal online stemprotocol op basis van blockchain. Hoewel het protocol geen absolute anonimiteit bereikt, is er wel sprake van maximale privacy en afscherming van de gegevens van de gebruiker.

%---------- Methodologie ------------------------------------------------------
\section{Methodologie}
\label{sec:methodologie}

Dit onderzoek zal op gelijkaardige manier te werk gaan als het ~\textcite{McCorry2017} onderzoek. Aan de hand van een technische implementatie zal een praktische gids worden opgesteld met best practices, voorkomende problemen, valkuilen, enz. Een belangrijk verschil met het genoemde onderzoek is dat in dit onderzoek niet op voorhand assumpties worden gemaakt over de haalbaarheid  wat betreft de schaalbaarheid, en het zich dus niet tot dezelfde schaal zal beperken als ~\textcite{McCorry2017}. De hoop is dat -na bijna 2 jaar van verbeteringen en updates van de technologie- nu meer zaken mogelijk zijn.

%---------- Verwachte resultaten ----------------------------------------------
\section{Verwachte resultaten}
\label{sec:verwachte_resultaten}

Het verwachte resultaat van dit onderzoek is dat er een praktische gids kan worden uitgewerkt voor het opzetten van een blockchain-gebaseerd stemsysteem dat draait op ethereum. Daarnaast verwacht dit onderzoek ook een kleine bijdrage te kunnen leveren door een kleine verbetering aan te brengen op het vlak van schaalbaarheid. Het werk van ~\textcite{McCorry2017} verbeteren op het vlak van privacy verwacht dit onderzoek niet te doen. 

%---------- Verwachte conclusies ----------------------------------------------
\section{Verwachte conclusies}
\label{sec:verwachte_conclusies}

De verwachte conclusie van dit onderzoek is in de eerste plaats dat de schaalbaarheid van een blockchain-systeem nog steeds de grootste hindernis vormt voor een bredere toepassing op het vlak van stemsystemen. Deze conclusie wordt verwacht omdat de meeste andere onderzoeken betreffende dit onderwerp tot diezelfde conclusie kwamen. Desalniettemin verwacht dit onderzoek door positieve resultaten toch een positieve conclusie te kunnen trekken over de toekomst van blockchain-technologie inzake stemsystemen.



%%---------- Andere bijlagen --------------------------------------------------
% TODO: Voeg hier eventuele andere bijlagen toe
%\input{...}

%%---------- Referentielijst --------------------------------------------------

\printbibliography
\addcontentsline{toc}{chapter}{\textcolor{maincolor}{\IfLanguageName{dutch}{Bibliografie}{Bibliography}}}

\end{document}
